%% Generated by Sphinx.
\def\sphinxdocclass{report}
\documentclass[letterpaper,10pt,english]{sphinxmanual}
\ifdefined\pdfpxdimen
   \let\sphinxpxdimen\pdfpxdimen\else\newdimen\sphinxpxdimen
\fi \sphinxpxdimen=.75bp\relax

\PassOptionsToPackage{warn}{textcomp}
\usepackage[utf8]{inputenc}
\ifdefined\DeclareUnicodeCharacter
% support both utf8 and utf8x syntaxes
  \ifdefined\DeclareUnicodeCharacterAsOptional
    \def\sphinxDUC#1{\DeclareUnicodeCharacter{"#1}}
  \else
    \let\sphinxDUC\DeclareUnicodeCharacter
  \fi
  \sphinxDUC{00A0}{\nobreakspace}
  \sphinxDUC{2500}{\sphinxunichar{2500}}
  \sphinxDUC{2502}{\sphinxunichar{2502}}
  \sphinxDUC{2514}{\sphinxunichar{2514}}
  \sphinxDUC{251C}{\sphinxunichar{251C}}
  \sphinxDUC{2572}{\textbackslash}
\fi
\usepackage{cmap}
\usepackage[T1]{fontenc}
\usepackage{amsmath,amssymb,amstext}
\usepackage{babel}



\usepackage{times}
\expandafter\ifx\csname T@LGR\endcsname\relax
\else
% LGR was declared as font encoding
  \substitutefont{LGR}{\rmdefault}{cmr}
  \substitutefont{LGR}{\sfdefault}{cmss}
  \substitutefont{LGR}{\ttdefault}{cmtt}
\fi
\expandafter\ifx\csname T@X2\endcsname\relax
  \expandafter\ifx\csname T@T2A\endcsname\relax
  \else
  % T2A was declared as font encoding
    \substitutefont{T2A}{\rmdefault}{cmr}
    \substitutefont{T2A}{\sfdefault}{cmss}
    \substitutefont{T2A}{\ttdefault}{cmtt}
  \fi
\else
% X2 was declared as font encoding
  \substitutefont{X2}{\rmdefault}{cmr}
  \substitutefont{X2}{\sfdefault}{cmss}
  \substitutefont{X2}{\ttdefault}{cmtt}
\fi


\usepackage[Bjarne]{fncychap}
\usepackage{sphinx}

\fvset{fontsize=\small}
\usepackage{geometry}

% Include hyperref last.
\usepackage{hyperref}
% Fix anchor placement for figures with captions.
\usepackage{hypcap}% it must be loaded after hyperref.
% Set up styles of URL: it should be placed after hyperref.
\urlstyle{same}

\usepackage{sphinxmessages}
\setcounter{tocdepth}{1}



\title{MAterials Simulation Toolkit for Machine Learning (MAST-ML)}
\date{Apr 13, 2021}
\release{2.0}
\author{University of Wisconsin-Madison Computational Materials Group}
\newcommand{\sphinxlogo}{\vbox{}}
\renewcommand{\releasename}{Release}
\makeindex
\begin{document}

\pagestyle{empty}
\sphinxmaketitle
\pagestyle{plain}
\sphinxtableofcontents
\pagestyle{normal}
\phantomsection\label{\detokenize{index::doc}}



\chapter{Acknowledgements}
\label{\detokenize{00_acknowledgements:acknowledgements}}\label{\detokenize{00_acknowledgements::doc}}
Materials Simulation Toolkit for Machine Learning (MAST-ML)

MAST-ML is an open-source Python package designed to broaden and accelerate the use of machine learning in materials science research

As of MAST-ML version 3.x, much of the original code and workflow have been rewritten. The use of an input file in version 2.x and older
has been removed in favor of a more modular Jupyter notebook computing environment. Please see the examples and tutorials under
the mastml/examples folder for a guide in using MAST-ML

\sphinxstylestrong{Contributors}

University of Wisconsin-Madison Computational Materials Group:
\begin{itemize}
\item {} 
Prof. Dane Morgan

\item {} 
Dr. Ryan Jacobs

\item {} 
Dr. Tam Mayeshiba

\item {} 
Ben Afflerbach

\item {} 
Dr. Henry Wu

\end{itemize}

University of Kentucky contributors:
\begin{itemize}
\item {} 
Luke Harold Miles

\item {} 
Robert Max Williams

\item {} 
Prof. Raphael Finkel

\end{itemize}

University of Wisconsin-Madison Undergraduate Skunkworks members (Spring 2021):
\begin{itemize}
\item {} 
Avery Chan

\item {} 
Min Yi Lin

\item {} 
Hock Lye Lee

\end{itemize}

MAST-ML documentation:

An overview of code documentation and guides for installing MAST-ML can be found \sphinxhref{https://mastmldocs.readthedocs.io/en/latest/}{here}

A number of Jupyter notebook tutorials detailing different MAST-ML use cases can be found \sphinxhref{https://github.com/uw-cmg/MAST-ML/tree/dev\_Ryan\_2020-12-21/examples}{here}

\sphinxstylestrong{Funding}

This work was and is funded by the National Science Foundation (NSF) SI2 award No. 1148011 and DMREF award number DMR-1332851

\sphinxstylestrong{Citing MAST-ML}

If you find MAST-ML useful, please cite the following publication:

Jacobs, R., Mayeshiba, T., Afflerbach, B., Miles, L., Williams, M., Turner, M., Finkel, R., Morgan, D., “The Materials Simulation Toolkit for Machine Learning (MAST-ML): An automated open source toolkit to accelerate data- driven materials research”, Computational Materials Science 175 (2020), 109544. \sphinxurl{https://doi.org/10.1016/j.commatsci.2020.109544}

\sphinxstylestrong{Code Repository}

MAST-ML is available via PyPi: pip install mastml

MAST-ML is available via \sphinxhref{https://github.com/uw-cmg/MAST-ML}{Github}

git clone \textendash{}single-branch master \sphinxurl{https://github.com/uw-cmg/MAST-ML}


\chapter{MAST-ML version 3.x}
\label{\detokenize{0_4_majorchanges:mast-ml-version-3-x}}\label{\detokenize{0_4_majorchanges::doc}}

\section{New changes to MAST-ML}
\label{\detokenize{0_4_majorchanges:new-changes-to-mast-ml}}
As of MAST-ML version update 3.x and going forward, there are some significant changes to MAST-ML for users to
be aware of:

MAST-ML major updates:
\begin{itemize}
\item {} 
MAST-ML no longer uses an input file. The core functionality and workflow of MAST-ML has been rewritten to be more conducive to use in a Jupyter notebook environment. This major change has made the code more modular and transparent, and we believe more intuitive and easier to use in a research setting. The last version of MAST-ML to have input file support was version 2.0.20 on PyPi.

\item {} 
Each component of MAST-ML can be run in a Jupyter notebook environment, either locally or through a cloud-based service like Google Colab. As a result, we have completely reworked our use-case tutorials and examples. All of these MAST-ML tutorials are in the form of Jupyter notebooks and can be found in the mastml/examples folder on Github.

\item {} 
An active part of improving MAST-ML is to provide an automated, quantitative analysis of model domain assessement and model prediction uncertainty quantification (UQ). Version 3.x of MAST-ML includes more detailed implementation of model UQ using new and established techniques.

\end{itemize}

MAST-ML minor updates:
\begin{itemize}
\item {} 
More straightforward implementation of left-out test data, both designated manually by the user and via nested cross validation.

\item {} 
Improved integration of feature generation schemes in complimentary materials informatics packages, particularly matminer.

\item {} 
Improved data import schema based on locally-stored files, and via downloading data hosted on databases including Figshare, matminer, Materials Data Facility, and Foundry.

\item {} 
Support for generalized ensemble models with user-specified choice of model type to use as the weak learner, including support for ensembles of Keras-based neural networks.

\end{itemize}


\chapter{Installing MAST-ML}
\label{\detokenize{0_installation:installing-mast-ml}}\label{\detokenize{0_installation::doc}}

\section{Hardware and Data Requirements}
\label{\detokenize{0_1_hardware_data_requirements:hardware-and-data-requirements}}\label{\detokenize{0_1_hardware_data_requirements::doc}}

\subsection{Hardware}
\label{\detokenize{0_1_hardware_data_requirements:hardware}}
PC, Mac, computing cluster or Cloud resource (e.g. Google Colab) capable of running Python 3.


\subsection{Data}
\label{\detokenize{0_1_hardware_data_requirements:data}}\begin{itemize}
\item {} 
Numeric data file in the form of .xlsx file. There must be at least some target feature data, so that models can be fit.

\item {} 
First row of data file (each column) should have a text name (as string) which will be used for importing data with MAST-ML.

\item {} 
For more information and examples of how to import data into MAST-ML, see the mastml/examples folder and the MASTML\_examples\_dataimport.ipynb Jupyter notebook.

\end{itemize}


\section{Terminal installation (Linux or linux-like terminal environment e.g. Mac)}
\label{\detokenize{0_1_terminal_installation:terminal-installation-linux-or-linux-like-terminal-environment-e-g-mac}}\label{\detokenize{0_1_terminal_installation::doc}}
This documentation provides a few ways to install MAST-ML. If you don’t have python 3 on your system, begin
with the section “Install Python3”. If you already have python 3 installed,


\subsection{Install Python3}
\label{\detokenize{0_1_terminal_installation:install-python3}}
Install Python 3: for easier installation of numpy and scipy dependencies,
download Anaconda from \sphinxurl{https://www.continuum.io/downloads}


\subsubsection{Create a conda environment (if using Anaconda)}
\label{\detokenize{0_1_terminal_installation:create-a-conda-environment-if-using-anaconda}}
Create an anaconda python environment:

\begin{sphinxVerbatim}[commandchars=\\\{\}]
\PYG{n}{conda} \PYG{n}{create} \PYG{o}{\PYGZhy{}}\PYG{o}{\PYGZhy{}}\PYG{n}{name} \PYG{n}{MAST\PYGZus{}ML\PYGZus{}env} \PYG{n}{python}\PYG{o}{=}\PYG{l+m+mf}{3.7}
\PYG{n}{conda} \PYG{n}{activate} \PYG{n}{MAST\PYGZus{}ML\PYGZus{}env}
\end{sphinxVerbatim}


\subsubsection{Create a virtualenv environment (if not using Anaconda)}
\label{\detokenize{0_1_terminal_installation:create-a-virtualenv-environment-if-not-using-anaconda}}
Create a virtualenv environment:

\begin{sphinxVerbatim}[commandchars=\\\{\}]
\PYG{n}{python3} \PYG{o}{\PYGZhy{}}\PYG{n}{m} \PYG{n}{venv} \PYG{n}{MAST\PYGZus{}ML\PYGZus{}env}
\PYG{n}{source} \PYG{n}{MAST\PYGZus{}ML\PYGZus{}env}\PYG{o}{/}\PYG{n+nb}{bin}\PYG{o}{/}\PYG{n}{activate}
\end{sphinxVerbatim}


\subsection{Install the MAST-ML package via PyPi}
\label{\detokenize{0_1_terminal_installation:install-the-mast-ml-package-via-pypi}}
Pip install MAST-ML from PyPi:

\begin{sphinxVerbatim}[commandchars=\\\{\}]
\PYG{n}{pip} \PYG{n}{install} \PYG{n}{mastml}
\end{sphinxVerbatim}


\subsection{Install the MAST-ML package via Git}
\label{\detokenize{0_1_terminal_installation:install-the-mast-ml-package-via-git}}
As an alternative to PyPi, you can git clone the Github repository, for example:

\begin{sphinxVerbatim}[commandchars=\\\{\}]
\PYG{n}{git} \PYG{n}{clone} \PYG{o}{\PYGZhy{}}\PYG{o}{\PYGZhy{}}\PYG{n}{single}\PYG{o}{\PYGZhy{}}\PYG{n}{branch} \PYG{o}{\PYGZhy{}}\PYG{o}{\PYGZhy{}}\PYG{n}{branch} \PYG{n}{master} \PYG{n}{https}\PYG{p}{:}\PYG{o}{/}\PYG{o}{/}\PYG{n}{github}\PYG{o}{.}\PYG{n}{com}\PYG{o}{/}\PYG{n}{uw}\PYG{o}{\PYGZhy{}}\PYG{n}{cmg}\PYG{o}{/}\PYG{n}{MAST}\PYG{o}{\PYGZhy{}}\PYG{n}{ML}
\end{sphinxVerbatim}

Once the branch is downloaded, install the needed dependencies with:

\begin{sphinxVerbatim}[commandchars=\\\{\}]
\PYG{n}{pip} \PYG{n}{install} \PYG{o}{\PYGZhy{}}\PYG{n}{r} \PYG{n}{MAST}\PYG{o}{\PYGZhy{}}\PYG{n}{ML}\PYG{o}{/}\PYG{n}{requirements}\PYG{o}{.}\PYG{n}{txt}
\end{sphinxVerbatim}

Note that MAST-ML will need to be imported from within the MAST-ML directory as mastml is not
located in the usual spot where python looks for imported packages.


\subsubsection{Set up Juptyer notebooks}
\label{\detokenize{0_1_terminal_installation:set-up-juptyer-notebooks}}
There is no separate setup for Jupyter notebooks necessary;
once MAST-ML has been run and created a notebook, then in the terminal,
navigate to a directory housing the notebook and type:

\begin{sphinxVerbatim}[commandchars=\\\{\}]
\PYG{n}{jupyter} \PYG{n}{notebook}
\end{sphinxVerbatim}

and a browser window with the notebook should appear.


\subsubsection{Imports that don’t work}
\label{\detokenize{0_1_terminal_installation:imports-that-dont-work}}
First try anaconda install, and if that gives errors try pip install
Example: conda install numpy , or pip install numpy
Put the path to the installed MAST-ML folder in your PYTHONPATH if it isn’t already


\section{Windows installation}
\label{\detokenize{0_2_windows_installation:windows-installation}}\label{\detokenize{0_2_windows_installation::doc}}

\subsection{Install Python3}
\label{\detokenize{0_2_windows_installation:install-python3}}
Install Python 3: for easier installation of numpy and scipy dependencies,
download anaconda from \sphinxurl{https://www.continuum.io/downloads}


\subsubsection{Create a conda environment}
\label{\detokenize{0_2_windows_installation:create-a-conda-environment}}
From the Anaconda Navigator, go to Environments and create a new environment
Select python version \sphinxstylestrong{3.6}

Under “Channels”, along with defaults channel, “Add” the “materials” channel.
The Channels list should now read:

\begin{sphinxVerbatim}[commandchars=\\\{\}]
\PYG{n}{defaults}
\PYG{n}{materials}
\end{sphinxVerbatim}

(may be the “matsci” channel instead of the “materials” channel;
this channel is used to install pymatgen)


\subsubsection{Set up the Spyder IDE and Jupyter notebooks}
\label{\detokenize{0_2_windows_installation:set-up-the-spyder-ide-and-jupyter-notebooks}}
From the Anaconda Navigator, go to Home
With the newly created environment selected, click on “Install” below Jupyter.
Click on “Install” below Spyder.

Once the MASTML has been run and has created a jupyter notebook (run MASTML
from a location inside the anaconda environment, so that the notebook will
also be inside the environment tree), from the Anaconda Navigator, go to
Environments, make sure the environment is selected, press the green arrow
button, and select Open jupyter notebook.


\subsection{Install the MAST-ML package}
\label{\detokenize{0_2_windows_installation:install-the-mast-ml-package}}
Pip install MAST-ML from PyPi:

\begin{sphinxVerbatim}[commandchars=\\\{\}]
\PYG{n}{pip} \PYG{n}{install} \PYG{n}{mastml}
\end{sphinxVerbatim}

Alternatively, git clone the Github repository, for example:

\begin{sphinxVerbatim}[commandchars=\\\{\}]
\PYG{n}{git} \PYG{n}{clone} \PYG{n}{https}\PYG{p}{:}\PYG{o}{/}\PYG{o}{/}\PYG{n}{github}\PYG{o}{.}\PYG{n}{com}\PYG{o}{/}\PYG{n}{uw}\PYG{o}{\PYGZhy{}}\PYG{n}{cmg}\PYG{o}{/}\PYG{n}{MAST}\PYG{o}{\PYGZhy{}}\PYG{n}{ML}
\end{sphinxVerbatim}

Clone from “master” unless instructed specifically to use another branch.
Ask for access if you cannot find this code.

Check status.github.com for issues if you believe github may be malfunctioning

Run:

\begin{sphinxVerbatim}[commandchars=\\\{\}]
\PYG{n}{python} \PYG{n}{setup}\PYG{o}{.}\PYG{n}{py} \PYG{n}{install}
\end{sphinxVerbatim}


\subsubsection{Imports that don’t work}
\label{\detokenize{0_2_windows_installation:imports-that-dont-work}}
First try anaconda install, and if that gives errors try pip install
Example: conda install numpy , or pip install numpy
Put the path to the installed MAST-ML folder in your PYTHONPATH if it isn’t already


\subsection{Windows 10 install: step-by-step guide (credit Joe Kern)}
\label{\detokenize{0_2_windows_installation:windows-10-install-step-by-step-guide-credit-joe-kern}}
First, figure out if your computer is 32 or 64-bit. Type “system information” in your search bar. Look at system type. x86 is a 32-bit computer, x64 is a 64-bit.

Second, download an environment manager. Environments are directories in your computer that store dependencies. For instance, one program you run might be dependent on version 1.0 of another program x. However, another program you have might be dependent on version 2.0 of program x. Having multiple environments allows you utilize both programs and dependencies on your computer. I will recommend you download anaconda, not because it is the best, but because it is an environment manager I know how to get working with MAST-ML. Feel free to experiment with other managers. Download the Python 3.7 version at \sphinxurl{https://www.anaconda.com/distribution/}, just follow the installation instructions. Pick the graphical installer that corresponds with your computer system (64 bit or 32 bit).

Third, download Visual studio. Some of the MAST-ML dependencies require C++ distributables in order to run. Visual Studio Code is a code editor made for Windows 10. The dependencies for MAST-ML will look in the Visual Studio Code folder for these C++ distributables when they download. There may be another way to download these these C++ distributables without Visual Studio Code, but I am not sure how to do that. Go here to download \sphinxurl{https://visualstudio.microsoft.com/downloads/\#build-tools-for-visual-studio-2017}

Fourth, download Visual Studio with C++ build tools and restart the computer

Fifth, Open anaconda navigator. Click Environments and create at the bottom. Name it MASTML and make it Python 3.6. DO NOT MAKE IT Python 3.7 or Python version 2.6 or 2.7. Some dependencies do not work with those other version.

Sixth, click the arrow next to your environment name and open a command shell. In the command line type “pip install “ and then copy paste the dependency names from the dependency file into your command prompt.

Seventh, test if MAST-ML runs. There are multiple ways to do this, but I will outline one. Navigate to your MAST-ML folder in the command prompt. To do this, you need to know the command ‘cd’. Typing ‘cd’ will let you change the directory you command prompt is operating in. In order to navigate to your mast-ml folder, right click the folder and click properties. Copy the location and in the command prompt type ‘cd’ and paste the location after. Add a ‘Mast-ml’ or whatever your folder is called to the end of the pasted value so you can get to mastml

Finally, copy paste python -m mastml.mastml\_driver mastml/tests/conf/example\_input.conf mastml/tests/csv/example\_data.csv -o results/mastml\_tutorial into your command prompt and run. If it all works, you’re good to go.


\chapter{Getting Started with MAST-ML}
\label{\detokenize{0_3_startup:getting-started-with-mast-ml}}\label{\detokenize{0_3_startup::doc}}

\section{Installing MAST-ML}
\label{\detokenize{0_3_startup:installing-mast-ml}}
If you have not done so, the first step is to install MAST-ML. More information on how to install MAST-ML can be found
by navigating to the “Installing MAST-ML” tab on the left-hand side of this documentation page.


\section{Performing your first MAST-ML run}
\label{\detokenize{0_3_startup:performing-your-first-mast-ml-run}}
Once MAST-ML is installed, you are ready to perform your first MAST-ML run

The first MAST-ML tutorial can be found under the mastml/examples folder, and is named \sphinxstylestrong{MASTML\_Tutorial\_1\_GettingStarted.ipynb}

This first notebook can also be downloaded via this link: \sphinxcode{\sphinxupquote{MASTML\_Tutorial\_1\_GettingStarted}}

Open this first example notebook either in Google Colab if running on the cloud or locally by starting a Jupyter notebook
session. There are explanations for each cell of the notebook. Reading through and running this tutorial should take
about 10 minutes. At the end, you will have performed your first MAST-ML run!

Once complete, there are a series of other example/tutorial notebooks that can be found in the mastml/examples folder
on Github.


\chapter{Overview of MAST-ML tutorials and examples}
\label{\detokenize{0_5_tutorials:overview-of-mast-ml-tutorials-and-examples}}\label{\detokenize{0_5_tutorials::doc}}

\section{MAST-ML tutorials}
\label{\detokenize{0_5_tutorials:mast-ml-tutorials}}
There are numerous MAST-ML tutorial and example Jupyter notebooks. These notebooks
can be found in the mastml/examples folder. Here, a brief overview of the contents
of each tutorial is provided:
\begin{description}
\item[{Tutorial 1: Getting Started (MASTML\_Tutorial\_1\_GettingStarted.ipynb):}] \leavevmode\begin{description}
\item[{In this notebook, we will perform a first, basic run where we:}] \leavevmode\begin{enumerate}
\sphinxsetlistlabels{\arabic}{enumi}{enumii}{}{.}%
\item {} 
Import example data of Boston housing prices

\item {} 
Define a data preprocessor to normalize the data

\item {} 
Define a linear regression model and kernel ridge model to fit the data

\item {} 
Evaluate each of our models with 5-fold cross validation

\item {} 
Add a random forest model to our run and compare model performance

\end{enumerate}

\end{description}

\item[{Tutorial 2: Data Import and Cleaning (MASTML\_Tutorial\_2\_DataImport.ipynb):}] \leavevmode\begin{description}
\item[{In this notebook, we will learn different ways to download and import data into a MAST-ML run:}] \leavevmode\begin{enumerate}
\sphinxsetlistlabels{\arabic}{enumi}{enumii}{}{.}%
\item {} 
Import model datasets from scikit-learn

\item {} 
Conduct different data cleaning methods

\item {} 
Import and prepare a real dataset that is stored locally

\item {} 
Download data from various materials databases

\end{enumerate}

\end{description}

\item[{Tutorial 3: Feature Generation and Selection (MASTML\_Tutorial\_3\_FeatureEngineering.ipynb):}] \leavevmode\begin{description}
\item[{In this notebook, we will learn different ways to generate, preprocess, and select features:}] \leavevmode\begin{enumerate}
\sphinxsetlistlabels{\arabic}{enumi}{enumii}{}{.}%
\item {} 
Generate features based on material composition

\item {} 
Generate one-hot encoded features based on group labels

\item {} 
Preprocess features to be normalized

\item {} 
Select features using an ensemble model-based approach

\item {} 
Generate learning curves using a basic feature selection approach

\item {} 
Select features using forward selection

\end{enumerate}

\end{description}

\item[{Tutorial 4: Model Fits and Data Split Tests (MASTML\_Tutorial\_4\_Models\_and\_Tests.ipynb):}] \leavevmode
In this notebook, we will learn how to run a few different types of models on a select dataset, and conduct
a few different types of data splits to evaluate our model performance. In this tutorial, we will:
\begin{enumerate}
\sphinxsetlistlabels{\arabic}{enumi}{enumii}{}{.}%
\item {} 
Run a variety of model types from the scikit-learn package

\item {} 
Run a bootstrapped ensemble of neural networks

\item {} 
Compare performance of scikit-learn’s gradient boosting method and XGBoost

\item {} 
Compare performance of scikit-learn’s neural network and Keras-based neural network regressor

\item {} 
Compare model performance using random k-fold cross validation and leave out group cross validation

\item {} 
Explore the limits of model performance when up to 90\% of data is left out using leave out percent cross validation

\end{enumerate}

\item[{Tutorial 5: Left-out data, Nested cross-validation, and Optimized models (MASTML\_Tutorial\_5\_NestedCV\_and\_OptimizedModels.ipynb):}] \leavevmode
In this notebook, we will perform more advanced model fitting routines, including nested cross validation and
hyperparameter optimization. In this tutorial, we will learn how to use MAST-ML to:
\begin{enumerate}
\sphinxsetlistlabels{\arabic}{enumi}{enumii}{}{.}%
\item {} 
Assess performance on manually left-out test data

\item {} 
Perform nested cross validation to assess model performance on unseen data

\item {} 
Optimize the hyperparameters of our models to create the best model

\end{enumerate}

\item[{Tutorial 6: Model Error Analysis, Uncertainty Quantification (MASTML\_Tutorial\_6\_ErrorAnalysis\_UncertaintyQuantification.ipynb):}] \leavevmode\begin{description}
\item[{In this notebook tutorial, we will learn about how MAST-ML can be used to:}] \leavevmode\begin{enumerate}
\sphinxsetlistlabels{\arabic}{enumi}{enumii}{}{.}%
\item {} 
Assess the true and predicted errors of our model, and some useful measures of their statistical distributions

\item {} 
Explore different methods of quantifying and calibrating model uncertainties.

\item {} 
Compare the uncertainty quantification behavior of Bayesian and ensemble-based models.

\end{enumerate}

\end{description}

\end{description}


\chapter{Code Documentation: Data Cleaning}
\label{\detokenize{1_data_cleaning:code-documentation-data-cleaning}}\label{\detokenize{1_data_cleaning::doc}}

\section{mastml.data\_cleaning Module}
\label{\detokenize{1_data_cleaning:module-mastml.data_cleaning}}\label{\detokenize{1_data_cleaning:mastml-data-cleaning-module}}\index{mastml.data\_cleaning (module)@\spxentry{mastml.data\_cleaning}\spxextra{module}}
This module provides various methods for cleaning data that has been imported into MAST-ML, prior to model fitting.
\begin{description}
\item[{DataCleaning:}] \leavevmode
Class that enables easy use of various data cleaning methods, such as removal of missing values, different
modes of data imputation, or using principal componenet analysis to fill interpolate missing values.

\item[{DataUtilities:}] \leavevmode
Support class used to evaluate some basic statistics of imported data, such as its distribution, mean, etc.
Also provides a means of flagging potential outlier datapoints based on their deviation from the overall data
distribution.

\item[{PPCA:}] \leavevmode
Class used by the PCA data cleaning routine in the DataCleaning class to perform probabilistic PCA to fill in
missing data.

\end{description}


\subsection{Classes}
\label{\detokenize{1_data_cleaning:classes}}

\begin{savenotes}\sphinxatlongtablestart\begin{longtable}[c]{\X{1}{2}\X{1}{2}}
\hline

\endfirsthead

\multicolumn{2}{c}%
{\makebox[0pt]{\sphinxtablecontinued{\tablename\ \thetable{} -- continued from previous page}}}\\
\hline

\endhead

\hline
\multicolumn{2}{r}{\makebox[0pt][r]{\sphinxtablecontinued{Continued on next page}}}\\
\endfoot

\endlastfoot

\sphinxcode{\sphinxupquote{Counter}}(**kwds)
&
Dict subclass for counting hashable items.
\\
\hline
{\hyperref[\detokenize{api/mastml.data_cleaning.DataCleaning:mastml.data_cleaning.DataCleaning}]{\sphinxcrossref{\sphinxcode{\sphinxupquote{DataCleaning}}}}}()
&
Class to perform various data cleaning operations, such as imputation or NaN removal
\\
\hline
{\hyperref[\detokenize{api/mastml.data_cleaning.DataUtilities:mastml.data_cleaning.DataUtilities}]{\sphinxcrossref{\sphinxcode{\sphinxupquote{DataUtilities}}}}}
&
Class that contains some basic data analysis utilities, such as flagging columns that contain problematic string entries, or flagging potential outlier values based on threshold values
\\
\hline
\sphinxcode{\sphinxupquote{Histogram}}
&
Class to generate histogram plots, such as histograms of residual values
\\
\hline
{\hyperref[\detokenize{api/mastml.data_cleaning.PPCA:mastml.data_cleaning.PPCA}]{\sphinxcrossref{\sphinxcode{\sphinxupquote{PPCA}}}}}()
&
Class to perform probabilistic principal component analysis (PPCA) to fill in missing data.
\\
\hline
\sphinxcode{\sphinxupquote{SimpleImputer}}(*{[}, missing\_values, strategy, …{]})
&
Imputation transformer for completing missing values.
\\
\hline
\sphinxcode{\sphinxupquote{datetime}}(year, month, day{[}, hour{[}, minute{[}, …)
&
The year, month and day arguments are required.
\\
\hline
\end{longtable}\sphinxatlongtableend\end{savenotes}


\subsubsection{DataCleaning}
\label{\detokenize{api/mastml.data_cleaning.DataCleaning:datacleaning}}\label{\detokenize{api/mastml.data_cleaning.DataCleaning::doc}}\index{DataCleaning (class in mastml.data\_cleaning)@\spxentry{DataCleaning}\spxextra{class in mastml.data\_cleaning}}

\begin{fulllineitems}
\phantomsection\label{\detokenize{api/mastml.data_cleaning.DataCleaning:mastml.data_cleaning.DataCleaning}}\pysigline{\sphinxbfcode{\sphinxupquote{class }}\sphinxcode{\sphinxupquote{mastml.data\_cleaning.}}\sphinxbfcode{\sphinxupquote{DataCleaning}}}
Bases: \sphinxcode{\sphinxupquote{object}}

Class to perform various data cleaning operations, such as imputation or NaN removal
\begin{description}
\item[{Args:}] \leavevmode
None

\item[{Methods:}] \leavevmode\begin{description}
\item[{remove: Method that removes a full column or row of data values if one column or row contains NaN or is blank}] \leavevmode\begin{description}
\item[{Args:}] \leavevmode
X: (pd.DataFrame), dataframe containing X data

y: (pd.Series), series containing y data

axis: (int), whether to remove rows (axis=0) or columns (axis=1)

\item[{Returns:}] \leavevmode
X: (pd.DataFrame): dataframe of cleaned X data

y: (pd.Series): series of cleaned y data

\end{description}

\item[{imputation: Method that imputes values to the missing places based on the median, mean, etc. of the data in the column}] \leavevmode\begin{description}
\item[{Args:}] \leavevmode
X: (pd.DataFrame), dataframe containing X data

y: (pd.Series), series containing y data

strategy: (str), method of imputation, e.g. median, mean, etc.

\item[{Returns:}] \leavevmode
X: (pd.DataFrame): dataframe of cleaned X data

y: (pd.Series): series of cleaned y data

\end{description}

\item[{ppca: Method that imputes data using principal component analysis to interpolate missing values}] \leavevmode\begin{description}
\item[{Args:}] \leavevmode
X: (pd.DataFrame), dataframe containing X data

y: (pd.Series), series containing y data

\item[{Returns:}] \leavevmode
X: (pd.DataFrame): dataframe of cleaned X data

y: (pd.Series): series of cleaned y data

\end{description}

\item[{evaluate: Main method to evaluate initial data analysis routines (e.g. flag outliers), perform data cleaning and save output to folder}] \leavevmode\begin{description}
\item[{Args:}] \leavevmode
X: (pd.DataFrame), dataframe containing X data

y: (pd.Series), series containing y data

method: (str), data cleaning method name, must be one of ‘remove’, ‘imputation’ or ‘ppca’

savepath: (str), string containing the savepath information

kwargs: additional keyword arguments needed for the remove, imputation or ppca methods

\item[{Returns:}] \leavevmode
X: (pd.DataFrame): dataframe of cleaned X data

y: (pd.Series): series of cleaned y data

\end{description}

\item[{\_setup\_savedir: method to create a savedir based on the provided model, splitter, selector names and datetime}] \leavevmode\begin{description}
\item[{Args:}] \leavevmode
savepath: (str), string designating the savepath

\item[{Returns:}] \leavevmode
splitdir: (str), string containing the new subdirectory to save results to

\end{description}

\end{description}

\end{description}
\subsubsection*{Methods Summary}


\begin{savenotes}\sphinxatlongtablestart\begin{longtable}[c]{\X{1}{2}\X{1}{2}}
\hline

\endfirsthead

\multicolumn{2}{c}%
{\makebox[0pt]{\sphinxtablecontinued{\tablename\ \thetable{} -- continued from previous page}}}\\
\hline

\endhead

\hline
\multicolumn{2}{r}{\makebox[0pt][r]{\sphinxtablecontinued{Continued on next page}}}\\
\endfoot

\endlastfoot

{\hyperref[\detokenize{api/mastml.data_cleaning.DataCleaning:mastml.data_cleaning.DataCleaning.evaluate}]{\sphinxcrossref{\sphinxcode{\sphinxupquote{evaluate}}}}}(X, y, method{[}, savepath, make\_new\_dir{]})
&

\\
\hline
{\hyperref[\detokenize{api/mastml.data_cleaning.DataCleaning:mastml.data_cleaning.DataCleaning.imputation}]{\sphinxcrossref{\sphinxcode{\sphinxupquote{imputation}}}}}(X, y, strategy)
&

\\
\hline
{\hyperref[\detokenize{api/mastml.data_cleaning.DataCleaning:mastml.data_cleaning.DataCleaning.ppca}]{\sphinxcrossref{\sphinxcode{\sphinxupquote{ppca}}}}}(X, y)
&

\\
\hline
{\hyperref[\detokenize{api/mastml.data_cleaning.DataCleaning:mastml.data_cleaning.DataCleaning.remove}]{\sphinxcrossref{\sphinxcode{\sphinxupquote{remove}}}}}(X, y, axis)
&

\\
\hline
\end{longtable}\sphinxatlongtableend\end{savenotes}
\subsubsection*{Methods Documentation}
\index{evaluate() (mastml.data\_cleaning.DataCleaning method)@\spxentry{evaluate()}\spxextra{mastml.data\_cleaning.DataCleaning method}}

\begin{fulllineitems}
\phantomsection\label{\detokenize{api/mastml.data_cleaning.DataCleaning:mastml.data_cleaning.DataCleaning.evaluate}}\pysiglinewithargsret{\sphinxbfcode{\sphinxupquote{evaluate}}}{\emph{X}, \emph{y}, \emph{method}, \emph{savepath=None}, \emph{make\_new\_dir=True}, \emph{**kwargs}}{}
\end{fulllineitems}

\index{imputation() (mastml.data\_cleaning.DataCleaning method)@\spxentry{imputation()}\spxextra{mastml.data\_cleaning.DataCleaning method}}

\begin{fulllineitems}
\phantomsection\label{\detokenize{api/mastml.data_cleaning.DataCleaning:mastml.data_cleaning.DataCleaning.imputation}}\pysiglinewithargsret{\sphinxbfcode{\sphinxupquote{imputation}}}{\emph{X}, \emph{y}, \emph{strategy}}{}
\end{fulllineitems}

\index{ppca() (mastml.data\_cleaning.DataCleaning method)@\spxentry{ppca()}\spxextra{mastml.data\_cleaning.DataCleaning method}}

\begin{fulllineitems}
\phantomsection\label{\detokenize{api/mastml.data_cleaning.DataCleaning:mastml.data_cleaning.DataCleaning.ppca}}\pysiglinewithargsret{\sphinxbfcode{\sphinxupquote{ppca}}}{\emph{X}, \emph{y}}{}
\end{fulllineitems}

\index{remove() (mastml.data\_cleaning.DataCleaning method)@\spxentry{remove()}\spxextra{mastml.data\_cleaning.DataCleaning method}}

\begin{fulllineitems}
\phantomsection\label{\detokenize{api/mastml.data_cleaning.DataCleaning:mastml.data_cleaning.DataCleaning.remove}}\pysiglinewithargsret{\sphinxbfcode{\sphinxupquote{remove}}}{\emph{X}, \emph{y}, \emph{axis}}{}
\end{fulllineitems}


\end{fulllineitems}



\subsubsection{DataUtilities}
\label{\detokenize{api/mastml.data_cleaning.DataUtilities:datautilities}}\label{\detokenize{api/mastml.data_cleaning.DataUtilities::doc}}\index{DataUtilities (class in mastml.data\_cleaning)@\spxentry{DataUtilities}\spxextra{class in mastml.data\_cleaning}}

\begin{fulllineitems}
\phantomsection\label{\detokenize{api/mastml.data_cleaning.DataUtilities:mastml.data_cleaning.DataUtilities}}\pysigline{\sphinxbfcode{\sphinxupquote{class }}\sphinxcode{\sphinxupquote{mastml.data\_cleaning.}}\sphinxbfcode{\sphinxupquote{DataUtilities}}}
Bases: \sphinxcode{\sphinxupquote{object}}

Class that contains some basic data analysis utilities, such as flagging columns that contain problematic string
entries, or flagging potential outlier values based on threshold values
\begin{description}
\item[{Args:}] \leavevmode
None

\item[{Methods:}] \leavevmode\begin{description}
\item[{flag\_outliers: Method that scans values in each X feature matrix column and flags values that are larger than X standard deviations from the average of that column value. The index and column values of potentially problematic points are listed and written to an output file.}] \leavevmode\begin{description}
\item[{Args:}] \leavevmode
X: (pd.DataFrame), dataframe containing X data

y: (pd.Series), series containing y data

savepath: (str), string containing the save path directory

n\_stdevs: (int), number of standard deviations to use as threshold value

\item[{Returns:}] \leavevmode
None

\end{description}

\item[{flag\_columns\_with\_strings: Method that ascertains which columns in data contain string entries}] \leavevmode\begin{description}
\item[{Args:}] \leavevmode
X: (pd.DataFrame), dataframe containing X data

y: (pd.Series), series containing y data

savepath: (str), string containing the save path directory

\item[{Returns:}] \leavevmode
None

\end{description}

\end{description}

\end{description}
\subsubsection*{Methods Summary}


\begin{savenotes}\sphinxatlongtablestart\begin{longtable}[c]{\X{1}{2}\X{1}{2}}
\hline

\endfirsthead

\multicolumn{2}{c}%
{\makebox[0pt]{\sphinxtablecontinued{\tablename\ \thetable{} -- continued from previous page}}}\\
\hline

\endhead

\hline
\multicolumn{2}{r}{\makebox[0pt][r]{\sphinxtablecontinued{Continued on next page}}}\\
\endfoot

\endlastfoot

{\hyperref[\detokenize{api/mastml.data_cleaning.DataUtilities:mastml.data_cleaning.DataUtilities.flag_columns_with_strings}]{\sphinxcrossref{\sphinxcode{\sphinxupquote{flag\_columns\_with\_strings}}}}}(X, y, savepath)
&

\\
\hline
{\hyperref[\detokenize{api/mastml.data_cleaning.DataUtilities:mastml.data_cleaning.DataUtilities.flag_outliers}]{\sphinxcrossref{\sphinxcode{\sphinxupquote{flag\_outliers}}}}}(X, y, savepath{[}, n\_stdevs{]})
&

\\
\hline
\end{longtable}\sphinxatlongtableend\end{savenotes}
\subsubsection*{Methods Documentation}
\index{flag\_columns\_with\_strings() (mastml.data\_cleaning.DataUtilities class method)@\spxentry{flag\_columns\_with\_strings()}\spxextra{mastml.data\_cleaning.DataUtilities class method}}

\begin{fulllineitems}
\phantomsection\label{\detokenize{api/mastml.data_cleaning.DataUtilities:mastml.data_cleaning.DataUtilities.flag_columns_with_strings}}\pysiglinewithargsret{\sphinxbfcode{\sphinxupquote{classmethod }}\sphinxbfcode{\sphinxupquote{flag\_columns\_with\_strings}}}{\emph{X}, \emph{y}, \emph{savepath}}{}
\end{fulllineitems}

\index{flag\_outliers() (mastml.data\_cleaning.DataUtilities class method)@\spxentry{flag\_outliers()}\spxextra{mastml.data\_cleaning.DataUtilities class method}}

\begin{fulllineitems}
\phantomsection\label{\detokenize{api/mastml.data_cleaning.DataUtilities:mastml.data_cleaning.DataUtilities.flag_outliers}}\pysiglinewithargsret{\sphinxbfcode{\sphinxupquote{classmethod }}\sphinxbfcode{\sphinxupquote{flag\_outliers}}}{\emph{X}, \emph{y}, \emph{savepath}, \emph{n\_stdevs=3}}{}
\end{fulllineitems}


\end{fulllineitems}



\subsubsection{PPCA}
\label{\detokenize{api/mastml.data_cleaning.PPCA:ppca}}\label{\detokenize{api/mastml.data_cleaning.PPCA::doc}}\index{PPCA (class in mastml.data\_cleaning)@\spxentry{PPCA}\spxextra{class in mastml.data\_cleaning}}

\begin{fulllineitems}
\phantomsection\label{\detokenize{api/mastml.data_cleaning.PPCA:mastml.data_cleaning.PPCA}}\pysigline{\sphinxbfcode{\sphinxupquote{class }}\sphinxcode{\sphinxupquote{mastml.data\_cleaning.}}\sphinxbfcode{\sphinxupquote{PPCA}}}
Bases: \sphinxcode{\sphinxupquote{object}}

Class to perform probabilistic principal component analysis (PPCA) to fill in missing data.

This PPCA routine was taken directly from \sphinxurl{https://github.com/allentran/pca-magic}. Due to import errors, for ease of use
we have elected to copy the module here. This github repo was last accessed on 8/27/18. The code comprising the PPCA
class below was not developed by and is not owned by the University of Wisconsin-Madison MAST-ML development team.
\subsubsection*{Methods Summary}


\begin{savenotes}\sphinxatlongtablestart\begin{longtable}[c]{\X{1}{2}\X{1}{2}}
\hline

\endfirsthead

\multicolumn{2}{c}%
{\makebox[0pt]{\sphinxtablecontinued{\tablename\ \thetable{} -- continued from previous page}}}\\
\hline

\endhead

\hline
\multicolumn{2}{r}{\makebox[0pt][r]{\sphinxtablecontinued{Continued on next page}}}\\
\endfoot

\endlastfoot

{\hyperref[\detokenize{api/mastml.data_cleaning.PPCA:mastml.data_cleaning.PPCA.fit}]{\sphinxcrossref{\sphinxcode{\sphinxupquote{fit}}}}}(data{[}, d, tol, min\_obs, verbose{]})
&

\\
\hline
{\hyperref[\detokenize{api/mastml.data_cleaning.PPCA:mastml.data_cleaning.PPCA.load}]{\sphinxcrossref{\sphinxcode{\sphinxupquote{load}}}}}(fpath)
&

\\
\hline
{\hyperref[\detokenize{api/mastml.data_cleaning.PPCA:mastml.data_cleaning.PPCA.save}]{\sphinxcrossref{\sphinxcode{\sphinxupquote{save}}}}}(fpath)
&

\\
\hline
{\hyperref[\detokenize{api/mastml.data_cleaning.PPCA:mastml.data_cleaning.PPCA.transform}]{\sphinxcrossref{\sphinxcode{\sphinxupquote{transform}}}}}({[}data{]})
&

\\
\hline
\end{longtable}\sphinxatlongtableend\end{savenotes}
\subsubsection*{Methods Documentation}
\index{fit() (mastml.data\_cleaning.PPCA method)@\spxentry{fit()}\spxextra{mastml.data\_cleaning.PPCA method}}

\begin{fulllineitems}
\phantomsection\label{\detokenize{api/mastml.data_cleaning.PPCA:mastml.data_cleaning.PPCA.fit}}\pysiglinewithargsret{\sphinxbfcode{\sphinxupquote{fit}}}{\emph{data}, \emph{d=None}, \emph{tol=0.0001}, \emph{min\_obs=10}, \emph{verbose=False}}{}
\end{fulllineitems}

\index{load() (mastml.data\_cleaning.PPCA method)@\spxentry{load()}\spxextra{mastml.data\_cleaning.PPCA method}}

\begin{fulllineitems}
\phantomsection\label{\detokenize{api/mastml.data_cleaning.PPCA:mastml.data_cleaning.PPCA.load}}\pysiglinewithargsret{\sphinxbfcode{\sphinxupquote{load}}}{\emph{fpath}}{}
\end{fulllineitems}

\index{save() (mastml.data\_cleaning.PPCA method)@\spxentry{save()}\spxextra{mastml.data\_cleaning.PPCA method}}

\begin{fulllineitems}
\phantomsection\label{\detokenize{api/mastml.data_cleaning.PPCA:mastml.data_cleaning.PPCA.save}}\pysiglinewithargsret{\sphinxbfcode{\sphinxupquote{save}}}{\emph{fpath}}{}
\end{fulllineitems}

\index{transform() (mastml.data\_cleaning.PPCA method)@\spxentry{transform()}\spxextra{mastml.data\_cleaning.PPCA method}}

\begin{fulllineitems}
\phantomsection\label{\detokenize{api/mastml.data_cleaning.PPCA:mastml.data_cleaning.PPCA.transform}}\pysiglinewithargsret{\sphinxbfcode{\sphinxupquote{transform}}}{\emph{data=None}}{}
\end{fulllineitems}


\end{fulllineitems}



\subsection{Class Inheritance Diagram}
\label{\detokenize{1_data_cleaning:class-inheritance-diagram}}
\sphinxincludegraphics[]{None}


\chapter{Code Documentation: Data Splitters}
\label{\detokenize{2_data_splitters:code-documentation-data-splitters}}\label{\detokenize{2_data_splitters::doc}}

\section{mastml.data\_splitters Module}
\label{\detokenize{2_data_splitters:module-mastml.data_splitters}}\label{\detokenize{2_data_splitters:mastml-data-splitters-module}}\index{mastml.data\_splitters (module)@\spxentry{mastml.data\_splitters}\spxextra{module}}
This module contains a collection of methods to split data into different types of train/test sets. Data splitters
are the core component to evaluating model performance.
\begin{description}
\item[{BaseSplitter:}] \leavevmode
Base class that handles the core MAST-ML data splitting and model evaluation workflow. This class is responsible
for looping over provided feature selectors, models, and data splits and training and evaluating the model for each
split, then generating the necessary plots and performance statistics. All different splitter types inherit this
base class.

\item[{SklearnDataSplitter:}] \leavevmode
Wrapper class to enable MAST-ML workflow compatible use of any data splitter contained in scikit-learn, e.g. KFold,
RepeatedKFold, LeaveOneGroupOut, etc.

\item[{NoSplit:}] \leavevmode
Class that doesn’t perform any data split. Equivalent to a “full fit” of the data where all data is used in training.

\item[{JustEachGroup:}] \leavevmode
Class that splits data so each individual group is used as training with all other groups used as testing. Essentially
the inverse of LeaveOneGroupOut, this class trains only on one group and predicts the rest, as opposed to training
on all but one group and testing on the left-out group.

\item[{LeaveCloseCompositionsOut:}] \leavevmode
Class to split data based on their compositional similiarity. A useful means to separate compositionally similar
compounds into the training or testing set, so that similar materials are not contained in both sets.

\item[{LeaveOutPercent:}] \leavevmode
Method to randomly split the data based on fraction of total data points, rather than a designated number of splits.
Enables one to do higher than 50\% leave out (this is highest leave out possible with KFold where k=2), so can do e.g.
leave out 90\% data.

\item[{LeaveOutTwinCV:}] \leavevmode
Another method to help separate similar data from the training and testing set. This method makes use of a general
distance metric on the provided features, and flags twins as those data points within some provided distance threshold
in the feature space.

\item[{Bootstrap:}] \leavevmode
Method to perform bootstrap resampling, i.e. random leave-out with replacement.

\end{description}


\subsection{Classes}
\label{\detokenize{2_data_splitters:classes}}

\begin{savenotes}\sphinxatlongtablestart\begin{longtable}[c]{\X{1}{2}\X{1}{2}}
\hline

\endfirsthead

\multicolumn{2}{c}%
{\makebox[0pt]{\sphinxtablecontinued{\tablename\ \thetable{} -- continued from previous page}}}\\
\hline

\endhead

\hline
\multicolumn{2}{r}{\makebox[0pt][r]{\sphinxtablecontinued{Continued on next page}}}\\
\endfoot

\endlastfoot

{\hyperref[\detokenize{api/mastml.data_splitters.BaseSplitter:mastml.data_splitters.BaseSplitter}]{\sphinxcrossref{\sphinxcode{\sphinxupquote{BaseSplitter}}}}}()
&
Class functioning as a base splitter with methods for organizing output and evaluating any mastml data splitter
\\
\hline
{\hyperref[\detokenize{api/mastml.data_splitters.Bootstrap:mastml.data_splitters.Bootstrap}]{\sphinxcrossref{\sphinxcode{\sphinxupquote{Bootstrap}}}}}(n{[}, n\_bootstraps, train\_size, …{]})
&
\# Note: Bootstrap taken directly from sklearn Github (\sphinxurl{https://github.com/scikit-learn/scikit-learn/blob/0.11.X/sklearn/cross\_validation.py}) \# which was necessary as it was later removed from more recent sklearn releases Random sampling with replacement cross-validation iterator Provides train/test indices to split data in train test sets while resampling the input n\_bootstraps times: each time a new random split of the data is performed and then samples are drawn (with replacement) on each side of the split to build the training and test sets.
\\
\hline
\sphinxcode{\sphinxupquote{Composition}}(*args{[}, strict{]})
&
Represents a Composition, which is essentially a \{element:amount\} mapping type.
\\
\hline
\sphinxcode{\sphinxupquote{ElementFraction}}()
&
Class to calculate the atomic fraction of each element in a composition.
\\
\hline
\sphinxcode{\sphinxupquote{ErrorUtils}}
&
Collection of functions to conduct error analysis on certain types of models (uncertainty quantification), and prepare residual and model error data for plotting, as well as recalibrate model errors with various methods
\\
\hline
{\hyperref[\detokenize{api/mastml.data_splitters.JustEachGroup:mastml.data_splitters.JustEachGroup}]{\sphinxcrossref{\sphinxcode{\sphinxupquote{JustEachGroup}}}}}()
&
Class to train the model on one group at a time and test it on the rest of the data This class wraps scikit-learn’s LeavePGroupsOut with P set to n-1.
\\
\hline
{\hyperref[\detokenize{api/mastml.data_splitters.LeaveCloseCompositionsOut:mastml.data_splitters.LeaveCloseCompositionsOut}]{\sphinxcrossref{\sphinxcode{\sphinxupquote{LeaveCloseCompositionsOut}}}}}(composition\_df{[}, …{]})
&
Leave-P-out where you exclude materials with compositions close to those the test set
\\
\hline
{\hyperref[\detokenize{api/mastml.data_splitters.LeaveOutPercent:mastml.data_splitters.LeaveOutPercent}]{\sphinxcrossref{\sphinxcode{\sphinxupquote{LeaveOutPercent}}}}}({[}percent\_leave\_out, n\_repeats{]})
&
Class to train the model using a certain percentage of data as training data
\\
\hline
{\hyperref[\detokenize{api/mastml.data_splitters.LeaveOutTwinCV:mastml.data_splitters.LeaveOutTwinCV}]{\sphinxcrossref{\sphinxcode{\sphinxupquote{LeaveOutTwinCV}}}}}({[}threshold, ord, debug, …{]})
&
Class to remove data twins from the test data.
\\
\hline
\sphinxcode{\sphinxupquote{Metrics}}(metrics\_list{[}, metrics\_type{]})
&
Class containing access to a wide range of metrics from scikit-learn and a number of MAST-ML custom-written metrics
\\
\hline
\sphinxcode{\sphinxupquote{NearestNeighbors}}(*{[}, n\_neighbors, radius, …{]})
&
Unsupervised learner for implementing neighbor searches.
\\
\hline
\sphinxcode{\sphinxupquote{NoPreprocessor}}({[}preprocessor, as\_frame{]})
&
Class for having a “null” transform where the output is the same as the input.
\\
\hline
\sphinxcode{\sphinxupquote{NoSelect}}()
&
Class for having a “null” transform where the output is the same as the input.
\\
\hline
{\hyperref[\detokenize{api/mastml.data_splitters.NoSplit:mastml.data_splitters.NoSplit}]{\sphinxcrossref{\sphinxcode{\sphinxupquote{NoSplit}}}}}()
&
Class to just train the model on the training data and test it on that same data.
\\
\hline
{\hyperref[\detokenize{api/mastml.data_splitters.SklearnDataSplitter:mastml.data_splitters.SklearnDataSplitter}]{\sphinxcrossref{\sphinxcode{\sphinxupquote{SklearnDataSplitter}}}}}(splitter, **kwargs)
&
Class to wrap any scikit-learn based data splitter, e.g.
\\
\hline
\sphinxcode{\sphinxupquote{datetime}}(year, month, day{[}, hour{[}, minute{[}, …)
&
The year, month and day arguments are required.
\\
\hline
\end{longtable}\sphinxatlongtableend\end{savenotes}


\subsubsection{BaseSplitter}
\label{\detokenize{api/mastml.data_splitters.BaseSplitter:basesplitter}}\label{\detokenize{api/mastml.data_splitters.BaseSplitter::doc}}\index{BaseSplitter (class in mastml.data\_splitters)@\spxentry{BaseSplitter}\spxextra{class in mastml.data\_splitters}}

\begin{fulllineitems}
\phantomsection\label{\detokenize{api/mastml.data_splitters.BaseSplitter:mastml.data_splitters.BaseSplitter}}\pysigline{\sphinxbfcode{\sphinxupquote{class }}\sphinxcode{\sphinxupquote{mastml.data\_splitters.}}\sphinxbfcode{\sphinxupquote{BaseSplitter}}}
Bases: \sphinxcode{\sphinxupquote{sklearn.model\_selection.\_split.BaseCrossValidator}}

Class functioning as a base splitter with methods for organizing output and evaluating any mastml data splitter
\begin{description}
\item[{Args:}] \leavevmode
None

\item[{Methods:}] \leavevmode\begin{description}
\item[{split\_asframe: method to perform split into train indices and test indices, but return as dataframes}] \leavevmode\begin{description}
\item[{Args:}] \leavevmode
X: (pd.DataFrame), dataframe of X features

y: (pd.Series), series of y target data

groups: (pd.Series), series of group designations

\item[{Returns:}] \leavevmode
X\_splits: (list), list of dataframes for X splits

y\_splits: (list), list of dataframes for y splits

\end{description}

\item[{evaluate: main method to evaluate a sequence of models, selectors, and hyperparameter optimizers, build directories and perform analysis and output plots}] \leavevmode\begin{description}
\item[{Args:}] \leavevmode
X: (pd.DataFrame), dataframe of X features

y: (pd.Series), series of y target data

models: (list), list containing mastml.models instances

preprocessor: (mastml.preprocessor), mastml.preprocessor object to normalize the training data in each split.

groups: (pd.Series), series of group designations

hyperopts: (list), list containing mastml.hyperopt instances. One for each provided model is needed.

selectors: (list), list containing mastml.feature\_selectors instances

metrics: (list), list of metric names to evaluate true vs. pred data in each split

plots: (list), list of names denoting which types of plots to make. Valid names are ‘Scatter’, ‘Error’, and ‘Histogram’

savepath: (str), string containing main savepath to construct splits for saving output

X\_extra: (pd.DataFrame), dataframe of extra X data not used in model fitting

leaveout\_inds: (list), list of arrays containing indices of data to be held out and evaluated using best model from set of train/validation splits

best\_run\_metric: (str), metric name to be used to decide which model performed best. Defaults to first listed metric in metrics.

nested\_CV: (bool), whether to perform nested cross-validation. The nesting is done using the same splitter object as self.splitter

error\_method: (str), the type of model error evaluation method to perform. Only applies to certain models. Valid names are ‘stdev\_weak\_learners’ and ‘jackknife\_after\_bootstrap’

remove\_outlier\_learners: (bool), whether to remove weak learners from ensemble models whose predictions are found to be outliers. Default False.

recalibrate\_errors: (bool), whether to perform the predicted error bar recalibration method of Palmer et al. Default False.

verbosity: (int), the output plotting verbosity. Default is 1. Valid choices are 0, 1, 2, and 3.

\item[{Returns:}] \leavevmode
None

\end{description}

\item[{\_evaluate\_split\_sets: method to evaluate a set of train/test splits. At the end of the split set, the left-out data (if any) is evaluated using the best model from the train/test splits}] \leavevmode\begin{description}
\item[{Args:}] \leavevmode
X\_splits: (list), list of dataframes for X splits

y\_splits: (list), list of dataframes for y splits

train\_inds: (list), list of arrays of indices denoting the training data

test\_inds: (list), list of arrays of indices denoting the testing data

model: (mastml.models instance), an estimator for fitting data

model\_name: (str), class name of the model being evaluated

selector: (mastml.selector), a feature selector to select features in each split

preprocessor: (mastml.preprocessor), mastml.preprocessor object to normalize the training data in each split.

X\_extra: (pd.DataFrame), dataframe of extra X data not used in model fitting

groups: (pd.Series), series of group designations

splitdir: (str), string denoting the split path in the save directory

hyperopt: (mastml.hyperopt), mastml.hyperopt instance to perform model hyperparameter optimization in each split

metrics: (list), list of metric names to evaluate true vs. pred data in each split

plots: (list), list of names denoting which types of plots to make. Valid names are ‘Scatter’, ‘Error’, and ‘Histogram’

has\_model\_errors: (bool), whether the model used has error bars (uncertainty quantification)

error\_method: (str), the type of model error evaluation method to perform. Only applies to certain models. Valid names are ‘stdev\_weak\_learners’ and ‘jackknife\_after\_bootstrap’

remove\_outlier\_learners: (bool), whether to remove weak learners from ensemble models whose predictions are found to be outliers. Default False.

recalibrate\_errors: (bool), whether to perform the predicted error bar recalibration method of Palmer et al. Default False.

verbosity: (int), the output plotting verbosity. Default is 1. Valid choices are 0, 1, 2, and 3.

\item[{Returns:}] \leavevmode
None

\end{description}

\item[{\_evaluate\_split: method to evaluate a single data split, i.e. fit model, predict test data, and perform some plots and analysis}] \leavevmode\begin{description}
\item[{Args:}] \leavevmode
X\_train: (pd.DataFrame), dataframe of X training features

X\_test: (pd.DataFrame), dataframe of X test features

y\_train: (pd.Series), series of y training features

y\_test: (pd.Series), series of y test features

model: (mastml.models instance), an estimator for fitting data

model\_name: (str), class name of the model being evaluated

preprocessor: (mastml.preprocessor), mastml.preprocessor object to normalize the training data in each split.

selector: (mastml.selector), a feature selector to select features in each split

hyperopt: (mastml.hyperopt), mastml.hyperopt instance to perform model hyperparameter optimization in each split

metrics: (list), list of metric names to evaluate true vs. pred data in each split

plots: (list), list of names denoting which types of plots to make. Valid names are ‘Scatter’, ‘Error’, and ‘Histogram’

groups: (str), string denoting the test group, if applicable

splitpath:(str), string denoting the split path in the save directory

has\_model\_errors: (bool), whether the model used has error bars (uncertainty quantification)

X\_extra\_train: (pd.DataFrame), dataframe of the extra X data of the training split (not used in fit)

X\_extra\_test: (pd.DataFrame), dataframe of the extra X data of the testing split (not used in fit)

error\_method: (str), the type of model error evaluation method to perform. Only applies to certain models. Valid names are ‘stdev\_weak\_learners’ and ‘jackknife\_after\_bootstrap’

remove\_outlier\_learners: (bool), whether to remove weak learners from ensemble models whose predictions are found to be outliers. Default False.

verbosity: (int), the output plotting verbosity. Default is 1. Valid choices are 0, 1, 2, and 3.

\item[{Returns:}] \leavevmode
None

\end{description}

\item[{\_setup\_savedir: method to create a save directory based on model/selector/preprocessor names}] \leavevmode\begin{description}
\item[{Args:}] \leavevmode
model: (mastml.models instance), an estimator for fitting data

preprocessor: (mastml.preprocessor), mastml.preprocessor object to normalize the training data in each split.

selector: (mastml.selector), a feature selector to select features in each split

savepath: (str), string denoting the save path of the file

\end{description}

\item[{\_save\_split\_data: method to save the X and y split data to excel files}] \leavevmode\begin{description}
\item[{Args:}] \leavevmode
df: (pd.DataFrame), dataframe of X or y data to save to file

filename: (str), string denoting the filename, e.g. ‘Xtest’

savepath: (str), string denoting the save path of the file

columns: (list), list of dataframe column names, e.g. X feature names

\item[{Returns:}] \leavevmode
None

\end{description}

\item[{\_collect\_data: method to collect all pd.Series (e.g. ytrain/ytest) data into single series over many splits (directories)}] \leavevmode\begin{description}
\item[{Args:}] \leavevmode
filename: (str), string denoting the filename, e.g. ‘ytest’

savepath: (str), string denoting the save path of the file

\item[{Returns:}] \leavevmode
data: (list), list containing flattened array of all data of a given type over many splits, e.g. all ypred data

\end{description}

\item[{\_collect\_df\_data: method to collect all pd.DataFrame (e.g. Xtrain/Xtest) data into single dataframe over many splits (directories)}] \leavevmode\begin{description}
\item[{Args:}] \leavevmode
filename: (str), string denoting the filename, e.g. ‘Xtest’

savepath: (str), string denoting the save path of the file

\item[{Returns:}] \leavevmode
data: (list), list containing flattened array of all data of a given type over many splits, e.g. all Xtest data

\end{description}

\item[{\_get\_best\_split: method to find the best performing model in a set of train/test splits}] \leavevmode\begin{description}
\item[{Args:}] \leavevmode
savepath: (str), string denoting the save path of the file

preprocessor: (mastml.preprocessor), mastml.preprocessor object to normalize the training data in each split.

best\_run\_metric: (str), name of the metric to use to find the best performing model

model\_name: (str), class name of model being evaluated

\item[{Returns:}] \leavevmode
best\_split\_dict: (dict), dictionary containing the path locations of the best model and corresponding preprocessor and selected feature list

\end{description}

\item[{\_get\_average\_recalibration\_params: method to get the average and standard deviation of the recalibration factors in all train/test CV sets}] \leavevmode\begin{description}
\item[{Args:}] \leavevmode
savepath: (str), string denoting the save path of the file

data\_type: (str), string denoting the type of data to examine (e.g. test or leftout)

\item[{Returns:}] \leavevmode
recalibrate\_avg\_dict: (dict): dictionary of average recalibration parameters

recalibrate\_stdev\_dict: (dict): dictionary of stdev of recalibration parameters

\end{description}

\item[{\_get\_recalibration\_params: method to get the recalibration factors for a single evaluation}] \leavevmode\begin{description}
\item[{Args:}] \leavevmode
savepath: (str), string denoting the save path of the file

data\_type: (str), string denoting the type of data to examine (e.g. test or leftout)

\item[{Returns:}] \leavevmode
recalibrate\_dict: (dict): dictionary of recalibration parameters

\end{description}

\item[{help: method to output key information on class use, e.g. methods and parameters}] \leavevmode\begin{description}
\item[{Args:}] \leavevmode
None

\item[{Returns:}] \leavevmode
None, but outputs help to screen

\end{description}

\end{description}

\end{description}
\subsubsection*{Methods Summary}


\begin{savenotes}\sphinxatlongtablestart\begin{longtable}[c]{\X{1}{2}\X{1}{2}}
\hline

\endfirsthead

\multicolumn{2}{c}%
{\makebox[0pt]{\sphinxtablecontinued{\tablename\ \thetable{} -- continued from previous page}}}\\
\hline

\endhead

\hline
\multicolumn{2}{r}{\makebox[0pt][r]{\sphinxtablecontinued{Continued on next page}}}\\
\endfoot

\endlastfoot

{\hyperref[\detokenize{api/mastml.data_splitters.BaseSplitter:mastml.data_splitters.BaseSplitter.evaluate}]{\sphinxcrossref{\sphinxcode{\sphinxupquote{evaluate}}}}}(X, y, models{[}, preprocessor, …{]})
&

\\
\hline
{\hyperref[\detokenize{api/mastml.data_splitters.BaseSplitter:mastml.data_splitters.BaseSplitter.help}]{\sphinxcrossref{\sphinxcode{\sphinxupquote{help}}}}}()
&

\\
\hline
{\hyperref[\detokenize{api/mastml.data_splitters.BaseSplitter:mastml.data_splitters.BaseSplitter.split_asframe}]{\sphinxcrossref{\sphinxcode{\sphinxupquote{split\_asframe}}}}}(X, y{[}, groups{]})
&

\\
\hline
\end{longtable}\sphinxatlongtableend\end{savenotes}
\subsubsection*{Methods Documentation}
\index{evaluate() (mastml.data\_splitters.BaseSplitter method)@\spxentry{evaluate()}\spxextra{mastml.data\_splitters.BaseSplitter method}}

\begin{fulllineitems}
\phantomsection\label{\detokenize{api/mastml.data_splitters.BaseSplitter:mastml.data_splitters.BaseSplitter.evaluate}}\pysiglinewithargsret{\sphinxbfcode{\sphinxupquote{evaluate}}}{\emph{X}, \emph{y}, \emph{models}, \emph{preprocessor=None}, \emph{groups=None}, \emph{hyperopts=None}, \emph{selectors=None}, \emph{metrics=None}, \emph{plots=None}, \emph{savepath=None}, \emph{X\_extra=None}, \emph{leaveout\_inds={[}{]}}, \emph{best\_run\_metric=None}, \emph{nested\_CV=False}, \emph{error\_method='stdev\_weak\_learners'}, \emph{remove\_outlier\_learners=False}, \emph{recalibrate\_errors=False}, \emph{verbosity=1}}{}
\end{fulllineitems}

\index{help() (mastml.data\_splitters.BaseSplitter method)@\spxentry{help()}\spxextra{mastml.data\_splitters.BaseSplitter method}}

\begin{fulllineitems}
\phantomsection\label{\detokenize{api/mastml.data_splitters.BaseSplitter:mastml.data_splitters.BaseSplitter.help}}\pysiglinewithargsret{\sphinxbfcode{\sphinxupquote{help}}}{}{}
\end{fulllineitems}

\index{split\_asframe() (mastml.data\_splitters.BaseSplitter method)@\spxentry{split\_asframe()}\spxextra{mastml.data\_splitters.BaseSplitter method}}

\begin{fulllineitems}
\phantomsection\label{\detokenize{api/mastml.data_splitters.BaseSplitter:mastml.data_splitters.BaseSplitter.split_asframe}}\pysiglinewithargsret{\sphinxbfcode{\sphinxupquote{split\_asframe}}}{\emph{X}, \emph{y}, \emph{groups=None}}{}
\end{fulllineitems}


\end{fulllineitems}



\subsubsection{Bootstrap}
\label{\detokenize{api/mastml.data_splitters.Bootstrap:bootstrap}}\label{\detokenize{api/mastml.data_splitters.Bootstrap::doc}}\index{Bootstrap (class in mastml.data\_splitters)@\spxentry{Bootstrap}\spxextra{class in mastml.data\_splitters}}

\begin{fulllineitems}
\phantomsection\label{\detokenize{api/mastml.data_splitters.Bootstrap:mastml.data_splitters.Bootstrap}}\pysiglinewithargsret{\sphinxbfcode{\sphinxupquote{class }}\sphinxcode{\sphinxupquote{mastml.data\_splitters.}}\sphinxbfcode{\sphinxupquote{Bootstrap}}}{\emph{n}, \emph{n\_bootstraps=3}, \emph{train\_size=0.5}, \emph{test\_size=None}, \emph{n\_train=None}, \emph{n\_test=None}, \emph{random\_state=0}}{}
Bases: {\hyperref[\detokenize{api/mastml.data_splitters.BaseSplitter:mastml.data_splitters.BaseSplitter}]{\sphinxcrossref{\sphinxcode{\sphinxupquote{mastml.data\_splitters.BaseSplitter}}}}}

\# Note: Bootstrap taken directly from sklearn Github (\sphinxurl{https://github.com/scikit-learn/scikit-learn/blob/0.11.X/sklearn/cross\_validation.py})
\# which was necessary as it was later removed from more recent sklearn releases
Random sampling with replacement cross-validation iterator
Provides train/test indices to split data in train test sets
while resampling the input n\_bootstraps times: each time a new
random split of the data is performed and then samples are drawn
(with replacement) on each side of the split to build the training
and test sets.
Note: contrary to other cross-validation strategies, bootstrapping
will allow some samples to occur several times in each splits. However
a sample that occurs in the train split will never occur in the test
split and vice-versa.
If you want each sample to occur at most once you should probably
use ShuffleSplit cross validation instead.
\begin{description}
\item[{Args:}] \leavevmode
n: (int), total number of elements in the dataset

n\_bootstraps: (int), (default is 3) Number of bootstrapping iterations
\begin{description}
\item[{train\_size: (int or float), (default is 0.5) If int, number of samples to include in the training split}] \leavevmode
(should be smaller than the total number of samples passed in the dataset).
If float, should be between 0.0 and 1.0 and represent the
proportion of the dataset to include in the train split.

\item[{test\_size: (int or float or None), (default is None)}] \leavevmode
If int, number of samples to include in the training set
(should be smaller than the total number of samples passed
in the dataset).
If float, should be between 0.0 and 1.0 and represent the
proportion of the dataset to include in the test split.
If None, n\_test is set as the complement of n\_train.

\end{description}

random\_state: (int or RandomState), Pseudo number generator state used for random sampling.

\end{description}
\subsubsection*{Attributes Summary}


\begin{savenotes}\sphinxatlongtablestart\begin{longtable}[c]{\X{1}{2}\X{1}{2}}
\hline

\endfirsthead

\multicolumn{2}{c}%
{\makebox[0pt]{\sphinxtablecontinued{\tablename\ \thetable{} -- continued from previous page}}}\\
\hline

\endhead

\hline
\multicolumn{2}{r}{\makebox[0pt][r]{\sphinxtablecontinued{Continued on next page}}}\\
\endfoot

\endlastfoot

{\hyperref[\detokenize{api/mastml.data_splitters.Bootstrap:mastml.data_splitters.Bootstrap.indices}]{\sphinxcrossref{\sphinxcode{\sphinxupquote{indices}}}}}
&

\\
\hline
\end{longtable}\sphinxatlongtableend\end{savenotes}
\subsubsection*{Methods Summary}


\begin{savenotes}\sphinxatlongtablestart\begin{longtable}[c]{\X{1}{2}\X{1}{2}}
\hline

\endfirsthead

\multicolumn{2}{c}%
{\makebox[0pt]{\sphinxtablecontinued{\tablename\ \thetable{} -- continued from previous page}}}\\
\hline

\endhead

\hline
\multicolumn{2}{r}{\makebox[0pt][r]{\sphinxtablecontinued{Continued on next page}}}\\
\endfoot

\endlastfoot

{\hyperref[\detokenize{api/mastml.data_splitters.Bootstrap:mastml.data_splitters.Bootstrap.get_n_splits}]{\sphinxcrossref{\sphinxcode{\sphinxupquote{get\_n\_splits}}}}}({[}X, y, groups{]})
&
Returns the number of splitting iterations in the cross-validator
\\
\hline
{\hyperref[\detokenize{api/mastml.data_splitters.Bootstrap:mastml.data_splitters.Bootstrap.split}]{\sphinxcrossref{\sphinxcode{\sphinxupquote{split}}}}}(X{[}, y, groups{]})
&
Generate indices to split data into training and test set.
\\
\hline
\end{longtable}\sphinxatlongtableend\end{savenotes}
\subsubsection*{Attributes Documentation}
\index{indices (mastml.data\_splitters.Bootstrap attribute)@\spxentry{indices}\spxextra{mastml.data\_splitters.Bootstrap attribute}}

\begin{fulllineitems}
\phantomsection\label{\detokenize{api/mastml.data_splitters.Bootstrap:mastml.data_splitters.Bootstrap.indices}}\pysigline{\sphinxbfcode{\sphinxupquote{indices}}\sphinxbfcode{\sphinxupquote{ = True}}}
\end{fulllineitems}

\subsubsection*{Methods Documentation}
\index{get\_n\_splits() (mastml.data\_splitters.Bootstrap method)@\spxentry{get\_n\_splits()}\spxextra{mastml.data\_splitters.Bootstrap method}}

\begin{fulllineitems}
\phantomsection\label{\detokenize{api/mastml.data_splitters.Bootstrap:mastml.data_splitters.Bootstrap.get_n_splits}}\pysiglinewithargsret{\sphinxbfcode{\sphinxupquote{get\_n\_splits}}}{\emph{X=None}, \emph{y=None}, \emph{groups=None}}{}
Returns the number of splitting iterations in the cross-validator

\end{fulllineitems}

\index{split() (mastml.data\_splitters.Bootstrap method)@\spxentry{split()}\spxextra{mastml.data\_splitters.Bootstrap method}}

\begin{fulllineitems}
\phantomsection\label{\detokenize{api/mastml.data_splitters.Bootstrap:mastml.data_splitters.Bootstrap.split}}\pysiglinewithargsret{\sphinxbfcode{\sphinxupquote{split}}}{\emph{X}, \emph{y=None}, \emph{groups=None}}{}
Generate indices to split data into training and test set.
\begin{description}
\item[{X}] \leavevmode{[}array-like of shape (n\_samples, n\_features){]}
Training data, where n\_samples is the number of samples
and n\_features is the number of features.

\item[{y}] \leavevmode{[}array-like of shape (n\_samples,){]}
The target variable for supervised learning problems.

\item[{groups}] \leavevmode{[}array-like of shape (n\_samples,), default=None{]}
Group labels for the samples used while splitting the dataset into
train/test set.

\end{description}
\begin{description}
\item[{train}] \leavevmode{[}ndarray{]}
The training set indices for that split.

\item[{test}] \leavevmode{[}ndarray{]}
The testing set indices for that split.

\end{description}

\end{fulllineitems}


\end{fulllineitems}



\subsubsection{JustEachGroup}
\label{\detokenize{api/mastml.data_splitters.JustEachGroup:justeachgroup}}\label{\detokenize{api/mastml.data_splitters.JustEachGroup::doc}}\index{JustEachGroup (class in mastml.data\_splitters)@\spxentry{JustEachGroup}\spxextra{class in mastml.data\_splitters}}

\begin{fulllineitems}
\phantomsection\label{\detokenize{api/mastml.data_splitters.JustEachGroup:mastml.data_splitters.JustEachGroup}}\pysigline{\sphinxbfcode{\sphinxupquote{class }}\sphinxcode{\sphinxupquote{mastml.data\_splitters.}}\sphinxbfcode{\sphinxupquote{JustEachGroup}}}
Bases: {\hyperref[\detokenize{api/mastml.data_splitters.BaseSplitter:mastml.data_splitters.BaseSplitter}]{\sphinxcrossref{\sphinxcode{\sphinxupquote{mastml.data\_splitters.BaseSplitter}}}}}

Class to train the model on one group at a time and test it on the rest of the data
This class wraps scikit-learn’s LeavePGroupsOut with P set to n-1. More information is available at:
\sphinxurl{http://scikit-learn.org/stable/modules/generated/sklearn.model\_selection.LeavePGroupsOut.html}
\begin{description}
\item[{Args:}] \leavevmode
None (only object instance)

\item[{Methods:}] \leavevmode\begin{description}
\item[{get\_n\_splits: method to calculate the number of splits to perform}] \leavevmode\begin{description}
\item[{Args:}] \leavevmode
groups: (numpy array), array of group labels

\item[{Returns:}] \leavevmode
(int), number of unique groups, indicating number of splits to perform

\end{description}

\item[{split: method to perform split into train indices and test indices}] \leavevmode\begin{description}
\item[{Args:}] \leavevmode
X: (numpy array), array of X features

y: (numpy array), array of y data

groups: (numpy array), array of group labels

\item[{Returns:}] \leavevmode
(numpy array), array of train and test indices

\end{description}

\end{description}

\end{description}
\subsubsection*{Methods Summary}


\begin{savenotes}\sphinxatlongtablestart\begin{longtable}[c]{\X{1}{2}\X{1}{2}}
\hline

\endfirsthead

\multicolumn{2}{c}%
{\makebox[0pt]{\sphinxtablecontinued{\tablename\ \thetable{} -- continued from previous page}}}\\
\hline

\endhead

\hline
\multicolumn{2}{r}{\makebox[0pt][r]{\sphinxtablecontinued{Continued on next page}}}\\
\endfoot

\endlastfoot

{\hyperref[\detokenize{api/mastml.data_splitters.JustEachGroup:mastml.data_splitters.JustEachGroup.get_n_splits}]{\sphinxcrossref{\sphinxcode{\sphinxupquote{get\_n\_splits}}}}}({[}X, y, groups{]})
&
Returns the number of splitting iterations in the cross-validator
\\
\hline
{\hyperref[\detokenize{api/mastml.data_splitters.JustEachGroup:mastml.data_splitters.JustEachGroup.split}]{\sphinxcrossref{\sphinxcode{\sphinxupquote{split}}}}}(X, y, groups)
&
Generate indices to split data into training and test set.
\\
\hline
\end{longtable}\sphinxatlongtableend\end{savenotes}
\subsubsection*{Methods Documentation}
\index{get\_n\_splits() (mastml.data\_splitters.JustEachGroup method)@\spxentry{get\_n\_splits()}\spxextra{mastml.data\_splitters.JustEachGroup method}}

\begin{fulllineitems}
\phantomsection\label{\detokenize{api/mastml.data_splitters.JustEachGroup:mastml.data_splitters.JustEachGroup.get_n_splits}}\pysiglinewithargsret{\sphinxbfcode{\sphinxupquote{get\_n\_splits}}}{\emph{X=None}, \emph{y=None}, \emph{groups=None}}{}
Returns the number of splitting iterations in the cross-validator

\end{fulllineitems}

\index{split() (mastml.data\_splitters.JustEachGroup method)@\spxentry{split()}\spxextra{mastml.data\_splitters.JustEachGroup method}}

\begin{fulllineitems}
\phantomsection\label{\detokenize{api/mastml.data_splitters.JustEachGroup:mastml.data_splitters.JustEachGroup.split}}\pysiglinewithargsret{\sphinxbfcode{\sphinxupquote{split}}}{\emph{X}, \emph{y}, \emph{groups}}{}
Generate indices to split data into training and test set.
\begin{description}
\item[{X}] \leavevmode{[}array-like of shape (n\_samples, n\_features){]}
Training data, where n\_samples is the number of samples
and n\_features is the number of features.

\item[{y}] \leavevmode{[}array-like of shape (n\_samples,){]}
The target variable for supervised learning problems.

\item[{groups}] \leavevmode{[}array-like of shape (n\_samples,), default=None{]}
Group labels for the samples used while splitting the dataset into
train/test set.

\end{description}
\begin{description}
\item[{train}] \leavevmode{[}ndarray{]}
The training set indices for that split.

\item[{test}] \leavevmode{[}ndarray{]}
The testing set indices for that split.

\end{description}

\end{fulllineitems}


\end{fulllineitems}



\subsubsection{LeaveCloseCompositionsOut}
\label{\detokenize{api/mastml.data_splitters.LeaveCloseCompositionsOut:leaveclosecompositionsout}}\label{\detokenize{api/mastml.data_splitters.LeaveCloseCompositionsOut::doc}}\index{LeaveCloseCompositionsOut (class in mastml.data\_splitters)@\spxentry{LeaveCloseCompositionsOut}\spxextra{class in mastml.data\_splitters}}

\begin{fulllineitems}
\phantomsection\label{\detokenize{api/mastml.data_splitters.LeaveCloseCompositionsOut:mastml.data_splitters.LeaveCloseCompositionsOut}}\pysiglinewithargsret{\sphinxbfcode{\sphinxupquote{class }}\sphinxcode{\sphinxupquote{mastml.data\_splitters.}}\sphinxbfcode{\sphinxupquote{LeaveCloseCompositionsOut}}}{\emph{composition\_df}, \emph{dist\_threshold=0.1}, \emph{nn\_kwargs=None}}{}
Bases: {\hyperref[\detokenize{api/mastml.data_splitters.BaseSplitter:mastml.data_splitters.BaseSplitter}]{\sphinxcrossref{\sphinxcode{\sphinxupquote{mastml.data\_splitters.BaseSplitter}}}}}

Leave-P-out where you exclude materials with compositions close to those the test set

Computes the distance between the element fraction vectors. For example, the \(L_2\)
distance between Al and Cu is \(\sqrt{2}\) and the \(L_1\) distance between Al
and Al0.9Cu0.1 is 0.2.

Consequently, this splitter requires a list of compositions as the input to \sphinxtitleref{split} rather
than the features.
\begin{description}
\item[{Args:}] \leavevmode
composition\_df (pd.DataFrame): dataframe containing the vector of material compositions to analyze

dist\_threshold (float): Entries must be farther than this distance to be included in the training set

nn\_kwargs (dict): Keyword arguments for the scikit-learn NearestNeighbor class used to find nearest points

\end{description}
\subsubsection*{Methods Summary}


\begin{savenotes}\sphinxatlongtablestart\begin{longtable}[c]{\X{1}{2}\X{1}{2}}
\hline

\endfirsthead

\multicolumn{2}{c}%
{\makebox[0pt]{\sphinxtablecontinued{\tablename\ \thetable{} -- continued from previous page}}}\\
\hline

\endhead

\hline
\multicolumn{2}{r}{\makebox[0pt][r]{\sphinxtablecontinued{Continued on next page}}}\\
\endfoot

\endlastfoot

{\hyperref[\detokenize{api/mastml.data_splitters.LeaveCloseCompositionsOut:mastml.data_splitters.LeaveCloseCompositionsOut.get_n_splits}]{\sphinxcrossref{\sphinxcode{\sphinxupquote{get\_n\_splits}}}}}({[}X, y, groups{]})
&
Returns the number of splitting iterations in the cross-validator
\\
\hline
{\hyperref[\detokenize{api/mastml.data_splitters.LeaveCloseCompositionsOut:mastml.data_splitters.LeaveCloseCompositionsOut.split}]{\sphinxcrossref{\sphinxcode{\sphinxupquote{split}}}}}(X{[}, y, groups{]})
&
Generate indices to split data into training and test set.
\\
\hline
\end{longtable}\sphinxatlongtableend\end{savenotes}
\subsubsection*{Methods Documentation}
\index{get\_n\_splits() (mastml.data\_splitters.LeaveCloseCompositionsOut method)@\spxentry{get\_n\_splits()}\spxextra{mastml.data\_splitters.LeaveCloseCompositionsOut method}}

\begin{fulllineitems}
\phantomsection\label{\detokenize{api/mastml.data_splitters.LeaveCloseCompositionsOut:mastml.data_splitters.LeaveCloseCompositionsOut.get_n_splits}}\pysiglinewithargsret{\sphinxbfcode{\sphinxupquote{get\_n\_splits}}}{\emph{X=None}, \emph{y=None}, \emph{groups=None}}{}
Returns the number of splitting iterations in the cross-validator

\end{fulllineitems}

\index{split() (mastml.data\_splitters.LeaveCloseCompositionsOut method)@\spxentry{split()}\spxextra{mastml.data\_splitters.LeaveCloseCompositionsOut method}}

\begin{fulllineitems}
\phantomsection\label{\detokenize{api/mastml.data_splitters.LeaveCloseCompositionsOut:mastml.data_splitters.LeaveCloseCompositionsOut.split}}\pysiglinewithargsret{\sphinxbfcode{\sphinxupquote{split}}}{\emph{X}, \emph{y=None}, \emph{groups=None}}{}
Generate indices to split data into training and test set.
\begin{description}
\item[{X}] \leavevmode{[}array-like of shape (n\_samples, n\_features){]}
Training data, where n\_samples is the number of samples
and n\_features is the number of features.

\item[{y}] \leavevmode{[}array-like of shape (n\_samples,){]}
The target variable for supervised learning problems.

\item[{groups}] \leavevmode{[}array-like of shape (n\_samples,), default=None{]}
Group labels for the samples used while splitting the dataset into
train/test set.

\end{description}
\begin{description}
\item[{train}] \leavevmode{[}ndarray{]}
The training set indices for that split.

\item[{test}] \leavevmode{[}ndarray{]}
The testing set indices for that split.

\end{description}

\end{fulllineitems}


\end{fulllineitems}



\subsubsection{LeaveOutPercent}
\label{\detokenize{api/mastml.data_splitters.LeaveOutPercent:leaveoutpercent}}\label{\detokenize{api/mastml.data_splitters.LeaveOutPercent::doc}}\index{LeaveOutPercent (class in mastml.data\_splitters)@\spxentry{LeaveOutPercent}\spxextra{class in mastml.data\_splitters}}

\begin{fulllineitems}
\phantomsection\label{\detokenize{api/mastml.data_splitters.LeaveOutPercent:mastml.data_splitters.LeaveOutPercent}}\pysiglinewithargsret{\sphinxbfcode{\sphinxupquote{class }}\sphinxcode{\sphinxupquote{mastml.data\_splitters.}}\sphinxbfcode{\sphinxupquote{LeaveOutPercent}}}{\emph{percent\_leave\_out=0.2}, \emph{n\_repeats=5}}{}
Bases: {\hyperref[\detokenize{api/mastml.data_splitters.BaseSplitter:mastml.data_splitters.BaseSplitter}]{\sphinxcrossref{\sphinxcode{\sphinxupquote{mastml.data\_splitters.BaseSplitter}}}}}

Class to train the model using a certain percentage of data as training data
\begin{description}
\item[{Args:}] \leavevmode
percent\_leave\_out (float): fraction of data to use in training (must be \textgreater{} 0 and \textless{} 1)

n\_repeats (int): number of repeated splits to perform (must be \textgreater{}= 1)

\item[{Methods:}] \leavevmode\begin{description}
\item[{get\_n\_splits: method to return the number of splits to perform}] \leavevmode\begin{description}
\item[{Args:}] \leavevmode
groups: (numpy array), array of group labels

\item[{Returns:}] \leavevmode
(int), number of unique groups, indicating number of splits to perform

\end{description}

\item[{split: method to perform split into train indices and test indices}] \leavevmode\begin{description}
\item[{Args:}] \leavevmode
X: (numpy array), array of X features

y: (numpy array), array of y data

groups: (numpy array), array of group labels

\item[{Returns:}] \leavevmode
(numpy array), array of train and test indices

\end{description}

\end{description}

\end{description}
\subsubsection*{Methods Summary}


\begin{savenotes}\sphinxatlongtablestart\begin{longtable}[c]{\X{1}{2}\X{1}{2}}
\hline

\endfirsthead

\multicolumn{2}{c}%
{\makebox[0pt]{\sphinxtablecontinued{\tablename\ \thetable{} -- continued from previous page}}}\\
\hline

\endhead

\hline
\multicolumn{2}{r}{\makebox[0pt][r]{\sphinxtablecontinued{Continued on next page}}}\\
\endfoot

\endlastfoot

{\hyperref[\detokenize{api/mastml.data_splitters.LeaveOutPercent:mastml.data_splitters.LeaveOutPercent.get_n_splits}]{\sphinxcrossref{\sphinxcode{\sphinxupquote{get\_n\_splits}}}}}({[}X, y, groups{]})
&
Returns the number of splitting iterations in the cross-validator
\\
\hline
{\hyperref[\detokenize{api/mastml.data_splitters.LeaveOutPercent:mastml.data_splitters.LeaveOutPercent.split}]{\sphinxcrossref{\sphinxcode{\sphinxupquote{split}}}}}(X{[}, y, groups{]})
&
Generate indices to split data into training and test set.
\\
\hline
\end{longtable}\sphinxatlongtableend\end{savenotes}
\subsubsection*{Methods Documentation}
\index{get\_n\_splits() (mastml.data\_splitters.LeaveOutPercent method)@\spxentry{get\_n\_splits()}\spxextra{mastml.data\_splitters.LeaveOutPercent method}}

\begin{fulllineitems}
\phantomsection\label{\detokenize{api/mastml.data_splitters.LeaveOutPercent:mastml.data_splitters.LeaveOutPercent.get_n_splits}}\pysiglinewithargsret{\sphinxbfcode{\sphinxupquote{get\_n\_splits}}}{\emph{X=None}, \emph{y=None}, \emph{groups=None}}{}
Returns the number of splitting iterations in the cross-validator

\end{fulllineitems}

\index{split() (mastml.data\_splitters.LeaveOutPercent method)@\spxentry{split()}\spxextra{mastml.data\_splitters.LeaveOutPercent method}}

\begin{fulllineitems}
\phantomsection\label{\detokenize{api/mastml.data_splitters.LeaveOutPercent:mastml.data_splitters.LeaveOutPercent.split}}\pysiglinewithargsret{\sphinxbfcode{\sphinxupquote{split}}}{\emph{X}, \emph{y=None}, \emph{groups=None}}{}
Generate indices to split data into training and test set.
\begin{description}
\item[{X}] \leavevmode{[}array-like of shape (n\_samples, n\_features){]}
Training data, where n\_samples is the number of samples
and n\_features is the number of features.

\item[{y}] \leavevmode{[}array-like of shape (n\_samples,){]}
The target variable for supervised learning problems.

\item[{groups}] \leavevmode{[}array-like of shape (n\_samples,), default=None{]}
Group labels for the samples used while splitting the dataset into
train/test set.

\end{description}
\begin{description}
\item[{train}] \leavevmode{[}ndarray{]}
The training set indices for that split.

\item[{test}] \leavevmode{[}ndarray{]}
The testing set indices for that split.

\end{description}

\end{fulllineitems}


\end{fulllineitems}



\subsubsection{LeaveOutTwinCV}
\label{\detokenize{api/mastml.data_splitters.LeaveOutTwinCV:leaveouttwincv}}\label{\detokenize{api/mastml.data_splitters.LeaveOutTwinCV::doc}}\index{LeaveOutTwinCV (class in mastml.data\_splitters)@\spxentry{LeaveOutTwinCV}\spxextra{class in mastml.data\_splitters}}

\begin{fulllineitems}
\phantomsection\label{\detokenize{api/mastml.data_splitters.LeaveOutTwinCV:mastml.data_splitters.LeaveOutTwinCV}}\pysiglinewithargsret{\sphinxbfcode{\sphinxupquote{class }}\sphinxcode{\sphinxupquote{mastml.data\_splitters.}}\sphinxbfcode{\sphinxupquote{LeaveOutTwinCV}}}{\emph{threshold=0}, \emph{ord=2}, \emph{debug=False}, \emph{auto\_threshold=False}, \emph{ceiling=0}}{}
Bases: {\hyperref[\detokenize{api/mastml.data_splitters.BaseSplitter:mastml.data_splitters.BaseSplitter}]{\sphinxcrossref{\sphinxcode{\sphinxupquote{mastml.data\_splitters.BaseSplitter}}}}}

Class to remove data twins from the test data.
\begin{description}
\item[{Args:}] \leavevmode
threshold: (int), the threshold at which two data points are considered twins. Default 0.

ord: (int), The order of the norm of the difference (see scipy.spatial.distance.minkowski). Default 2 (Euclidean Distance).

auto\_threshold: (boolean), true if threshold should be automatically increased until twins corresponding to the ceiling parameter are found. Default False.
ceiling: (float), fraction of total data to find as twins. Default 0.

\item[{Methods:}] \leavevmode\begin{description}
\item[{get\_n\_splits: method to calculate the number of splits to perform across all splitters}] \leavevmode\begin{description}
\item[{Args:}] \leavevmode
X: (numpy array), array of X features

y: (numpy array), array of y data

groups: (numpy array), array of group labels

\item[{Returns:}] \leavevmode
(int), the number 1 always

\end{description}

\item[{split: method to perform split into train indices and test indices}] \leavevmode\begin{description}
\item[{Args:}] \leavevmode
X: (numpy array), array of X features

y: (numpy array), array of y data

groups: (numpy array), array of group labels

\item[{Returns:}] \leavevmode
(numpy array), array of train and test indices

\end{description}

\end{description}

\end{description}
\subsubsection*{Methods Summary}


\begin{savenotes}\sphinxatlongtablestart\begin{longtable}[c]{\X{1}{2}\X{1}{2}}
\hline

\endfirsthead

\multicolumn{2}{c}%
{\makebox[0pt]{\sphinxtablecontinued{\tablename\ \thetable{} -- continued from previous page}}}\\
\hline

\endhead

\hline
\multicolumn{2}{r}{\makebox[0pt][r]{\sphinxtablecontinued{Continued on next page}}}\\
\endfoot

\endlastfoot

{\hyperref[\detokenize{api/mastml.data_splitters.LeaveOutTwinCV:mastml.data_splitters.LeaveOutTwinCV.get_n_splits}]{\sphinxcrossref{\sphinxcode{\sphinxupquote{get\_n\_splits}}}}}({[}X, y, groups{]})
&
Returns the number of splitting iterations in the cross-validator
\\
\hline
{\hyperref[\detokenize{api/mastml.data_splitters.LeaveOutTwinCV:mastml.data_splitters.LeaveOutTwinCV.split}]{\sphinxcrossref{\sphinxcode{\sphinxupquote{split}}}}}(X, y{[}, X\_noinput, groups{]})
&
Generate indices to split data into training and test set.
\\
\hline
\end{longtable}\sphinxatlongtableend\end{savenotes}
\subsubsection*{Methods Documentation}
\index{get\_n\_splits() (mastml.data\_splitters.LeaveOutTwinCV method)@\spxentry{get\_n\_splits()}\spxextra{mastml.data\_splitters.LeaveOutTwinCV method}}

\begin{fulllineitems}
\phantomsection\label{\detokenize{api/mastml.data_splitters.LeaveOutTwinCV:mastml.data_splitters.LeaveOutTwinCV.get_n_splits}}\pysiglinewithargsret{\sphinxbfcode{\sphinxupquote{get\_n\_splits}}}{\emph{X=None}, \emph{y=None}, \emph{groups=None}}{}
Returns the number of splitting iterations in the cross-validator

\end{fulllineitems}

\index{split() (mastml.data\_splitters.LeaveOutTwinCV method)@\spxentry{split()}\spxextra{mastml.data\_splitters.LeaveOutTwinCV method}}

\begin{fulllineitems}
\phantomsection\label{\detokenize{api/mastml.data_splitters.LeaveOutTwinCV:mastml.data_splitters.LeaveOutTwinCV.split}}\pysiglinewithargsret{\sphinxbfcode{\sphinxupquote{split}}}{\emph{X}, \emph{y}, \emph{X\_noinput=None}, \emph{groups=None}}{}
Generate indices to split data into training and test set.
\begin{description}
\item[{X}] \leavevmode{[}array-like of shape (n\_samples, n\_features){]}
Training data, where n\_samples is the number of samples
and n\_features is the number of features.

\item[{y}] \leavevmode{[}array-like of shape (n\_samples,){]}
The target variable for supervised learning problems.

\item[{groups}] \leavevmode{[}array-like of shape (n\_samples,), default=None{]}
Group labels for the samples used while splitting the dataset into
train/test set.

\end{description}
\begin{description}
\item[{train}] \leavevmode{[}ndarray{]}
The training set indices for that split.

\item[{test}] \leavevmode{[}ndarray{]}
The testing set indices for that split.

\end{description}

\end{fulllineitems}


\end{fulllineitems}



\subsubsection{NoSplit}
\label{\detokenize{api/mastml.data_splitters.NoSplit:nosplit}}\label{\detokenize{api/mastml.data_splitters.NoSplit::doc}}\index{NoSplit (class in mastml.data\_splitters)@\spxentry{NoSplit}\spxextra{class in mastml.data\_splitters}}

\begin{fulllineitems}
\phantomsection\label{\detokenize{api/mastml.data_splitters.NoSplit:mastml.data_splitters.NoSplit}}\pysigline{\sphinxbfcode{\sphinxupquote{class }}\sphinxcode{\sphinxupquote{mastml.data\_splitters.}}\sphinxbfcode{\sphinxupquote{NoSplit}}}
Bases: {\hyperref[\detokenize{api/mastml.data_splitters.BaseSplitter:mastml.data_splitters.BaseSplitter}]{\sphinxcrossref{\sphinxcode{\sphinxupquote{mastml.data\_splitters.BaseSplitter}}}}}

Class to just train the model on the training data and test it on that same data. Sometimes referred to as a “Full fit”
or a “Single fit”, equivalent to just plotting y vs. x.
\begin{description}
\item[{Args:}] \leavevmode
None (only object instance)

\item[{Methods:}] \leavevmode\begin{description}
\item[{get\_n\_splits: method to calculate the number of splits to perform}] \leavevmode\begin{description}
\item[{Args:}] \leavevmode
None

\item[{Returns:}] \leavevmode
(int), always 1 as only a single split is performed

\end{description}

\item[{split: method to perform split into train indices and test indices}] \leavevmode\begin{description}
\item[{Args:}] \leavevmode
X: (numpy array), array of X features

\item[{Returns:}] \leavevmode
(numpy array), array of train and test indices (all data used as train and test for NoSplit)

\end{description}

\end{description}

\end{description}
\subsubsection*{Methods Summary}


\begin{savenotes}\sphinxatlongtablestart\begin{longtable}[c]{\X{1}{2}\X{1}{2}}
\hline

\endfirsthead

\multicolumn{2}{c}%
{\makebox[0pt]{\sphinxtablecontinued{\tablename\ \thetable{} -- continued from previous page}}}\\
\hline

\endhead

\hline
\multicolumn{2}{r}{\makebox[0pt][r]{\sphinxtablecontinued{Continued on next page}}}\\
\endfoot

\endlastfoot

{\hyperref[\detokenize{api/mastml.data_splitters.NoSplit:mastml.data_splitters.NoSplit.get_n_splits}]{\sphinxcrossref{\sphinxcode{\sphinxupquote{get\_n\_splits}}}}}({[}X, y, groups{]})
&
Returns the number of splitting iterations in the cross-validator
\\
\hline
{\hyperref[\detokenize{api/mastml.data_splitters.NoSplit:mastml.data_splitters.NoSplit.split}]{\sphinxcrossref{\sphinxcode{\sphinxupquote{split}}}}}(X{[}, y, groups{]})
&
Generate indices to split data into training and test set.
\\
\hline
\end{longtable}\sphinxatlongtableend\end{savenotes}
\subsubsection*{Methods Documentation}
\index{get\_n\_splits() (mastml.data\_splitters.NoSplit method)@\spxentry{get\_n\_splits()}\spxextra{mastml.data\_splitters.NoSplit method}}

\begin{fulllineitems}
\phantomsection\label{\detokenize{api/mastml.data_splitters.NoSplit:mastml.data_splitters.NoSplit.get_n_splits}}\pysiglinewithargsret{\sphinxbfcode{\sphinxupquote{get\_n\_splits}}}{\emph{X=None}, \emph{y=None}, \emph{groups=None}}{}
Returns the number of splitting iterations in the cross-validator

\end{fulllineitems}

\index{split() (mastml.data\_splitters.NoSplit method)@\spxentry{split()}\spxextra{mastml.data\_splitters.NoSplit method}}

\begin{fulllineitems}
\phantomsection\label{\detokenize{api/mastml.data_splitters.NoSplit:mastml.data_splitters.NoSplit.split}}\pysiglinewithargsret{\sphinxbfcode{\sphinxupquote{split}}}{\emph{X}, \emph{y=None}, \emph{groups=None}}{}
Generate indices to split data into training and test set.
\begin{description}
\item[{X}] \leavevmode{[}array-like of shape (n\_samples, n\_features){]}
Training data, where n\_samples is the number of samples
and n\_features is the number of features.

\item[{y}] \leavevmode{[}array-like of shape (n\_samples,){]}
The target variable for supervised learning problems.

\item[{groups}] \leavevmode{[}array-like of shape (n\_samples,), default=None{]}
Group labels for the samples used while splitting the dataset into
train/test set.

\end{description}
\begin{description}
\item[{train}] \leavevmode{[}ndarray{]}
The training set indices for that split.

\item[{test}] \leavevmode{[}ndarray{]}
The testing set indices for that split.

\end{description}

\end{fulllineitems}


\end{fulllineitems}



\subsubsection{SklearnDataSplitter}
\label{\detokenize{api/mastml.data_splitters.SklearnDataSplitter:sklearndatasplitter}}\label{\detokenize{api/mastml.data_splitters.SklearnDataSplitter::doc}}\index{SklearnDataSplitter (class in mastml.data\_splitters)@\spxentry{SklearnDataSplitter}\spxextra{class in mastml.data\_splitters}}

\begin{fulllineitems}
\phantomsection\label{\detokenize{api/mastml.data_splitters.SklearnDataSplitter:mastml.data_splitters.SklearnDataSplitter}}\pysiglinewithargsret{\sphinxbfcode{\sphinxupquote{class }}\sphinxcode{\sphinxupquote{mastml.data\_splitters.}}\sphinxbfcode{\sphinxupquote{SklearnDataSplitter}}}{\emph{splitter}, \emph{**kwargs}}{}
Bases: {\hyperref[\detokenize{api/mastml.data_splitters.BaseSplitter:mastml.data_splitters.BaseSplitter}]{\sphinxcrossref{\sphinxcode{\sphinxupquote{mastml.data\_splitters.BaseSplitter}}}}}

Class to wrap any scikit-learn based data splitter, e.g. KFold
\begin{description}
\item[{Args:}] \leavevmode
splitter (str): string denoting the name of a sklearn.model\_selection object, e.g. ‘KFold’ will draw from sklearn.model\_selection.KFold()

kwargs : key word arguments for the sklearn.model\_selection object, e.g. n\_splits=5 for KFold()

\item[{Methods:}] \leavevmode\begin{description}
\item[{get\_n\_splits: method to calculate the number of splits to perform}] \leavevmode\begin{description}
\item[{Args:}] \leavevmode
None

\item[{Returns:}] \leavevmode
(int), number of train/test splits

\end{description}

\item[{split: method to perform split into train indices and test indices}] \leavevmode\begin{description}
\item[{Args:}] \leavevmode
X: (numpy array), array of X features

\item[{Returns:}] \leavevmode
(numpy array), array of train and test indices

\end{description}

\item[{\_setup\_savedir: method to create a savedir based on the provided model, splitter, selector names and datetime}] \leavevmode\begin{description}
\item[{Args:}] \leavevmode
model: (mastml.models.SklearnModel or other estimator object), an estimator, e.g. KernelRidge

selector: (mastml.feature\_selectors or other selector object), a selector, e.g. EnsembleModelFeatureSelector

savepath: (str), string designating the savepath

\item[{Returns:}] \leavevmode
splitdir: (str), string containing the new subdirectory to save results to

\end{description}

\end{description}

\end{description}
\subsubsection*{Methods Summary}


\begin{savenotes}\sphinxatlongtablestart\begin{longtable}[c]{\X{1}{2}\X{1}{2}}
\hline

\endfirsthead

\multicolumn{2}{c}%
{\makebox[0pt]{\sphinxtablecontinued{\tablename\ \thetable{} -- continued from previous page}}}\\
\hline

\endhead

\hline
\multicolumn{2}{r}{\makebox[0pt][r]{\sphinxtablecontinued{Continued on next page}}}\\
\endfoot

\endlastfoot

{\hyperref[\detokenize{api/mastml.data_splitters.SklearnDataSplitter:mastml.data_splitters.SklearnDataSplitter.get_n_splits}]{\sphinxcrossref{\sphinxcode{\sphinxupquote{get\_n\_splits}}}}}({[}X, y, groups{]})
&
Returns the number of splitting iterations in the cross-validator
\\
\hline
{\hyperref[\detokenize{api/mastml.data_splitters.SklearnDataSplitter:mastml.data_splitters.SklearnDataSplitter.split}]{\sphinxcrossref{\sphinxcode{\sphinxupquote{split}}}}}(X{[}, y, groups{]})
&
Generate indices to split data into training and test set.
\\
\hline
\end{longtable}\sphinxatlongtableend\end{savenotes}
\subsubsection*{Methods Documentation}
\index{get\_n\_splits() (mastml.data\_splitters.SklearnDataSplitter method)@\spxentry{get\_n\_splits()}\spxextra{mastml.data\_splitters.SklearnDataSplitter method}}

\begin{fulllineitems}
\phantomsection\label{\detokenize{api/mastml.data_splitters.SklearnDataSplitter:mastml.data_splitters.SklearnDataSplitter.get_n_splits}}\pysiglinewithargsret{\sphinxbfcode{\sphinxupquote{get\_n\_splits}}}{\emph{X=None}, \emph{y=None}, \emph{groups=None}}{}
Returns the number of splitting iterations in the cross-validator

\end{fulllineitems}

\index{split() (mastml.data\_splitters.SklearnDataSplitter method)@\spxentry{split()}\spxextra{mastml.data\_splitters.SklearnDataSplitter method}}

\begin{fulllineitems}
\phantomsection\label{\detokenize{api/mastml.data_splitters.SklearnDataSplitter:mastml.data_splitters.SklearnDataSplitter.split}}\pysiglinewithargsret{\sphinxbfcode{\sphinxupquote{split}}}{\emph{X}, \emph{y=None}, \emph{groups=None}}{}
Generate indices to split data into training and test set.
\begin{description}
\item[{X}] \leavevmode{[}array-like of shape (n\_samples, n\_features){]}
Training data, where n\_samples is the number of samples
and n\_features is the number of features.

\item[{y}] \leavevmode{[}array-like of shape (n\_samples,){]}
The target variable for supervised learning problems.

\item[{groups}] \leavevmode{[}array-like of shape (n\_samples,), default=None{]}
Group labels for the samples used while splitting the dataset into
train/test set.

\end{description}
\begin{description}
\item[{train}] \leavevmode{[}ndarray{]}
The training set indices for that split.

\item[{test}] \leavevmode{[}ndarray{]}
The testing set indices for that split.

\end{description}

\end{fulllineitems}


\end{fulllineitems}



\subsection{Class Inheritance Diagram}
\label{\detokenize{2_data_splitters:class-inheritance-diagram}}
\sphinxincludegraphics[]{None}


\chapter{Code Documentation: Datasets}
\label{\detokenize{3_datasets:code-documentation-datasets}}\label{\detokenize{3_datasets::doc}}

\section{mastml.datasets Module}
\label{\detokenize{3_datasets:module-mastml.datasets}}\label{\detokenize{3_datasets:mastml-datasets-module}}\index{mastml.datasets (module)@\spxentry{mastml.datasets}\spxextra{module}}
This module provides various methods for importing data into MAST-ML.
\begin{description}
\item[{SklearnDatasets:}] \leavevmode
Enables easy import of model datasets from scikit-learn, such as boston housing data, friedman, etc.

\item[{LocalDatasets:}] \leavevmode
Main method for importing datasets that are stored in an accessible path. Main file format is Excel
spreadsheet (.xls or .xlsx). This method also makes it easy for separately denoting other data features
that are not directly the X or y data, such as features used for grouping, extra features no used in
fitting, or features that denote manually held-out test data

\item[{FigshareDatasets:}] \leavevmode
Method to download data that is stored on Figshare, an open-source data hosting service. This class
can be used to download data, then subsquently the LocalDatasets class can be used to import the data.

\item[{FoundryDatasets:}] \leavevmode
Method to download data this stored on the Materials Data Facility (MDF) Foundry data hosting service.
This class can be used to download data, then subsquently the LocalDatasets class can be used to import
the data.

\item[{MatminerDatasets:}] \leavevmode
Method to download data this stored as part of the matminer machine learning package
(\sphinxurl{https://github.com/hackingmaterials/matminer}). This class can be used to download data, then
subsquently the LocalDatasets class can be used to import the data.

\end{description}


\subsection{Classes}
\label{\detokenize{3_datasets:classes}}

\begin{savenotes}\sphinxatlongtablestart\begin{longtable}[c]{\X{1}{2}\X{1}{2}}
\hline

\endfirsthead

\multicolumn{2}{c}%
{\makebox[0pt]{\sphinxtablecontinued{\tablename\ \thetable{} -- continued from previous page}}}\\
\hline

\endhead

\hline
\multicolumn{2}{r}{\makebox[0pt][r]{\sphinxtablecontinued{Continued on next page}}}\\
\endfoot

\endlastfoot

\sphinxcode{\sphinxupquote{Figshare}}({[}token, private{]})
&
A Python interface to Figshare
\\
\hline
{\hyperref[\detokenize{api/mastml.datasets.FigshareDatasets:mastml.datasets.FigshareDatasets}]{\sphinxcrossref{\sphinxcode{\sphinxupquote{FigshareDatasets}}}}}()
&
Class to download datasets hosted on Figshare.
\\
\hline
\sphinxcode{\sphinxupquote{Forge}}({[}index, local\_ep, anonymous, …{]})
&
Forge fetches metadata and files from the Materials Data Facility.
\\
\hline
{\hyperref[\detokenize{api/mastml.datasets.FoundryDatasets:mastml.datasets.FoundryDatasets}]{\sphinxcrossref{\sphinxcode{\sphinxupquote{FoundryDatasets}}}}}(no\_local\_server, anonymous, test)
&
Class to download datasets hosted on Materials Data Facility
\\
\hline
{\hyperref[\detokenize{api/mastml.datasets.LocalDatasets:mastml.datasets.LocalDatasets}]{\sphinxcrossref{\sphinxcode{\sphinxupquote{LocalDatasets}}}}}(file\_path{[}, feature\_names, …{]})
&
Class to handle import and organization of a dataset stored locally.
\\
\hline
{\hyperref[\detokenize{api/mastml.datasets.MatminerDatasets:mastml.datasets.MatminerDatasets}]{\sphinxcrossref{\sphinxcode{\sphinxupquote{MatminerDatasets}}}}}()
&
Class to download datasets hosted from the Matminer package’s Figshare page.
\\
\hline
{\hyperref[\detokenize{api/mastml.datasets.SklearnDatasets:mastml.datasets.SklearnDatasets}]{\sphinxcrossref{\sphinxcode{\sphinxupquote{SklearnDatasets}}}}}({[}return\_X\_y, as\_frame{]})
&
Class wrapping the sklearn.datasets funcionality for easy import of toy datasets from sklearn.
\\
\hline
\end{longtable}\sphinxatlongtableend\end{savenotes}


\subsubsection{FigshareDatasets}
\label{\detokenize{api/mastml.datasets.FigshareDatasets:figsharedatasets}}\label{\detokenize{api/mastml.datasets.FigshareDatasets::doc}}\index{FigshareDatasets (class in mastml.datasets)@\spxentry{FigshareDatasets}\spxextra{class in mastml.datasets}}

\begin{fulllineitems}
\phantomsection\label{\detokenize{api/mastml.datasets.FigshareDatasets:mastml.datasets.FigshareDatasets}}\pysigline{\sphinxbfcode{\sphinxupquote{class }}\sphinxcode{\sphinxupquote{mastml.datasets.}}\sphinxbfcode{\sphinxupquote{FigshareDatasets}}}
Bases: \sphinxcode{\sphinxupquote{object}}

Class to download datasets hosted on Figshare. To install: git clone \sphinxurl{https://github.com/cognoma/figshare.git}
\begin{description}
\item[{Args:}] \leavevmode
None

\item[{Methods:}] \leavevmode\begin{description}
\item[{download\_data: downloads specified data from Figshare and saves to current directory}] \leavevmode\begin{description}
\item[{Args:}] \leavevmode
article\_id: (int), the number denoting the Figshare article ID. Can be obtained from the URL to the Figshare dataset

savepath: (str), string denoting the savepath of the MAST-ML run

\item[{Returns:}] \leavevmode
None

\end{description}

\end{description}

\end{description}
\subsubsection*{Methods Summary}


\begin{savenotes}\sphinxatlongtablestart\begin{longtable}[c]{\X{1}{2}\X{1}{2}}
\hline

\endfirsthead

\multicolumn{2}{c}%
{\makebox[0pt]{\sphinxtablecontinued{\tablename\ \thetable{} -- continued from previous page}}}\\
\hline

\endhead

\hline
\multicolumn{2}{r}{\makebox[0pt][r]{\sphinxtablecontinued{Continued on next page}}}\\
\endfoot

\endlastfoot

{\hyperref[\detokenize{api/mastml.datasets.FigshareDatasets:mastml.datasets.FigshareDatasets.download_data}]{\sphinxcrossref{\sphinxcode{\sphinxupquote{download\_data}}}}}(article\_id{[}, savepath{]})
&

\\
\hline
\end{longtable}\sphinxatlongtableend\end{savenotes}
\subsubsection*{Methods Documentation}
\index{download\_data() (mastml.datasets.FigshareDatasets method)@\spxentry{download\_data()}\spxextra{mastml.datasets.FigshareDatasets method}}

\begin{fulllineitems}
\phantomsection\label{\detokenize{api/mastml.datasets.FigshareDatasets:mastml.datasets.FigshareDatasets.download_data}}\pysiglinewithargsret{\sphinxbfcode{\sphinxupquote{download\_data}}}{\emph{article\_id}, \emph{savepath=None}}{}
\end{fulllineitems}


\end{fulllineitems}



\subsubsection{FoundryDatasets}
\label{\detokenize{api/mastml.datasets.FoundryDatasets:foundrydatasets}}\label{\detokenize{api/mastml.datasets.FoundryDatasets::doc}}\index{FoundryDatasets (class in mastml.datasets)@\spxentry{FoundryDatasets}\spxextra{class in mastml.datasets}}

\begin{fulllineitems}
\phantomsection\label{\detokenize{api/mastml.datasets.FoundryDatasets:mastml.datasets.FoundryDatasets}}\pysiglinewithargsret{\sphinxbfcode{\sphinxupquote{class }}\sphinxcode{\sphinxupquote{mastml.datasets.}}\sphinxbfcode{\sphinxupquote{FoundryDatasets}}}{\emph{no\_local\_server}, \emph{anonymous}, \emph{test}}{}
Bases: \sphinxcode{\sphinxupquote{object}}

Class to download datasets hosted on Materials Data Facility
\begin{description}
\item[{Args:}] \leavevmode
no\_local\_server: (bool), whether or not the server is local. Set to True if running on e.g. Google Colab

anonymous: (bool), whether to use your MDF user or be anonymous. Some functionality may be disabled if True

test: (bool), whether to be in test mode. Some functionality may be disabled if True

\item[{Methods:}] \leavevmode\begin{description}
\item[{download\_data: downloads specified data from MDF and saves to current directory}] \leavevmode\begin{description}
\item[{Args:}] \leavevmode
name: (str), name of the dataset to download

doi: (str), digital object identifier of the dataset to download

download: (bool), whether or not to download the full dataset

\item[{Returns:}] \leavevmode
None

\end{description}

\end{description}

\end{description}
\subsubsection*{Methods Summary}


\begin{savenotes}\sphinxatlongtablestart\begin{longtable}[c]{\X{1}{2}\X{1}{2}}
\hline

\endfirsthead

\multicolumn{2}{c}%
{\makebox[0pt]{\sphinxtablecontinued{\tablename\ \thetable{} -- continued from previous page}}}\\
\hline

\endhead

\hline
\multicolumn{2}{r}{\makebox[0pt][r]{\sphinxtablecontinued{Continued on next page}}}\\
\endfoot

\endlastfoot

{\hyperref[\detokenize{api/mastml.datasets.FoundryDatasets:mastml.datasets.FoundryDatasets.download_data}]{\sphinxcrossref{\sphinxcode{\sphinxupquote{download\_data}}}}}({[}name, doi, download{]})
&

\\
\hline
\end{longtable}\sphinxatlongtableend\end{savenotes}
\subsubsection*{Methods Documentation}
\index{download\_data() (mastml.datasets.FoundryDatasets method)@\spxentry{download\_data()}\spxextra{mastml.datasets.FoundryDatasets method}}

\begin{fulllineitems}
\phantomsection\label{\detokenize{api/mastml.datasets.FoundryDatasets:mastml.datasets.FoundryDatasets.download_data}}\pysiglinewithargsret{\sphinxbfcode{\sphinxupquote{download\_data}}}{\emph{name=None}, \emph{doi=None}, \emph{download=False}}{}
\end{fulllineitems}


\end{fulllineitems}



\subsubsection{LocalDatasets}
\label{\detokenize{api/mastml.datasets.LocalDatasets:localdatasets}}\label{\detokenize{api/mastml.datasets.LocalDatasets::doc}}\index{LocalDatasets (class in mastml.datasets)@\spxentry{LocalDatasets}\spxextra{class in mastml.datasets}}

\begin{fulllineitems}
\phantomsection\label{\detokenize{api/mastml.datasets.LocalDatasets:mastml.datasets.LocalDatasets}}\pysiglinewithargsret{\sphinxbfcode{\sphinxupquote{class }}\sphinxcode{\sphinxupquote{mastml.datasets.}}\sphinxbfcode{\sphinxupquote{LocalDatasets}}}{\emph{file\_path}, \emph{feature\_names=None}, \emph{target=None}, \emph{extra\_columns=None}, \emph{group\_column=None}, \emph{testdata\_columns=None}, \emph{as\_frame=False}}{}
Bases: \sphinxcode{\sphinxupquote{object}}

Class to handle import and organization of a dataset stored locally.
\begin{description}
\item[{Args:}] \leavevmode
file\_path: (str), path to the data file to import

feature\_names: (list), list of strings containing the X feature names

target: (str), string denoting the y data (target) name

extra\_columns: (list), list of strings containing additional column names that are not features or target

group\_column: (str), string denoting the name of an input column to be used to group data

testdata\_columns: (list), list of strings containing column names denoting sets of left-out data. Entries should be marked with a 0 (not left out) or 1 (left out)

as\_frame: (bool), whether to return data as pandas dataframe (otherwise will be numpy array)

\item[{Methods:}] \leavevmode\begin{description}
\item[{\_import: imports the data. Should be either .csv or .xlsx format}] \leavevmode\begin{description}
\item[{Args:}] \leavevmode
None

\item[{Returns:}] \leavevmode
df: (pd.DataFrame), pandas dataframe of full dataset

\end{description}

\item[{\_get\_features: Method to assess which columns below to target, feature\_names}] \leavevmode\begin{description}
\item[{Args:}] \leavevmode
df: (pd.DataFrame), pandas dataframe of full dataset

\item[{Returns:}] \leavevmode
None

\end{description}

\item[{load\_data: Method to import the data and ascertain which columns are features, target and extra based on provided input.}] \leavevmode\begin{description}
\item[{Args:}] \leavevmode
None

\item[{Returns:}] \leavevmode
data\_dict: (dict), dictionary containing dataframes of X, y, groups, X\_extra, X\_testdata

\end{description}

\end{description}

\end{description}
\subsubsection*{Methods Summary}


\begin{savenotes}\sphinxatlongtablestart\begin{longtable}[c]{\X{1}{2}\X{1}{2}}
\hline

\endfirsthead

\multicolumn{2}{c}%
{\makebox[0pt]{\sphinxtablecontinued{\tablename\ \thetable{} -- continued from previous page}}}\\
\hline

\endhead

\hline
\multicolumn{2}{r}{\makebox[0pt][r]{\sphinxtablecontinued{Continued on next page}}}\\
\endfoot

\endlastfoot

{\hyperref[\detokenize{api/mastml.datasets.LocalDatasets:mastml.datasets.LocalDatasets.load_data}]{\sphinxcrossref{\sphinxcode{\sphinxupquote{load\_data}}}}}()
&

\\
\hline
\end{longtable}\sphinxatlongtableend\end{savenotes}
\subsubsection*{Methods Documentation}
\index{load\_data() (mastml.datasets.LocalDatasets method)@\spxentry{load\_data()}\spxextra{mastml.datasets.LocalDatasets method}}

\begin{fulllineitems}
\phantomsection\label{\detokenize{api/mastml.datasets.LocalDatasets:mastml.datasets.LocalDatasets.load_data}}\pysiglinewithargsret{\sphinxbfcode{\sphinxupquote{load\_data}}}{}{}
\end{fulllineitems}


\end{fulllineitems}



\subsubsection{MatminerDatasets}
\label{\detokenize{api/mastml.datasets.MatminerDatasets:matminerdatasets}}\label{\detokenize{api/mastml.datasets.MatminerDatasets::doc}}\index{MatminerDatasets (class in mastml.datasets)@\spxentry{MatminerDatasets}\spxextra{class in mastml.datasets}}

\begin{fulllineitems}
\phantomsection\label{\detokenize{api/mastml.datasets.MatminerDatasets:mastml.datasets.MatminerDatasets}}\pysigline{\sphinxbfcode{\sphinxupquote{class }}\sphinxcode{\sphinxupquote{mastml.datasets.}}\sphinxbfcode{\sphinxupquote{MatminerDatasets}}}
Bases: \sphinxcode{\sphinxupquote{object}}

Class to download datasets hosted from the Matminer package’s Figshare page. A summary of available datasets
can be found at: \sphinxurl{https://hackingmaterials.lbl.gov/matminer/dataset\_summary.html}
\begin{description}
\item[{Args:}] \leavevmode
None

\item[{Methods:}] \leavevmode\begin{description}
\item[{download\_data: downloads specified data from Matminer/Figshare and saves to current directory}] \leavevmode\begin{description}
\item[{Args:}] \leavevmode
name: (str), name of the dataset to download. For compatible names, call get\_available\_datasets

save\_data: (bool), whether to save the downloaded data to the current working directory

\item[{Returns:}] \leavevmode
df: (dataframe), dataframe of downloaded data

\end{description}

\item[{get\_available\_datasets: returns information on the available dataset names and details one can downlaod}] \leavevmode\begin{description}
\item[{Args:}] \leavevmode
None.

\item[{Returns:}] \leavevmode
None.

\end{description}

\end{description}

\end{description}
\subsubsection*{Methods Summary}


\begin{savenotes}\sphinxatlongtablestart\begin{longtable}[c]{\X{1}{2}\X{1}{2}}
\hline

\endfirsthead

\multicolumn{2}{c}%
{\makebox[0pt]{\sphinxtablecontinued{\tablename\ \thetable{} -- continued from previous page}}}\\
\hline

\endhead

\hline
\multicolumn{2}{r}{\makebox[0pt][r]{\sphinxtablecontinued{Continued on next page}}}\\
\endfoot

\endlastfoot

{\hyperref[\detokenize{api/mastml.datasets.MatminerDatasets:mastml.datasets.MatminerDatasets.download_data}]{\sphinxcrossref{\sphinxcode{\sphinxupquote{download\_data}}}}}(name{[}, save\_data{]})
&

\\
\hline
{\hyperref[\detokenize{api/mastml.datasets.MatminerDatasets:mastml.datasets.MatminerDatasets.get_available_datasets}]{\sphinxcrossref{\sphinxcode{\sphinxupquote{get\_available\_datasets}}}}}()
&

\\
\hline
\end{longtable}\sphinxatlongtableend\end{savenotes}
\subsubsection*{Methods Documentation}
\index{download\_data() (mastml.datasets.MatminerDatasets method)@\spxentry{download\_data()}\spxextra{mastml.datasets.MatminerDatasets method}}

\begin{fulllineitems}
\phantomsection\label{\detokenize{api/mastml.datasets.MatminerDatasets:mastml.datasets.MatminerDatasets.download_data}}\pysiglinewithargsret{\sphinxbfcode{\sphinxupquote{download\_data}}}{\emph{name}, \emph{save\_data=True}}{}
\end{fulllineitems}

\index{get\_available\_datasets() (mastml.datasets.MatminerDatasets method)@\spxentry{get\_available\_datasets()}\spxextra{mastml.datasets.MatminerDatasets method}}

\begin{fulllineitems}
\phantomsection\label{\detokenize{api/mastml.datasets.MatminerDatasets:mastml.datasets.MatminerDatasets.get_available_datasets}}\pysiglinewithargsret{\sphinxbfcode{\sphinxupquote{get\_available\_datasets}}}{}{}
\end{fulllineitems}


\end{fulllineitems}



\subsubsection{SklearnDatasets}
\label{\detokenize{api/mastml.datasets.SklearnDatasets:sklearndatasets}}\label{\detokenize{api/mastml.datasets.SklearnDatasets::doc}}\index{SklearnDatasets (class in mastml.datasets)@\spxentry{SklearnDatasets}\spxextra{class in mastml.datasets}}

\begin{fulllineitems}
\phantomsection\label{\detokenize{api/mastml.datasets.SklearnDatasets:mastml.datasets.SklearnDatasets}}\pysiglinewithargsret{\sphinxbfcode{\sphinxupquote{class }}\sphinxcode{\sphinxupquote{mastml.datasets.}}\sphinxbfcode{\sphinxupquote{SklearnDatasets}}}{\emph{return\_X\_y=True}, \emph{as\_frame=False}}{}
Bases: \sphinxcode{\sphinxupquote{object}}

Class wrapping the sklearn.datasets funcionality for easy import of toy datasets from sklearn. Added some changes
to make all datasets operate more consistently, e.g. boston housing data
\begin{description}
\item[{Args:}] \leavevmode
return\_X\_y: (bool), whether to return X, y data as (X, y) tuple (should be true for easiest use in MASTML)

as\_frame: (bool), whether to return X, y data as pandas dataframe objects

n\_class: (int), number of classes (only applies to load\_digits method)

\item[{Methods:}] \leavevmode
load\_boston: Loads the Boston housing data (regression)

load\_iris: Loads the flower iris data (classification)

load\_diabetes: Loads the diabetes data set (regression)

load\_digits: Loads the MNIST digits data set (classification)

load\_linnerud: Loads the linnerud data set (regression)

load\_wine: Loads the wine data set (classification)

load\_breast\_cancer: Loads the breast cancer data set (classification)

load\_friedman: Loads the Friedman data set (regression)

\end{description}
\subsubsection*{Methods Summary}


\begin{savenotes}\sphinxatlongtablestart\begin{longtable}[c]{\X{1}{2}\X{1}{2}}
\hline

\endfirsthead

\multicolumn{2}{c}%
{\makebox[0pt]{\sphinxtablecontinued{\tablename\ \thetable{} -- continued from previous page}}}\\
\hline

\endhead

\hline
\multicolumn{2}{r}{\makebox[0pt][r]{\sphinxtablecontinued{Continued on next page}}}\\
\endfoot

\endlastfoot

{\hyperref[\detokenize{api/mastml.datasets.SklearnDatasets:mastml.datasets.SklearnDatasets.load_boston}]{\sphinxcrossref{\sphinxcode{\sphinxupquote{load\_boston}}}}}()
&

\\
\hline
{\hyperref[\detokenize{api/mastml.datasets.SklearnDatasets:mastml.datasets.SklearnDatasets.load_breast_cancer}]{\sphinxcrossref{\sphinxcode{\sphinxupquote{load\_breast\_cancer}}}}}()
&

\\
\hline
{\hyperref[\detokenize{api/mastml.datasets.SklearnDatasets:mastml.datasets.SklearnDatasets.load_diabetes}]{\sphinxcrossref{\sphinxcode{\sphinxupquote{load\_diabetes}}}}}()
&

\\
\hline
{\hyperref[\detokenize{api/mastml.datasets.SklearnDatasets:mastml.datasets.SklearnDatasets.load_digits}]{\sphinxcrossref{\sphinxcode{\sphinxupquote{load\_digits}}}}}({[}n\_class{]})
&

\\
\hline
{\hyperref[\detokenize{api/mastml.datasets.SklearnDatasets:mastml.datasets.SklearnDatasets.load_friedman}]{\sphinxcrossref{\sphinxcode{\sphinxupquote{load\_friedman}}}}}({[}n\_samples, n\_features, noise{]})
&

\\
\hline
{\hyperref[\detokenize{api/mastml.datasets.SklearnDatasets:mastml.datasets.SklearnDatasets.load_iris}]{\sphinxcrossref{\sphinxcode{\sphinxupquote{load\_iris}}}}}()
&

\\
\hline
{\hyperref[\detokenize{api/mastml.datasets.SklearnDatasets:mastml.datasets.SklearnDatasets.load_linnerud}]{\sphinxcrossref{\sphinxcode{\sphinxupquote{load\_linnerud}}}}}()
&

\\
\hline
{\hyperref[\detokenize{api/mastml.datasets.SklearnDatasets:mastml.datasets.SklearnDatasets.load_wine}]{\sphinxcrossref{\sphinxcode{\sphinxupquote{load\_wine}}}}}()
&

\\
\hline
\end{longtable}\sphinxatlongtableend\end{savenotes}
\subsubsection*{Methods Documentation}
\index{load\_boston() (mastml.datasets.SklearnDatasets method)@\spxentry{load\_boston()}\spxextra{mastml.datasets.SklearnDatasets method}}

\begin{fulllineitems}
\phantomsection\label{\detokenize{api/mastml.datasets.SklearnDatasets:mastml.datasets.SklearnDatasets.load_boston}}\pysiglinewithargsret{\sphinxbfcode{\sphinxupquote{load\_boston}}}{}{}
\end{fulllineitems}

\index{load\_breast\_cancer() (mastml.datasets.SklearnDatasets method)@\spxentry{load\_breast\_cancer()}\spxextra{mastml.datasets.SklearnDatasets method}}

\begin{fulllineitems}
\phantomsection\label{\detokenize{api/mastml.datasets.SklearnDatasets:mastml.datasets.SklearnDatasets.load_breast_cancer}}\pysiglinewithargsret{\sphinxbfcode{\sphinxupquote{load\_breast\_cancer}}}{}{}
\end{fulllineitems}

\index{load\_diabetes() (mastml.datasets.SklearnDatasets method)@\spxentry{load\_diabetes()}\spxextra{mastml.datasets.SklearnDatasets method}}

\begin{fulllineitems}
\phantomsection\label{\detokenize{api/mastml.datasets.SklearnDatasets:mastml.datasets.SklearnDatasets.load_diabetes}}\pysiglinewithargsret{\sphinxbfcode{\sphinxupquote{load\_diabetes}}}{}{}
\end{fulllineitems}

\index{load\_digits() (mastml.datasets.SklearnDatasets method)@\spxentry{load\_digits()}\spxextra{mastml.datasets.SklearnDatasets method}}

\begin{fulllineitems}
\phantomsection\label{\detokenize{api/mastml.datasets.SklearnDatasets:mastml.datasets.SklearnDatasets.load_digits}}\pysiglinewithargsret{\sphinxbfcode{\sphinxupquote{load\_digits}}}{\emph{n\_class=10}}{}
\end{fulllineitems}

\index{load\_friedman() (mastml.datasets.SklearnDatasets method)@\spxentry{load\_friedman()}\spxextra{mastml.datasets.SklearnDatasets method}}

\begin{fulllineitems}
\phantomsection\label{\detokenize{api/mastml.datasets.SklearnDatasets:mastml.datasets.SklearnDatasets.load_friedman}}\pysiglinewithargsret{\sphinxbfcode{\sphinxupquote{load\_friedman}}}{\emph{n\_samples=100}, \emph{n\_features=10}, \emph{noise=0.0}}{}
\end{fulllineitems}

\index{load\_iris() (mastml.datasets.SklearnDatasets method)@\spxentry{load\_iris()}\spxextra{mastml.datasets.SklearnDatasets method}}

\begin{fulllineitems}
\phantomsection\label{\detokenize{api/mastml.datasets.SklearnDatasets:mastml.datasets.SklearnDatasets.load_iris}}\pysiglinewithargsret{\sphinxbfcode{\sphinxupquote{load\_iris}}}{}{}
\end{fulllineitems}

\index{load\_linnerud() (mastml.datasets.SklearnDatasets method)@\spxentry{load\_linnerud()}\spxextra{mastml.datasets.SklearnDatasets method}}

\begin{fulllineitems}
\phantomsection\label{\detokenize{api/mastml.datasets.SklearnDatasets:mastml.datasets.SklearnDatasets.load_linnerud}}\pysiglinewithargsret{\sphinxbfcode{\sphinxupquote{load\_linnerud}}}{}{}
\end{fulllineitems}

\index{load\_wine() (mastml.datasets.SklearnDatasets method)@\spxentry{load\_wine()}\spxextra{mastml.datasets.SklearnDatasets method}}

\begin{fulllineitems}
\phantomsection\label{\detokenize{api/mastml.datasets.SklearnDatasets:mastml.datasets.SklearnDatasets.load_wine}}\pysiglinewithargsret{\sphinxbfcode{\sphinxupquote{load\_wine}}}{}{}
\end{fulllineitems}


\end{fulllineitems}



\subsection{Class Inheritance Diagram}
\label{\detokenize{3_datasets:class-inheritance-diagram}}
\sphinxincludegraphics[]{None}


\chapter{Code Documentation: Error Analysis}
\label{\detokenize{4_error_analysis:code-documentation-error-analysis}}\label{\detokenize{4_error_analysis::doc}}

\section{mastml.error\_analysis Module}
\label{\detokenize{4_error_analysis:module-mastml.error_analysis}}\label{\detokenize{4_error_analysis:mastml-error-analysis-module}}\index{mastml.error\_analysis (module)@\spxentry{mastml.error\_analysis}\spxextra{module}}
This module contains classes for quantifying the predicted model errors (uncertainty quantification), and preparing
provided residual (true errors) predicted model error data for plotting (e.g. residual vs. error plots), or for
recalibration of model errors using the method of Palmer et al.
\begin{description}
\item[{ErrorUtils:}] \leavevmode
Collection of functions to conduct error analysis on certain types of models (uncertainty quantification), and prepare
residual and model error data for plotting, as well as recalibrate model errors with various methods

\item[{CorrectionFactors}] \leavevmode
Class for performing recalibration of model errors (uncertainty quantification) based on the method from the
work of Palmer et al.

\end{description}


\subsection{Classes}
\label{\detokenize{4_error_analysis:classes}}

\begin{savenotes}\sphinxatlongtablestart\begin{longtable}[c]{\X{1}{2}\X{1}{2}}
\hline

\endfirsthead

\multicolumn{2}{c}%
{\makebox[0pt]{\sphinxtablecontinued{\tablename\ \thetable{} -- continued from previous page}}}\\
\hline

\endhead

\hline
\multicolumn{2}{r}{\makebox[0pt][r]{\sphinxtablecontinued{Continued on next page}}}\\
\endfoot

\endlastfoot

{\hyperref[\detokenize{api/mastml.error_analysis.CorrectionFactors:mastml.error_analysis.CorrectionFactors}]{\sphinxcrossref{\sphinxcode{\sphinxupquote{CorrectionFactors}}}}}(residuals, model\_errors)
&
Class for performing recalibration of model errors (uncertainty quantification) based on the method from the work of Palmer et al.
\\
\hline
{\hyperref[\detokenize{api/mastml.error_analysis.ErrorUtils:mastml.error_analysis.ErrorUtils}]{\sphinxcrossref{\sphinxcode{\sphinxupquote{ErrorUtils}}}}}
&
Collection of functions to conduct error analysis on certain types of models (uncertainty quantification), and prepare residual and model error data for plotting, as well as recalibrate model errors with various methods
\\
\hline
\end{longtable}\sphinxatlongtableend\end{savenotes}


\subsubsection{CorrectionFactors}
\label{\detokenize{api/mastml.error_analysis.CorrectionFactors:correctionfactors}}\label{\detokenize{api/mastml.error_analysis.CorrectionFactors::doc}}\index{CorrectionFactors (class in mastml.error\_analysis)@\spxentry{CorrectionFactors}\spxextra{class in mastml.error\_analysis}}

\begin{fulllineitems}
\phantomsection\label{\detokenize{api/mastml.error_analysis.CorrectionFactors:mastml.error_analysis.CorrectionFactors}}\pysiglinewithargsret{\sphinxbfcode{\sphinxupquote{class }}\sphinxcode{\sphinxupquote{mastml.error\_analysis.}}\sphinxbfcode{\sphinxupquote{CorrectionFactors}}}{\emph{residuals}, \emph{model\_errors}}{}
Bases: \sphinxcode{\sphinxupquote{object}}

Class for performing recalibration of model errors (uncertainty quantification) based on the method from the
work of Palmer et al.
\begin{description}
\item[{Args:}] \leavevmode
residuals: (pd.Series), series containing the true model errors (residuals)

model\_errors: (pd.Series), series containing the predicted model errors

\item[{Methods:}] \leavevmode\begin{description}
\item[{nll: Method to perform optimization of normalized model error distribution using the negative log likelihood function}] \leavevmode\begin{description}
\item[{Args:}] \leavevmode
None

\item[{Returns:}] \leavevmode
a: (float), the slope of the recalibration linear fit

b: (float), the intercept of the recalibration linear fit

\end{description}

\item[{\_nll\_opt: Method for evaluating the negative log likelihood function for set of residuals and model errors}] \leavevmode\begin{description}
\item[{Args:}] \leavevmode
x: (np.array), array providing the initial guess of (a, b)

\item[{Returns:}] \leavevmode
\_ : (float), the recalibrated error value

\end{description}

\end{description}

\end{description}
\subsubsection*{Methods Summary}


\begin{savenotes}\sphinxatlongtablestart\begin{longtable}[c]{\X{1}{2}\X{1}{2}}
\hline

\endfirsthead

\multicolumn{2}{c}%
{\makebox[0pt]{\sphinxtablecontinued{\tablename\ \thetable{} -- continued from previous page}}}\\
\hline

\endhead

\hline
\multicolumn{2}{r}{\makebox[0pt][r]{\sphinxtablecontinued{Continued on next page}}}\\
\endfoot

\endlastfoot

{\hyperref[\detokenize{api/mastml.error_analysis.CorrectionFactors:mastml.error_analysis.CorrectionFactors.nll}]{\sphinxcrossref{\sphinxcode{\sphinxupquote{nll}}}}}()
&

\\
\hline
\end{longtable}\sphinxatlongtableend\end{savenotes}
\subsubsection*{Methods Documentation}
\index{nll() (mastml.error\_analysis.CorrectionFactors method)@\spxentry{nll()}\spxextra{mastml.error\_analysis.CorrectionFactors method}}

\begin{fulllineitems}
\phantomsection\label{\detokenize{api/mastml.error_analysis.CorrectionFactors:mastml.error_analysis.CorrectionFactors.nll}}\pysiglinewithargsret{\sphinxbfcode{\sphinxupquote{nll}}}{}{}
\end{fulllineitems}


\end{fulllineitems}



\subsubsection{ErrorUtils}
\label{\detokenize{api/mastml.error_analysis.ErrorUtils:errorutils}}\label{\detokenize{api/mastml.error_analysis.ErrorUtils::doc}}\index{ErrorUtils (class in mastml.error\_analysis)@\spxentry{ErrorUtils}\spxextra{class in mastml.error\_analysis}}

\begin{fulllineitems}
\phantomsection\label{\detokenize{api/mastml.error_analysis.ErrorUtils:mastml.error_analysis.ErrorUtils}}\pysigline{\sphinxbfcode{\sphinxupquote{class }}\sphinxcode{\sphinxupquote{mastml.error\_analysis.}}\sphinxbfcode{\sphinxupquote{ErrorUtils}}}
Bases: \sphinxcode{\sphinxupquote{object}}

Collection of functions to conduct error analysis on certain types of models (uncertainty quantification), and prepare
residual and model error data for plotting, as well as recalibrate model errors with various methods
\begin{description}
\item[{Args:}] \leavevmode
None

\item[{Methods:}] \leavevmode\begin{description}
\item[{\_collect\_error\_data: method to collect all residuals, model errors, and dataset standard deviation over many data splits}] \leavevmode\begin{description}
\item[{Args:}] \leavevmode
savepath: (str), string denoting the path to save output to

data\_type: (str), string denoting the data type analyzed, e.g. train, test, leftout

\item[{Returns:}] \leavevmode
model\_errors: (pd.Series), series containing the predicted model errors

residuals: (pd.Series), series containing the true model errors (residuals)

dataset\_stdev: (float), standard deviation of the data set

\end{description}

\item[{\_recalibrate\_errors: method to recalibrate the model errors using negative log likelihood function from work of Palmer et al.}] \leavevmode\begin{description}
\item[{Args:}] \leavevmode
model\_errors: (pd.Series), series containing the predicted (uncalibrated) model errors

residuals: (pd.Series), series containing the true model errors (residuals)

\item[{Returns:}] \leavevmode
model\_errors: (pd.Series), series containing the predicted (calibrated) model errors

a: (float), the slope of the recalibration linear fit

b: (float), the intercept of the recalibration linear fit

\end{description}

\item[{\_parse\_error\_data: method to prepare the provided residuals and model errors for plotting the binned RvE (residual vs error) plots}] \leavevmode\begin{description}
\item[{Args:}] \leavevmode
model\_errors: (pd.Series), series containing the predicted model errors

residuals: (pd.Series), series containing the true model errors (residuals)

dataset\_stdev: (float), standard deviation of the data set

number\_of\_bins: (int), the number of bins to digitize the data into for making the RvE (residual vs. error) plot

\item[{Returns:}] \leavevmode
bin\_values: (np.array), the x-axis of the RvE plot: reduced model error values digitized into bins

rms\_residual\_values: (np.array), the y-axis of the RvE plot: the RMS of the residual values digitized into bins

num\_values\_per\_bin: (np.array), the number of data samples in each bin

number\_of\_bins: (int), the number of bins to put the model error and residual data into.

\end{description}

\item[{\_get\_model\_errors: method for generating the model error values using either the standard deviation of weak learners or jackknife-after-bootstrap method of Wager et al.}] \leavevmode\begin{description}
\item[{Args:}] \leavevmode
model: (mastml.models object), a MAST-ML model, e.g. SklearnModel or EnsembleModel

X: (pd.DataFrame), dataframe of the X feature matrix

X\_train: (pd.DataFrame), dataframe of the X training data feature matrix

X\_test: (pd.DataFrame), dataframe of the X test data feature matrix

error\_method: (str), string denoting the UQ error method to use. Viable options are ‘stdev\_weak\_learners’ and ‘jackknife\_after\_bootstrap’

remove\_outlier\_learners: (bool), whether specific weak learners that are found to deviate from 3 sigma of the average prediction for a given data point are removed (Default False)

\item[{Returns:}] \leavevmode
model\_errors: (pd.Series), series containing the predicted model errors

num\_removed\_learners: (list), list of number of removed weak learners for each data point

\end{description}

\item[{\_remove\_outlier\_preds: method to flag and remove outlier weak learner predictions}] \leavevmode\begin{description}
\item[{Args:}] \leavevmode
preds: (list), list of predicted values of a given data point from an ensemble of weak learners

\item[{Returns:}] \leavevmode
preds\_cleaned: (list), ammended list of predicted values of a given data point from an ensemble of weak learners, with predictions from outlier learners removed

num\_outliers: (int), the number of removed weak learners for the data point evaluated

\end{description}

\end{description}

\end{description}

\end{fulllineitems}



\subsection{Class Inheritance Diagram}
\label{\detokenize{4_error_analysis:class-inheritance-diagram}}
\sphinxincludegraphics[]{None}


\chapter{Code Documentation: Feature Generators}
\label{\detokenize{5_feature_generators:code-documentation-feature-generators}}\label{\detokenize{5_feature_generators::doc}}

\section{mastml.feature\_generators Module}
\label{\detokenize{5_feature_generators:module-mastml.feature_generators}}\label{\detokenize{5_feature_generators:mastml-feature-generators-module}}\index{mastml.feature\_generators (module)@\spxentry{mastml.feature\_generators}\spxextra{module}}
This module contains a collection of classes for generating input features to fit machine learning models to.
\begin{description}
\item[{BaseGenerator:}] \leavevmode
Base class to provide MAST-ML type functionality to all feature generators. All other feature generator classes
should inherit from this base class

\item[{ElementalFeatureGenerator:}] \leavevmode
Class written for MAST-ML to generate features for material compositions based on properties of the elements
comprising the composition. A number of mathematically derived variants are included, like arithmetic average,
composition-weighted average, range, max, and min. This generator also supports sublattice-based generation, where
the elemental features can be averaged for each sublattice as opposed to just the total composition together. To
use the sublattice feature of this generator, composition strings must include square brackets to separate the sublattices,
e.g. the perovskite material La0.75Sr0.25MnO3 would be written as {[}La0.75Sr0.25{]}{[}Mn{]}{[}O3{]}

\item[{PolynomialFeatureGenerator:}] \leavevmode
Class used to construct new features based on a polynomial expansion of existing features. The degree of the
polynomial is given as input. For example, for two features x1 and x2, the quadratic features x1\textasciicircum{}2, x2\textasciicircum{}2 and x1*x2
would be generated if the degree is set to 2.

\item[{OneHotGroupGenerator:}] \leavevmode
Class used to create a set of one-hot encoded features based on a single feature containing assorted categories.
For example, if a feature contains strings denoting each data point as belonging to one of three groups such as
“metal”, “semiconductor”, “insulator”, then the generated one-hot features are three feature columns containing a 1 or
0 to denote which group each data point is in

\item[{OneHotElementEncoder:}] \leavevmode
Class used to create a set of one-hot encoded features based on elements present in a supplied chemical composition string.
For example, if the data set contains alloys of materials with chemical formulas such as “GaAs”, “InAs”, “InP”, etc.,
then the generated one-hot features are four feature columns containing a 1 or 0 to denote whether a particular data
point contains each of the unique elements, in this case Ga, As, In, or P.

\item[{MaterialsProjectFeatureGenerator:}] \leavevmode
Class used to search the Materials Project database for computed material property information for the
supplied composition. This only works if the material composition matches an entry present in the Materials Project.
Will return material properties like formation energy, volume, electronic bandgap, elastic constants, etc.

\item[{MatminerFeatureGenerator:}] \leavevmode
Class used to combine various composition and structure-based feature generation routines in the matminer package
into MAST-ML. The use of structure-based features will require pymatgen structure objects in the input
dataframe, while composition-based features require only a composition string. See the class documentation
for more information on the different types of feature generation this class supports.

\item[{DataframeUtilities:}] \leavevmode
Collection of helper routines for various common dataframe operations, like concatentation, merging, etc.

\end{description}


\subsection{Classes}
\label{\detokenize{5_feature_generators:classes}}

\begin{savenotes}\sphinxatlongtablestart\begin{longtable}[c]{\X{1}{2}\X{1}{2}}
\hline

\endfirsthead

\multicolumn{2}{c}%
{\makebox[0pt]{\sphinxtablecontinued{\tablename\ \thetable{} -- continued from previous page}}}\\
\hline

\endhead

\hline
\multicolumn{2}{r}{\makebox[0pt][r]{\sphinxtablecontinued{Continued on next page}}}\\
\endfoot

\endlastfoot

\sphinxcode{\sphinxupquote{BaseEstimator}}
&
Base class for all estimators in scikit-learn.
\\
\hline
{\hyperref[\detokenize{api/mastml.feature_generators.BaseGenerator:mastml.feature_generators.BaseGenerator}]{\sphinxcrossref{\sphinxcode{\sphinxupquote{BaseGenerator}}}}}()
&
Class functioning as a base generator to support directory organization and evaluating different feature generators
\\
\hline
\sphinxcode{\sphinxupquote{Composition}}(*args{[}, strict{]})
&
Represents a Composition, which is essentially a \{element:amount\} mapping type.
\\
\hline
\sphinxcode{\sphinxupquote{CompositionToOxidComposition}}({[}…{]})
&
Utility featurizer to add oxidation states to a pymatgen Composition.
\\
\hline
{\hyperref[\detokenize{api/mastml.feature_generators.DataframeUtilities:mastml.feature_generators.DataframeUtilities}]{\sphinxcrossref{\sphinxcode{\sphinxupquote{DataframeUtilities}}}}}
&
Class of basic utilities for dataframe manipulation, and exchanging between dataframes and numpy arrays
\\
\hline
\sphinxcode{\sphinxupquote{Element}}(symbol)
&
Enum representing an element in the periodic table.
\\
\hline
\sphinxcode{\sphinxupquote{ElementProperty}}(data\_source, features, stats)
&
Class to calculate elemental property attributes.
\\
\hline
{\hyperref[\detokenize{api/mastml.feature_generators.ElementalFeatureGenerator:mastml.feature_generators.ElementalFeatureGenerator}]{\sphinxcrossref{\sphinxcode{\sphinxupquote{ElementalFeatureGenerator}}}}}(composition\_df{[}, …{]})
&
Class that is used to create elemental-based features from material composition strings
\\
\hline
\sphinxcode{\sphinxupquote{MPRester}}({[}api\_key, endpoint, …{]})
&
A class to conveniently interface with the Materials Project REST interface.
\\
\hline
{\hyperref[\detokenize{api/mastml.feature_generators.MaterialsProjectFeatureGenerator:mastml.feature_generators.MaterialsProjectFeatureGenerator}]{\sphinxcrossref{\sphinxcode{\sphinxupquote{MaterialsProjectFeatureGenerator}}}}}(…)
&
Class that wraps MaterialsProjectFeatureGeneration, giving it scikit-learn structure
\\
\hline
{\hyperref[\detokenize{api/mastml.feature_generators.MatminerFeatureGenerator:mastml.feature_generators.MatminerFeatureGenerator}]{\sphinxcrossref{\sphinxcode{\sphinxupquote{MatminerFeatureGenerator}}}}}(featurize\_df, …)
&
Class to wrap feature generator routines contained in the matminer package to more neatly conform to the MAST-ML working environment, and have all under a single class
\\
\hline
{\hyperref[\detokenize{api/mastml.feature_generators.OneHotElementEncoder:mastml.feature_generators.OneHotElementEncoder}]{\sphinxcrossref{\sphinxcode{\sphinxupquote{OneHotElementEncoder}}}}}(composition\_df{[}, …{]})
&
Class to generate new categorical features (i.e.
\\
\hline
\sphinxcode{\sphinxupquote{OneHotEncoder}}(*{[}, categories, drop, sparse, …{]})
&
Encode categorical features as a one-hot numeric array.
\\
\hline
{\hyperref[\detokenize{api/mastml.feature_generators.OneHotGroupGenerator:mastml.feature_generators.OneHotGroupGenerator}]{\sphinxcrossref{\sphinxcode{\sphinxupquote{OneHotGroupGenerator}}}}}(groups{[}, …{]})
&
Class to generate one-hot encoded values from a list of categories using scikit-learn’s one hot encoder method More info at: \sphinxurl{https://scikit-learn.org/stable/modules/generated/sklearn.preprocessing.OneHotEncoder.html}
\\
\hline
\sphinxcode{\sphinxupquote{OxidationStates}}({[}stats{]})
&
Statistics about the oxidation states for each specie.
\\
\hline
{\hyperref[\detokenize{api/mastml.feature_generators.PolynomialFeatureGenerator:mastml.feature_generators.PolynomialFeatureGenerator}]{\sphinxcrossref{\sphinxcode{\sphinxupquote{PolynomialFeatureGenerator}}}}}({[}features, …{]})
&
Class to generate polynomial features using scikit-learn’s polynomial features method More info at: \sphinxurl{http://scikit-learn.org/stable/modules/generated/sklearn.preprocessing.PolynomialFeatures.html}
\\
\hline
\sphinxcode{\sphinxupquote{PolynomialFeatures}}({[}degree, …{]})
&
Generate polynomial and interaction features.
\\
\hline
\sphinxcode{\sphinxupquote{StrToComposition}}({[}reduce, target\_col\_id, …{]})
&
Utility featurizer to convert a string to a Composition
\\
\hline
\sphinxcode{\sphinxupquote{TransformerMixin}}
&
Mixin class for all transformers in scikit-learn.
\\
\hline
\sphinxcode{\sphinxupquote{datetime}}(year, month, day{[}, hour{[}, minute{[}, …)
&
The year, month and day arguments are required.
\\
\hline
\end{longtable}\sphinxatlongtableend\end{savenotes}


\subsubsection{BaseGenerator}
\label{\detokenize{api/mastml.feature_generators.BaseGenerator:basegenerator}}\label{\detokenize{api/mastml.feature_generators.BaseGenerator::doc}}\index{BaseGenerator (class in mastml.feature\_generators)@\spxentry{BaseGenerator}\spxextra{class in mastml.feature\_generators}}

\begin{fulllineitems}
\phantomsection\label{\detokenize{api/mastml.feature_generators.BaseGenerator:mastml.feature_generators.BaseGenerator}}\pysigline{\sphinxbfcode{\sphinxupquote{class }}\sphinxcode{\sphinxupquote{mastml.feature\_generators.}}\sphinxbfcode{\sphinxupquote{BaseGenerator}}}
Bases: \sphinxcode{\sphinxupquote{sklearn.base.BaseEstimator}}, \sphinxcode{\sphinxupquote{sklearn.base.TransformerMixin}}

Class functioning as a base generator to support directory organization and evaluating different feature generators
\begin{description}
\item[{Args:}] \leavevmode
None

\item[{Methods:}] \leavevmode\begin{description}
\item[{evaluate: main method to run feature generators on supplied data, and save to file}] \leavevmode\begin{description}
\item[{Args:}] \leavevmode
X: (pd.DataFrame), dataframe of X data containing features and composition string information

y: (pd.Series), series of y target data

savepath: (str) string denoting the main save path directory

\item[{Returns:}] \leavevmode
X: (pd.DataFrame), dataframe of X features containing newly generated features

y: (pd.Series), series of y target data

\end{description}

\item[{\_setup\_savedir: method to set up a save directory for the generated dataset}] \leavevmode\begin{description}
\item[{Args:}] \leavevmode
generator: mastml.feature\_generators instance, e.g. ElementalFeatureGenerator

savepath: (str) string denoting the main save path directory

\item[{Returns:}] \leavevmode
splitdir: (str) string of the split directory where generated feature data is saved

\end{description}

\end{description}

\end{description}
\subsubsection*{Methods Summary}


\begin{savenotes}\sphinxatlongtablestart\begin{longtable}[c]{\X{1}{2}\X{1}{2}}
\hline

\endfirsthead

\multicolumn{2}{c}%
{\makebox[0pt]{\sphinxtablecontinued{\tablename\ \thetable{} -- continued from previous page}}}\\
\hline

\endhead

\hline
\multicolumn{2}{r}{\makebox[0pt][r]{\sphinxtablecontinued{Continued on next page}}}\\
\endfoot

\endlastfoot

{\hyperref[\detokenize{api/mastml.feature_generators.BaseGenerator:mastml.feature_generators.BaseGenerator.evaluate}]{\sphinxcrossref{\sphinxcode{\sphinxupquote{evaluate}}}}}(X, y{[}, savepath, make\_new\_dir{]})
&

\\
\hline
\end{longtable}\sphinxatlongtableend\end{savenotes}
\subsubsection*{Methods Documentation}
\index{evaluate() (mastml.feature\_generators.BaseGenerator method)@\spxentry{evaluate()}\spxextra{mastml.feature\_generators.BaseGenerator method}}

\begin{fulllineitems}
\phantomsection\label{\detokenize{api/mastml.feature_generators.BaseGenerator:mastml.feature_generators.BaseGenerator.evaluate}}\pysiglinewithargsret{\sphinxbfcode{\sphinxupquote{evaluate}}}{\emph{X}, \emph{y}, \emph{savepath=None}, \emph{make\_new\_dir=True}}{}
\end{fulllineitems}


\end{fulllineitems}



\subsubsection{DataframeUtilities}
\label{\detokenize{api/mastml.feature_generators.DataframeUtilities:dataframeutilities}}\label{\detokenize{api/mastml.feature_generators.DataframeUtilities::doc}}\index{DataframeUtilities (class in mastml.feature\_generators)@\spxentry{DataframeUtilities}\spxextra{class in mastml.feature\_generators}}

\begin{fulllineitems}
\phantomsection\label{\detokenize{api/mastml.feature_generators.DataframeUtilities:mastml.feature_generators.DataframeUtilities}}\pysigline{\sphinxbfcode{\sphinxupquote{class }}\sphinxcode{\sphinxupquote{mastml.feature\_generators.}}\sphinxbfcode{\sphinxupquote{DataframeUtilities}}}
Bases: \sphinxcode{\sphinxupquote{object}}

Class of basic utilities for dataframe manipulation, and exchanging between dataframes and numpy arrays
\begin{description}
\item[{Args:}] \leavevmode
None

\item[{Methods:}] \leavevmode\begin{description}
\item[{clean\_dataframe}] \leavevmode{[}Method to clean dataframes after feature generation has occurred, to remove columns that have a single missing or NaN value, or remove a row that is fully empty{]}\begin{description}
\item[{Args:}] \leavevmode
df: (dataframe), a post feature generation dataframe that needs cleaning

\end{description}

\item[{Returns:}] \leavevmode
df: (dataframe), the cleaned dataframe

\item[{merge\_dataframe\_columns}] \leavevmode{[}merge two dataframes by concatenating the column names (duplicate columns omitted){]}\begin{description}
\item[{Args:}] \leavevmode
dataframe1: (dataframe), a pandas dataframe object

dataframe2: (dataframe), a pandas dataframe object

\item[{Returns:}] \leavevmode
dataframe: (dataframe), merged dataframe

\end{description}

\item[{merge\_dataframe\_rows}] \leavevmode{[}merge two dataframes by concatenating the row contents (duplicate rows omitted){]}\begin{description}
\item[{Args:}] \leavevmode
dataframe1: (dataframe), a pandas dataframe object

dataframe2: (dataframe), a pandas dataframe object

\item[{Returns:}] \leavevmode
dataframe: (dataframe), merged dataframe

\end{description}

\item[{get\_dataframe\_statistics}] \leavevmode{[}obtain basic statistics about data contained in the dataframe{]}\begin{description}
\item[{Args:}] \leavevmode
dataframe: (dataframe), a pandas dataframe object

\item[{Returns:}] \leavevmode
dataframe\_stats: (dataframe), dataframe containing input dataframe statistics

\end{description}

\item[{dataframe\_to\_array}] \leavevmode{[}transform a pandas dataframe to a numpy array{]}\begin{description}
\item[{Args:}] \leavevmode
dataframe: (dataframe), a pandas dataframe object

\item[{Returns:}] \leavevmode
array: (numpy array), a numpy array representation of the inputted dataframe

\end{description}

\item[{array\_to\_dataframe}] \leavevmode{[}transform a numpy array to a pandas dataframe{]}\begin{description}
\item[{Args:}] \leavevmode
array: (numpy array), a numpy array

\item[{Returns:}] \leavevmode
dataframe: (dataframe), a pandas dataframe representation of the inputted numpy array

\end{description}

\item[{concatenate\_arrays}] \leavevmode{[}merge two numpy arrays by concatenating along the columns{]}\begin{description}
\item[{Args:}] \leavevmode
Xarray: (numpy array), a numpy array object

yarray: (numpy array), a numpy array object

\item[{Returns:}] \leavevmode
array: (numpy array), a numpy array merging the two input arrays

\end{description}

\item[{assign\_columns\_as\_features}] \leavevmode{[}adds column names to dataframe based on the x and y feature names{]}\begin{description}
\item[{Args:}] \leavevmode
dataframe: (dataframe), a pandas dataframe object

x\_features: (list), list containing x feature names

y\_feature: (str), target feature name

\item[{Returns:}] \leavevmode
dataframe: (dataframe), dataframe containing same data as input, with columns labeled with features

\end{description}

\item[{save\_all\_dataframe\_statistics}] \leavevmode{[}obtain dataframe statistics and save it to a csv file{]}\begin{description}
\item[{Args:}] \leavevmode
dataframe: (dataframe), a pandas dataframe object

data\_path: (str), file path to save dataframe statistics to

\item[{Returns:}] \leavevmode
fname: (str), name of file dataframe stats saved to

\end{description}

\end{description}

\end{description}
\subsubsection*{Methods Summary}


\begin{savenotes}\sphinxatlongtablestart\begin{longtable}[c]{\X{1}{2}\X{1}{2}}
\hline

\endfirsthead

\multicolumn{2}{c}%
{\makebox[0pt]{\sphinxtablecontinued{\tablename\ \thetable{} -- continued from previous page}}}\\
\hline

\endhead

\hline
\multicolumn{2}{r}{\makebox[0pt][r]{\sphinxtablecontinued{Continued on next page}}}\\
\endfoot

\endlastfoot

{\hyperref[\detokenize{api/mastml.feature_generators.DataframeUtilities:mastml.feature_generators.DataframeUtilities.array_to_dataframe}]{\sphinxcrossref{\sphinxcode{\sphinxupquote{array\_to\_dataframe}}}}}(array)
&

\\
\hline
{\hyperref[\detokenize{api/mastml.feature_generators.DataframeUtilities:mastml.feature_generators.DataframeUtilities.assign_columns_as_features}]{\sphinxcrossref{\sphinxcode{\sphinxupquote{assign\_columns\_as\_features}}}}}(dataframe, …{[}, …{]})
&

\\
\hline
{\hyperref[\detokenize{api/mastml.feature_generators.DataframeUtilities:mastml.feature_generators.DataframeUtilities.clean_dataframe}]{\sphinxcrossref{\sphinxcode{\sphinxupquote{clean\_dataframe}}}}}(df)
&

\\
\hline
{\hyperref[\detokenize{api/mastml.feature_generators.DataframeUtilities:mastml.feature_generators.DataframeUtilities.concatenate_arrays}]{\sphinxcrossref{\sphinxcode{\sphinxupquote{concatenate\_arrays}}}}}(X\_array, y\_array)
&

\\
\hline
{\hyperref[\detokenize{api/mastml.feature_generators.DataframeUtilities:mastml.feature_generators.DataframeUtilities.dataframe_to_array}]{\sphinxcrossref{\sphinxcode{\sphinxupquote{dataframe\_to\_array}}}}}(dataframe)
&

\\
\hline
{\hyperref[\detokenize{api/mastml.feature_generators.DataframeUtilities:mastml.feature_generators.DataframeUtilities.get_dataframe_statistics}]{\sphinxcrossref{\sphinxcode{\sphinxupquote{get\_dataframe\_statistics}}}}}(dataframe)
&

\\
\hline
{\hyperref[\detokenize{api/mastml.feature_generators.DataframeUtilities:mastml.feature_generators.DataframeUtilities.merge_dataframe_columns}]{\sphinxcrossref{\sphinxcode{\sphinxupquote{merge\_dataframe\_columns}}}}}(dataframe1, dataframe2)
&

\\
\hline
{\hyperref[\detokenize{api/mastml.feature_generators.DataframeUtilities:mastml.feature_generators.DataframeUtilities.merge_dataframe_rows}]{\sphinxcrossref{\sphinxcode{\sphinxupquote{merge\_dataframe\_rows}}}}}(dataframe1, dataframe2)
&

\\
\hline
{\hyperref[\detokenize{api/mastml.feature_generators.DataframeUtilities:mastml.feature_generators.DataframeUtilities.remove_constant_columns}]{\sphinxcrossref{\sphinxcode{\sphinxupquote{remove\_constant\_columns}}}}}(dataframe)
&

\\
\hline
{\hyperref[\detokenize{api/mastml.feature_generators.DataframeUtilities:mastml.feature_generators.DataframeUtilities.save_all_dataframe_statistics}]{\sphinxcrossref{\sphinxcode{\sphinxupquote{save\_all\_dataframe\_statistics}}}}}(dataframe, …)
&

\\
\hline
\end{longtable}\sphinxatlongtableend\end{savenotes}
\subsubsection*{Methods Documentation}
\index{array\_to\_dataframe() (mastml.feature\_generators.DataframeUtilities class method)@\spxentry{array\_to\_dataframe()}\spxextra{mastml.feature\_generators.DataframeUtilities class method}}

\begin{fulllineitems}
\phantomsection\label{\detokenize{api/mastml.feature_generators.DataframeUtilities:mastml.feature_generators.DataframeUtilities.array_to_dataframe}}\pysiglinewithargsret{\sphinxbfcode{\sphinxupquote{classmethod }}\sphinxbfcode{\sphinxupquote{array\_to\_dataframe}}}{\emph{array}}{}
\end{fulllineitems}

\index{assign\_columns\_as\_features() (mastml.feature\_generators.DataframeUtilities class method)@\spxentry{assign\_columns\_as\_features()}\spxextra{mastml.feature\_generators.DataframeUtilities class method}}

\begin{fulllineitems}
\phantomsection\label{\detokenize{api/mastml.feature_generators.DataframeUtilities:mastml.feature_generators.DataframeUtilities.assign_columns_as_features}}\pysiglinewithargsret{\sphinxbfcode{\sphinxupquote{classmethod }}\sphinxbfcode{\sphinxupquote{assign\_columns\_as\_features}}}{\emph{dataframe}, \emph{x\_features}, \emph{y\_feature}, \emph{remove\_first\_row=True}}{}
\end{fulllineitems}

\index{clean\_dataframe() (mastml.feature\_generators.DataframeUtilities class method)@\spxentry{clean\_dataframe()}\spxextra{mastml.feature\_generators.DataframeUtilities class method}}

\begin{fulllineitems}
\phantomsection\label{\detokenize{api/mastml.feature_generators.DataframeUtilities:mastml.feature_generators.DataframeUtilities.clean_dataframe}}\pysiglinewithargsret{\sphinxbfcode{\sphinxupquote{classmethod }}\sphinxbfcode{\sphinxupquote{clean\_dataframe}}}{\emph{df}}{}
\end{fulllineitems}

\index{concatenate\_arrays() (mastml.feature\_generators.DataframeUtilities class method)@\spxentry{concatenate\_arrays()}\spxextra{mastml.feature\_generators.DataframeUtilities class method}}

\begin{fulllineitems}
\phantomsection\label{\detokenize{api/mastml.feature_generators.DataframeUtilities:mastml.feature_generators.DataframeUtilities.concatenate_arrays}}\pysiglinewithargsret{\sphinxbfcode{\sphinxupquote{classmethod }}\sphinxbfcode{\sphinxupquote{concatenate\_arrays}}}{\emph{X\_array}, \emph{y\_array}}{}
\end{fulllineitems}

\index{dataframe\_to\_array() (mastml.feature\_generators.DataframeUtilities class method)@\spxentry{dataframe\_to\_array()}\spxextra{mastml.feature\_generators.DataframeUtilities class method}}

\begin{fulllineitems}
\phantomsection\label{\detokenize{api/mastml.feature_generators.DataframeUtilities:mastml.feature_generators.DataframeUtilities.dataframe_to_array}}\pysiglinewithargsret{\sphinxbfcode{\sphinxupquote{classmethod }}\sphinxbfcode{\sphinxupquote{dataframe\_to\_array}}}{\emph{dataframe}}{}
\end{fulllineitems}

\index{get\_dataframe\_statistics() (mastml.feature\_generators.DataframeUtilities class method)@\spxentry{get\_dataframe\_statistics()}\spxextra{mastml.feature\_generators.DataframeUtilities class method}}

\begin{fulllineitems}
\phantomsection\label{\detokenize{api/mastml.feature_generators.DataframeUtilities:mastml.feature_generators.DataframeUtilities.get_dataframe_statistics}}\pysiglinewithargsret{\sphinxbfcode{\sphinxupquote{classmethod }}\sphinxbfcode{\sphinxupquote{get\_dataframe\_statistics}}}{\emph{dataframe}}{}
\end{fulllineitems}

\index{merge\_dataframe\_columns() (mastml.feature\_generators.DataframeUtilities class method)@\spxentry{merge\_dataframe\_columns()}\spxextra{mastml.feature\_generators.DataframeUtilities class method}}

\begin{fulllineitems}
\phantomsection\label{\detokenize{api/mastml.feature_generators.DataframeUtilities:mastml.feature_generators.DataframeUtilities.merge_dataframe_columns}}\pysiglinewithargsret{\sphinxbfcode{\sphinxupquote{classmethod }}\sphinxbfcode{\sphinxupquote{merge\_dataframe\_columns}}}{\emph{dataframe1}, \emph{dataframe2}}{}
\end{fulllineitems}

\index{merge\_dataframe\_rows() (mastml.feature\_generators.DataframeUtilities class method)@\spxentry{merge\_dataframe\_rows()}\spxextra{mastml.feature\_generators.DataframeUtilities class method}}

\begin{fulllineitems}
\phantomsection\label{\detokenize{api/mastml.feature_generators.DataframeUtilities:mastml.feature_generators.DataframeUtilities.merge_dataframe_rows}}\pysiglinewithargsret{\sphinxbfcode{\sphinxupquote{classmethod }}\sphinxbfcode{\sphinxupquote{merge\_dataframe\_rows}}}{\emph{dataframe1}, \emph{dataframe2}}{}
\end{fulllineitems}

\index{remove\_constant\_columns() (mastml.feature\_generators.DataframeUtilities class method)@\spxentry{remove\_constant\_columns()}\spxextra{mastml.feature\_generators.DataframeUtilities class method}}

\begin{fulllineitems}
\phantomsection\label{\detokenize{api/mastml.feature_generators.DataframeUtilities:mastml.feature_generators.DataframeUtilities.remove_constant_columns}}\pysiglinewithargsret{\sphinxbfcode{\sphinxupquote{classmethod }}\sphinxbfcode{\sphinxupquote{remove\_constant\_columns}}}{\emph{dataframe}}{}
\end{fulllineitems}

\index{save\_all\_dataframe\_statistics() (mastml.feature\_generators.DataframeUtilities class method)@\spxentry{save\_all\_dataframe\_statistics()}\spxextra{mastml.feature\_generators.DataframeUtilities class method}}

\begin{fulllineitems}
\phantomsection\label{\detokenize{api/mastml.feature_generators.DataframeUtilities:mastml.feature_generators.DataframeUtilities.save_all_dataframe_statistics}}\pysiglinewithargsret{\sphinxbfcode{\sphinxupquote{classmethod }}\sphinxbfcode{\sphinxupquote{save\_all\_dataframe\_statistics}}}{\emph{dataframe}, \emph{configdict}}{}
\end{fulllineitems}


\end{fulllineitems}



\subsubsection{ElementalFeatureGenerator}
\label{\detokenize{api/mastml.feature_generators.ElementalFeatureGenerator:elementalfeaturegenerator}}\label{\detokenize{api/mastml.feature_generators.ElementalFeatureGenerator::doc}}\index{ElementalFeatureGenerator (class in mastml.feature\_generators)@\spxentry{ElementalFeatureGenerator}\spxextra{class in mastml.feature\_generators}}

\begin{fulllineitems}
\phantomsection\label{\detokenize{api/mastml.feature_generators.ElementalFeatureGenerator:mastml.feature_generators.ElementalFeatureGenerator}}\pysiglinewithargsret{\sphinxbfcode{\sphinxupquote{class }}\sphinxcode{\sphinxupquote{mastml.feature\_generators.}}\sphinxbfcode{\sphinxupquote{ElementalFeatureGenerator}}}{\emph{composition\_df}, \emph{feature\_types=None}, \emph{remove\_constant\_columns=False}}{}
Bases: {\hyperref[\detokenize{api/mastml.feature_generators.BaseGenerator:mastml.feature_generators.BaseGenerator}]{\sphinxcrossref{\sphinxcode{\sphinxupquote{mastml.feature\_generators.BaseGenerator}}}}}

Class that is used to create elemental-based features from material composition strings
\begin{description}
\item[{Args:}] \leavevmode
composition\_df: (pd.DataFrame), dataframe containing vector of chemical compositions (strings) to generate elemental features from

feature\_types: (list), list of strings denoting which elemental feature types to include in the final feature matrix.

remove\_constant\_columns: (bool), whether to remove constant columns from the generated feature set

\item[{Methods:}] \leavevmode\begin{description}
\item[{fit: pass through, copies input columns as pre-generated features}] \leavevmode\begin{description}
\item[{Args:}] \leavevmode
X: (pd.DataFrame), input dataframe containing X data

y: (pd.Series), series containing y data

\end{description}

\item[{transform: generate the elemental feature matrix from composition strings}] \leavevmode\begin{description}
\item[{Args:}] \leavevmode
None.

\item[{Returns:}] \leavevmode
X: (dataframe), output dataframe containing generated features

y: (series), output y data as series

\end{description}

\end{description}

\end{description}
\subsubsection*{Methods Summary}


\begin{savenotes}\sphinxatlongtablestart\begin{longtable}[c]{\X{1}{2}\X{1}{2}}
\hline

\endfirsthead

\multicolumn{2}{c}%
{\makebox[0pt]{\sphinxtablecontinued{\tablename\ \thetable{} -- continued from previous page}}}\\
\hline

\endhead

\hline
\multicolumn{2}{r}{\makebox[0pt][r]{\sphinxtablecontinued{Continued on next page}}}\\
\endfoot

\endlastfoot

{\hyperref[\detokenize{api/mastml.feature_generators.ElementalFeatureGenerator:mastml.feature_generators.ElementalFeatureGenerator.fit}]{\sphinxcrossref{\sphinxcode{\sphinxupquote{fit}}}}}({[}X, y{]})
&

\\
\hline
{\hyperref[\detokenize{api/mastml.feature_generators.ElementalFeatureGenerator:mastml.feature_generators.ElementalFeatureGenerator.generate_magpie_features}]{\sphinxcrossref{\sphinxcode{\sphinxupquote{generate\_magpie\_features}}}}}()
&

\\
\hline
{\hyperref[\detokenize{api/mastml.feature_generators.ElementalFeatureGenerator:mastml.feature_generators.ElementalFeatureGenerator.transform}]{\sphinxcrossref{\sphinxcode{\sphinxupquote{transform}}}}}({[}X{]})
&

\\
\hline
\end{longtable}\sphinxatlongtableend\end{savenotes}
\subsubsection*{Methods Documentation}
\index{fit() (mastml.feature\_generators.ElementalFeatureGenerator method)@\spxentry{fit()}\spxextra{mastml.feature\_generators.ElementalFeatureGenerator method}}

\begin{fulllineitems}
\phantomsection\label{\detokenize{api/mastml.feature_generators.ElementalFeatureGenerator:mastml.feature_generators.ElementalFeatureGenerator.fit}}\pysiglinewithargsret{\sphinxbfcode{\sphinxupquote{fit}}}{\emph{X=None}, \emph{y=None}}{}
\end{fulllineitems}

\index{generate\_magpie\_features() (mastml.feature\_generators.ElementalFeatureGenerator method)@\spxentry{generate\_magpie\_features()}\spxextra{mastml.feature\_generators.ElementalFeatureGenerator method}}

\begin{fulllineitems}
\phantomsection\label{\detokenize{api/mastml.feature_generators.ElementalFeatureGenerator:mastml.feature_generators.ElementalFeatureGenerator.generate_magpie_features}}\pysiglinewithargsret{\sphinxbfcode{\sphinxupquote{generate\_magpie\_features}}}{}{}
\end{fulllineitems}

\index{transform() (mastml.feature\_generators.ElementalFeatureGenerator method)@\spxentry{transform()}\spxextra{mastml.feature\_generators.ElementalFeatureGenerator method}}

\begin{fulllineitems}
\phantomsection\label{\detokenize{api/mastml.feature_generators.ElementalFeatureGenerator:mastml.feature_generators.ElementalFeatureGenerator.transform}}\pysiglinewithargsret{\sphinxbfcode{\sphinxupquote{transform}}}{\emph{X=None}}{}
\end{fulllineitems}


\end{fulllineitems}



\subsubsection{MaterialsProjectFeatureGenerator}
\label{\detokenize{api/mastml.feature_generators.MaterialsProjectFeatureGenerator:materialsprojectfeaturegenerator}}\label{\detokenize{api/mastml.feature_generators.MaterialsProjectFeatureGenerator::doc}}\index{MaterialsProjectFeatureGenerator (class in mastml.feature\_generators)@\spxentry{MaterialsProjectFeatureGenerator}\spxextra{class in mastml.feature\_generators}}

\begin{fulllineitems}
\phantomsection\label{\detokenize{api/mastml.feature_generators.MaterialsProjectFeatureGenerator:mastml.feature_generators.MaterialsProjectFeatureGenerator}}\pysiglinewithargsret{\sphinxbfcode{\sphinxupquote{class }}\sphinxcode{\sphinxupquote{mastml.feature\_generators.}}\sphinxbfcode{\sphinxupquote{MaterialsProjectFeatureGenerator}}}{\emph{composition\_df}, \emph{api\_key}}{}
Bases: {\hyperref[\detokenize{api/mastml.feature_generators.BaseGenerator:mastml.feature_generators.BaseGenerator}]{\sphinxcrossref{\sphinxcode{\sphinxupquote{mastml.feature\_generators.BaseGenerator}}}}}

Class that wraps MaterialsProjectFeatureGeneration, giving it scikit-learn structure
\begin{description}
\item[{Args:}] \leavevmode
composition\_df: (pd.DataFrame), dataframe containing vector of chemical compositions (strings) to generate elemental features from

mapi\_key: (str), string denoting your Materials Project API key

\item[{Methods:}] \leavevmode\begin{description}
\item[{fit: pass through, copies input columns as pre-generated features}] \leavevmode\begin{description}
\item[{Args:}] \leavevmode
df: (dataframe), input dataframe containing X and y data

\end{description}

\item[{transform: generate Materials Project features}] \leavevmode\begin{description}
\item[{Args:}] \leavevmode
df: (dataframe), input dataframe containing X and y data

\end{description}

\item[{Returns:}] \leavevmode
df: (dataframe), output dataframe containing generated features, original features and y data

\end{description}

\end{description}
\subsubsection*{Methods Summary}


\begin{savenotes}\sphinxatlongtablestart\begin{longtable}[c]{\X{1}{2}\X{1}{2}}
\hline

\endfirsthead

\multicolumn{2}{c}%
{\makebox[0pt]{\sphinxtablecontinued{\tablename\ \thetable{} -- continued from previous page}}}\\
\hline

\endhead

\hline
\multicolumn{2}{r}{\makebox[0pt][r]{\sphinxtablecontinued{Continued on next page}}}\\
\endfoot

\endlastfoot

{\hyperref[\detokenize{api/mastml.feature_generators.MaterialsProjectFeatureGenerator:mastml.feature_generators.MaterialsProjectFeatureGenerator.fit}]{\sphinxcrossref{\sphinxcode{\sphinxupquote{fit}}}}}(X{[}, y{]})
&

\\
\hline
{\hyperref[\detokenize{api/mastml.feature_generators.MaterialsProjectFeatureGenerator:mastml.feature_generators.MaterialsProjectFeatureGenerator.generate_materialsproject_features}]{\sphinxcrossref{\sphinxcode{\sphinxupquote{generate\_materialsproject\_features}}}}}(X)
&

\\
\hline
{\hyperref[\detokenize{api/mastml.feature_generators.MaterialsProjectFeatureGenerator:mastml.feature_generators.MaterialsProjectFeatureGenerator.transform}]{\sphinxcrossref{\sphinxcode{\sphinxupquote{transform}}}}}(X)
&

\\
\hline
\end{longtable}\sphinxatlongtableend\end{savenotes}
\subsubsection*{Methods Documentation}
\index{fit() (mastml.feature\_generators.MaterialsProjectFeatureGenerator method)@\spxentry{fit()}\spxextra{mastml.feature\_generators.MaterialsProjectFeatureGenerator method}}

\begin{fulllineitems}
\phantomsection\label{\detokenize{api/mastml.feature_generators.MaterialsProjectFeatureGenerator:mastml.feature_generators.MaterialsProjectFeatureGenerator.fit}}\pysiglinewithargsret{\sphinxbfcode{\sphinxupquote{fit}}}{\emph{X}, \emph{y=None}}{}
\end{fulllineitems}

\index{generate\_materialsproject\_features() (mastml.feature\_generators.MaterialsProjectFeatureGenerator method)@\spxentry{generate\_materialsproject\_features()}\spxextra{mastml.feature\_generators.MaterialsProjectFeatureGenerator method}}

\begin{fulllineitems}
\phantomsection\label{\detokenize{api/mastml.feature_generators.MaterialsProjectFeatureGenerator:mastml.feature_generators.MaterialsProjectFeatureGenerator.generate_materialsproject_features}}\pysiglinewithargsret{\sphinxbfcode{\sphinxupquote{generate\_materialsproject\_features}}}{\emph{X}}{}
\end{fulllineitems}

\index{transform() (mastml.feature\_generators.MaterialsProjectFeatureGenerator method)@\spxentry{transform()}\spxextra{mastml.feature\_generators.MaterialsProjectFeatureGenerator method}}

\begin{fulllineitems}
\phantomsection\label{\detokenize{api/mastml.feature_generators.MaterialsProjectFeatureGenerator:mastml.feature_generators.MaterialsProjectFeatureGenerator.transform}}\pysiglinewithargsret{\sphinxbfcode{\sphinxupquote{transform}}}{\emph{X}}{}
\end{fulllineitems}


\end{fulllineitems}



\subsubsection{MatminerFeatureGenerator}
\label{\detokenize{api/mastml.feature_generators.MatminerFeatureGenerator:matminerfeaturegenerator}}\label{\detokenize{api/mastml.feature_generators.MatminerFeatureGenerator::doc}}\index{MatminerFeatureGenerator (class in mastml.feature\_generators)@\spxentry{MatminerFeatureGenerator}\spxextra{class in mastml.feature\_generators}}

\begin{fulllineitems}
\phantomsection\label{\detokenize{api/mastml.feature_generators.MatminerFeatureGenerator:mastml.feature_generators.MatminerFeatureGenerator}}\pysiglinewithargsret{\sphinxbfcode{\sphinxupquote{class }}\sphinxcode{\sphinxupquote{mastml.feature\_generators.}}\sphinxbfcode{\sphinxupquote{MatminerFeatureGenerator}}}{\emph{featurize\_df, featurizer, composition\_feature\_types={[}'magpie', 'deml', 'matminer'{]}, structure\_feature\_type='CoulombMatrix', remove\_constant\_columns=False, **kwargs}}{}
Bases: {\hyperref[\detokenize{api/mastml.feature_generators.BaseGenerator:mastml.feature_generators.BaseGenerator}]{\sphinxcrossref{\sphinxcode{\sphinxupquote{mastml.feature\_generators.BaseGenerator}}}}}

Class to wrap feature generator routines contained in the matminer package to more neatly conform to the
MAST-ML working environment, and have all under a single class
\begin{description}
\item[{Args:}] \leavevmode
featurize\_df: (pd.DataFrame), input dataframe to be featurized. Needs to contain at least a column with chemical compositions or pymatgen Structure objects

featurizer: (str), type of featurization to conduct. Valid names are “composition” or “structure”

composition\_feature\_types: (list of str), if featurizer=’composition’, the type of composition-based features to include. Valid values are ‘magpie’, ‘deml’, ‘matminer’, ‘matscholar\_el’, ‘megnet\_el’

structure\_feature\_type: (str), if featurizer=’structure’, the type of structure-based featurization to conduct. See list below for valid names.

remove\_constant\_columns: (bool), whether or not to remove feature columns that are constant values. Default is False.

kwargs: additional keyword arguments needed if structure based features are being made

\item[{Available structure featurizer types for structure\_feature\_types:}] \leavevmode
matminer.featurizers.structure
{[}‘BagofBonds’, x
‘BondFractions’, x
‘ChemicalOrdering’, x
‘CoulombMatrix’, x
‘DensityFeatures’, x
‘Dimensionality’, x
‘ElectronicRadialDistributionFunction’, NEEDS TO BE FLATTENED
‘EwaldEnergy’, x, returns all NaN
‘GlobalInstabilityIndex’, x, returns all NaN
‘GlobalSymmetryFeatures’, x
‘JarvisCFID’, x
‘MaximumPackingEfficiency’, x
‘MinimumRelativeDistances’, x
‘OrbitalFieldMatrix’, x
‘PartialRadialDistributionFunction’, x
‘RadialDistributionFunction’, NEEDS TO BE FLATTENED
‘SineCoulombMatrix’, x
‘SiteStatsFingerprint’, x
‘StructuralComplexity’, x
‘StructuralHeterogeneity’, x
‘XRDPowderPattern’{]} x

matminer.featurizers.site
{[}‘AGNIFingerprints’,
‘AngularFourierSeries’,
‘AseAtomsAdaptor’,
‘AverageBondAngle’,
‘AverageBondLength’,
‘BondOrientationalParameter’,
‘ChemEnvSiteFingerprint’,
‘ChemicalSRO’,
‘ConvexHull’,
‘CoordinationNumber’,
‘CrystalNN’,
‘CrystalNNFingerprint’,
‘Element’,
‘EwaldSiteEnergy’,
‘EwaldSummation’,
‘Gaussian’,
‘GaussianSymmFunc’,
‘GeneralizedRadialDistributionFunction’,
‘Histogram’,
‘IntersticeDistribution’,
‘LocalGeometryFinder’,
‘LocalPropertyDifference’,
‘LocalStructOrderParams’,
‘MagpieData’,
‘MultiWeightsChemenvStrategy’,
‘OPSiteFingerprint’,
‘SOAP’,
‘SimplestChemenvStrategy’,
‘SiteElementalProperty’,
‘VoronoiFingerprint’,
‘VoronoiNN’{]}

\item[{Methods:}] \leavevmode\begin{description}
\item[{fit: present for convenience and just passes through}] \leavevmode\begin{description}
\item[{Args:}] \leavevmode
X: (pd.DataFrame), the X feature matrix

\item[{Returns:}] \leavevmode
self

\end{description}

\item[{transform: generates new features and transforms to have new dataframe with generated features}] \leavevmode\begin{description}
\item[{Args:}] \leavevmode
X: (pd.DataFrame), the X feature matrix

\item[{Returns:}] \leavevmode
df: (pd.DataFrame), the transformed dataframe containing generated features

y: (pd.Series), the target y-data

\end{description}

\item[{generate\_matminer\_features: method to generate the composition or structure features of interest}] \leavevmode\begin{description}
\item[{Args:}] \leavevmode
X: (pd.DataFrame), the X feature matrix

\item[{Returns:}] \leavevmode
df: (pd.DataFrame), the transformed dataframe containing generated features

\end{description}

\end{description}

\end{description}
\subsubsection*{Methods Summary}


\begin{savenotes}\sphinxatlongtablestart\begin{longtable}[c]{\X{1}{2}\X{1}{2}}
\hline

\endfirsthead

\multicolumn{2}{c}%
{\makebox[0pt]{\sphinxtablecontinued{\tablename\ \thetable{} -- continued from previous page}}}\\
\hline

\endhead

\hline
\multicolumn{2}{r}{\makebox[0pt][r]{\sphinxtablecontinued{Continued on next page}}}\\
\endfoot

\endlastfoot

{\hyperref[\detokenize{api/mastml.feature_generators.MatminerFeatureGenerator:mastml.feature_generators.MatminerFeatureGenerator.fit}]{\sphinxcrossref{\sphinxcode{\sphinxupquote{fit}}}}}(X{[}, y{]})
&

\\
\hline
{\hyperref[\detokenize{api/mastml.feature_generators.MatminerFeatureGenerator:mastml.feature_generators.MatminerFeatureGenerator.generate_matminer_features}]{\sphinxcrossref{\sphinxcode{\sphinxupquote{generate\_matminer\_features}}}}}(X)
&

\\
\hline
{\hyperref[\detokenize{api/mastml.feature_generators.MatminerFeatureGenerator:mastml.feature_generators.MatminerFeatureGenerator.transform}]{\sphinxcrossref{\sphinxcode{\sphinxupquote{transform}}}}}(X)
&

\\
\hline
\end{longtable}\sphinxatlongtableend\end{savenotes}
\subsubsection*{Methods Documentation}
\index{fit() (mastml.feature\_generators.MatminerFeatureGenerator method)@\spxentry{fit()}\spxextra{mastml.feature\_generators.MatminerFeatureGenerator method}}

\begin{fulllineitems}
\phantomsection\label{\detokenize{api/mastml.feature_generators.MatminerFeatureGenerator:mastml.feature_generators.MatminerFeatureGenerator.fit}}\pysiglinewithargsret{\sphinxbfcode{\sphinxupquote{fit}}}{\emph{X}, \emph{y=None}}{}
\end{fulllineitems}

\index{generate\_matminer\_features() (mastml.feature\_generators.MatminerFeatureGenerator method)@\spxentry{generate\_matminer\_features()}\spxextra{mastml.feature\_generators.MatminerFeatureGenerator method}}

\begin{fulllineitems}
\phantomsection\label{\detokenize{api/mastml.feature_generators.MatminerFeatureGenerator:mastml.feature_generators.MatminerFeatureGenerator.generate_matminer_features}}\pysiglinewithargsret{\sphinxbfcode{\sphinxupquote{generate\_matminer\_features}}}{\emph{X}}{}
\end{fulllineitems}

\index{transform() (mastml.feature\_generators.MatminerFeatureGenerator method)@\spxentry{transform()}\spxextra{mastml.feature\_generators.MatminerFeatureGenerator method}}

\begin{fulllineitems}
\phantomsection\label{\detokenize{api/mastml.feature_generators.MatminerFeatureGenerator:mastml.feature_generators.MatminerFeatureGenerator.transform}}\pysiglinewithargsret{\sphinxbfcode{\sphinxupquote{transform}}}{\emph{X}}{}
\end{fulllineitems}


\end{fulllineitems}



\subsubsection{OneHotElementEncoder}
\label{\detokenize{api/mastml.feature_generators.OneHotElementEncoder:onehotelementencoder}}\label{\detokenize{api/mastml.feature_generators.OneHotElementEncoder::doc}}\index{OneHotElementEncoder (class in mastml.feature\_generators)@\spxentry{OneHotElementEncoder}\spxextra{class in mastml.feature\_generators}}

\begin{fulllineitems}
\phantomsection\label{\detokenize{api/mastml.feature_generators.OneHotElementEncoder:mastml.feature_generators.OneHotElementEncoder}}\pysiglinewithargsret{\sphinxbfcode{\sphinxupquote{class }}\sphinxcode{\sphinxupquote{mastml.feature\_generators.}}\sphinxbfcode{\sphinxupquote{OneHotElementEncoder}}}{\emph{composition\_df}, \emph{remove\_constant\_columns=False}}{}
Bases: {\hyperref[\detokenize{api/mastml.feature_generators.BaseGenerator:mastml.feature_generators.BaseGenerator}]{\sphinxcrossref{\sphinxcode{\sphinxupquote{mastml.feature\_generators.BaseGenerator}}}}}

Class to generate new categorical features (i.e. values of 1 or 0) based on whether an input composition contains a
certain designated element
\begin{description}
\item[{Args:}] \leavevmode
composition\_feature: (str), string denoting a chemical composition to generate elemental features from

element: (str), string representing the name of an element

new\_name: (str), the name of the new feature column to be generated

all\_elments: (bool), whether to generate new features for all elements present from all compositions in the dataset.

\item[{Methods:}] \leavevmode\begin{description}
\item[{fit: pass through, needed to maintain scikit-learn class structure}] \leavevmode\begin{description}
\item[{Args:}] \leavevmode
df: (dataframe), dataframe of input X and y data

\end{description}

\item[{transform: generate new element-specific features}] \leavevmode\begin{description}
\item[{Args:}] \leavevmode
df: (dataframe), dataframe of input X and y data

\end{description}

\item[{Returns:}] \leavevmode
df\_trans: (dataframe), dataframe with generated element-specific features

\end{description}

\end{description}
\subsubsection*{Methods Summary}


\begin{savenotes}\sphinxatlongtablestart\begin{longtable}[c]{\X{1}{2}\X{1}{2}}
\hline

\endfirsthead

\multicolumn{2}{c}%
{\makebox[0pt]{\sphinxtablecontinued{\tablename\ \thetable{} -- continued from previous page}}}\\
\hline

\endhead

\hline
\multicolumn{2}{r}{\makebox[0pt][r]{\sphinxtablecontinued{Continued on next page}}}\\
\endfoot

\endlastfoot

{\hyperref[\detokenize{api/mastml.feature_generators.OneHotElementEncoder:mastml.feature_generators.OneHotElementEncoder.fit}]{\sphinxcrossref{\sphinxcode{\sphinxupquote{fit}}}}}(X{[}, y{]})
&

\\
\hline
{\hyperref[\detokenize{api/mastml.feature_generators.OneHotElementEncoder:mastml.feature_generators.OneHotElementEncoder.transform}]{\sphinxcrossref{\sphinxcode{\sphinxupquote{transform}}}}}(X{[}, y{]})
&

\\
\hline
\end{longtable}\sphinxatlongtableend\end{savenotes}
\subsubsection*{Methods Documentation}
\index{fit() (mastml.feature\_generators.OneHotElementEncoder method)@\spxentry{fit()}\spxextra{mastml.feature\_generators.OneHotElementEncoder method}}

\begin{fulllineitems}
\phantomsection\label{\detokenize{api/mastml.feature_generators.OneHotElementEncoder:mastml.feature_generators.OneHotElementEncoder.fit}}\pysiglinewithargsret{\sphinxbfcode{\sphinxupquote{fit}}}{\emph{X}, \emph{y=None}}{}
\end{fulllineitems}

\index{transform() (mastml.feature\_generators.OneHotElementEncoder method)@\spxentry{transform()}\spxextra{mastml.feature\_generators.OneHotElementEncoder method}}

\begin{fulllineitems}
\phantomsection\label{\detokenize{api/mastml.feature_generators.OneHotElementEncoder:mastml.feature_generators.OneHotElementEncoder.transform}}\pysiglinewithargsret{\sphinxbfcode{\sphinxupquote{transform}}}{\emph{X}, \emph{y=None}}{}
\end{fulllineitems}


\end{fulllineitems}



\subsubsection{OneHotGroupGenerator}
\label{\detokenize{api/mastml.feature_generators.OneHotGroupGenerator:onehotgroupgenerator}}\label{\detokenize{api/mastml.feature_generators.OneHotGroupGenerator::doc}}\index{OneHotGroupGenerator (class in mastml.feature\_generators)@\spxentry{OneHotGroupGenerator}\spxextra{class in mastml.feature\_generators}}

\begin{fulllineitems}
\phantomsection\label{\detokenize{api/mastml.feature_generators.OneHotGroupGenerator:mastml.feature_generators.OneHotGroupGenerator}}\pysiglinewithargsret{\sphinxbfcode{\sphinxupquote{class }}\sphinxcode{\sphinxupquote{mastml.feature\_generators.}}\sphinxbfcode{\sphinxupquote{OneHotGroupGenerator}}}{\emph{groups}, \emph{remove\_constant\_columns=False}}{}
Bases: {\hyperref[\detokenize{api/mastml.feature_generators.BaseGenerator:mastml.feature_generators.BaseGenerator}]{\sphinxcrossref{\sphinxcode{\sphinxupquote{mastml.feature\_generators.BaseGenerator}}}}}

Class to generate one-hot encoded values from a list of categories using scikit-learn’s one hot encoder method
More info at: \sphinxurl{https://scikit-learn.org/stable/modules/generated/sklearn.preprocessing.OneHotEncoder.html}
\begin{description}
\item[{Args:}] \leavevmode
groups: (pd.Series): pandas Series of group (category) names

remove\_constant\_columns: (bool), whether to remove constant columns from the generated feature set

\item[{Methods:}] \leavevmode\begin{description}
\item[{fit: pass through, copies input columns as pre-generated features}] \leavevmode\begin{description}
\item[{Args:}] \leavevmode
X: (pd.DataFrame), input dataframe containing X data

y: (pd.Series), series containing y data

\end{description}

\item[{transform: generate the one-hot encoded features. There will be n columns made, where n = number of unique categories in groups}] \leavevmode\begin{description}
\item[{Args:}] \leavevmode
None.

\item[{Returns:}] \leavevmode
df: (dataframe), output dataframe containing generated features

y: (series), output y data as series

\end{description}

\end{description}

\end{description}
\subsubsection*{Methods Summary}


\begin{savenotes}\sphinxatlongtablestart\begin{longtable}[c]{\X{1}{2}\X{1}{2}}
\hline

\endfirsthead

\multicolumn{2}{c}%
{\makebox[0pt]{\sphinxtablecontinued{\tablename\ \thetable{} -- continued from previous page}}}\\
\hline

\endhead

\hline
\multicolumn{2}{r}{\makebox[0pt][r]{\sphinxtablecontinued{Continued on next page}}}\\
\endfoot

\endlastfoot

{\hyperref[\detokenize{api/mastml.feature_generators.OneHotGroupGenerator:mastml.feature_generators.OneHotGroupGenerator.fit}]{\sphinxcrossref{\sphinxcode{\sphinxupquote{fit}}}}}(X{[}, y{]})
&

\\
\hline
{\hyperref[\detokenize{api/mastml.feature_generators.OneHotGroupGenerator:mastml.feature_generators.OneHotGroupGenerator.transform}]{\sphinxcrossref{\sphinxcode{\sphinxupquote{transform}}}}}({[}X{]})
&

\\
\hline
\end{longtable}\sphinxatlongtableend\end{savenotes}
\subsubsection*{Methods Documentation}
\index{fit() (mastml.feature\_generators.OneHotGroupGenerator method)@\spxentry{fit()}\spxextra{mastml.feature\_generators.OneHotGroupGenerator method}}

\begin{fulllineitems}
\phantomsection\label{\detokenize{api/mastml.feature_generators.OneHotGroupGenerator:mastml.feature_generators.OneHotGroupGenerator.fit}}\pysiglinewithargsret{\sphinxbfcode{\sphinxupquote{fit}}}{\emph{X}, \emph{y=None}}{}
\end{fulllineitems}

\index{transform() (mastml.feature\_generators.OneHotGroupGenerator method)@\spxentry{transform()}\spxextra{mastml.feature\_generators.OneHotGroupGenerator method}}

\begin{fulllineitems}
\phantomsection\label{\detokenize{api/mastml.feature_generators.OneHotGroupGenerator:mastml.feature_generators.OneHotGroupGenerator.transform}}\pysiglinewithargsret{\sphinxbfcode{\sphinxupquote{transform}}}{\emph{X=None}}{}
\end{fulllineitems}


\end{fulllineitems}



\subsubsection{PolynomialFeatureGenerator}
\label{\detokenize{api/mastml.feature_generators.PolynomialFeatureGenerator:polynomialfeaturegenerator}}\label{\detokenize{api/mastml.feature_generators.PolynomialFeatureGenerator::doc}}\index{PolynomialFeatureGenerator (class in mastml.feature\_generators)@\spxentry{PolynomialFeatureGenerator}\spxextra{class in mastml.feature\_generators}}

\begin{fulllineitems}
\phantomsection\label{\detokenize{api/mastml.feature_generators.PolynomialFeatureGenerator:mastml.feature_generators.PolynomialFeatureGenerator}}\pysiglinewithargsret{\sphinxbfcode{\sphinxupquote{class }}\sphinxcode{\sphinxupquote{mastml.feature\_generators.}}\sphinxbfcode{\sphinxupquote{PolynomialFeatureGenerator}}}{\emph{features=None}, \emph{degree=2}, \emph{interaction\_only=False}, \emph{include\_bias=True}}{}
Bases: {\hyperref[\detokenize{api/mastml.feature_generators.BaseGenerator:mastml.feature_generators.BaseGenerator}]{\sphinxcrossref{\sphinxcode{\sphinxupquote{mastml.feature\_generators.BaseGenerator}}}}}

Class to generate polynomial features using scikit-learn’s polynomial features method
More info at: \sphinxurl{http://scikit-learn.org/stable/modules/generated/sklearn.preprocessing.PolynomialFeatures.html}
\begin{description}
\item[{Args:}] \leavevmode
degree: (int), degree of polynomial features

interaction\_only: (bool), If true, only interaction features are produced: features that are products of at most degree distinct input features (so not x{[}1{]} ** 2, x{[}0{]} * x{[}2{]} ** 3, etc.).

include\_bias: (bool),If True (default), then include a bias column, the feature in which all polynomial powers are zero (i.e. a column of ones - acts as an intercept term in a linear model).

\item[{Methods:}] \leavevmode\begin{description}
\item[{fit: conducts fit method of polynomial feature generation}] \leavevmode\begin{description}
\item[{Args:}] \leavevmode
df: (dataframe), dataframe of input X and y data

\end{description}

\item[{transform: generates dataframe containing polynomial features}] \leavevmode\begin{description}
\item[{Args:}] \leavevmode
df: (dataframe), dataframe of input X and y data

\end{description}

\item[{Returns:}] \leavevmode
(dataframe), dataframe containing new polynomial features, plus original features present

\end{description}

\end{description}
\subsubsection*{Methods Summary}


\begin{savenotes}\sphinxatlongtablestart\begin{longtable}[c]{\X{1}{2}\X{1}{2}}
\hline

\endfirsthead

\multicolumn{2}{c}%
{\makebox[0pt]{\sphinxtablecontinued{\tablename\ \thetable{} -- continued from previous page}}}\\
\hline

\endhead

\hline
\multicolumn{2}{r}{\makebox[0pt][r]{\sphinxtablecontinued{Continued on next page}}}\\
\endfoot

\endlastfoot

{\hyperref[\detokenize{api/mastml.feature_generators.PolynomialFeatureGenerator:mastml.feature_generators.PolynomialFeatureGenerator.fit}]{\sphinxcrossref{\sphinxcode{\sphinxupquote{fit}}}}}(X{[}, y{]})
&

\\
\hline
{\hyperref[\detokenize{api/mastml.feature_generators.PolynomialFeatureGenerator:mastml.feature_generators.PolynomialFeatureGenerator.transform}]{\sphinxcrossref{\sphinxcode{\sphinxupquote{transform}}}}}(X)
&

\\
\hline
\end{longtable}\sphinxatlongtableend\end{savenotes}
\subsubsection*{Methods Documentation}
\index{fit() (mastml.feature\_generators.PolynomialFeatureGenerator method)@\spxentry{fit()}\spxextra{mastml.feature\_generators.PolynomialFeatureGenerator method}}

\begin{fulllineitems}
\phantomsection\label{\detokenize{api/mastml.feature_generators.PolynomialFeatureGenerator:mastml.feature_generators.PolynomialFeatureGenerator.fit}}\pysiglinewithargsret{\sphinxbfcode{\sphinxupquote{fit}}}{\emph{X}, \emph{y=None}}{}
\end{fulllineitems}

\index{transform() (mastml.feature\_generators.PolynomialFeatureGenerator method)@\spxentry{transform()}\spxextra{mastml.feature\_generators.PolynomialFeatureGenerator method}}

\begin{fulllineitems}
\phantomsection\label{\detokenize{api/mastml.feature_generators.PolynomialFeatureGenerator:mastml.feature_generators.PolynomialFeatureGenerator.transform}}\pysiglinewithargsret{\sphinxbfcode{\sphinxupquote{transform}}}{\emph{X}}{}
\end{fulllineitems}


\end{fulllineitems}



\subsection{Class Inheritance Diagram}
\label{\detokenize{5_feature_generators:class-inheritance-diagram}}
\sphinxincludegraphics[]{None}


\chapter{Code Documentation: Feature Selectors}
\label{\detokenize{6_feature_selectors:code-documentation-feature-selectors}}\label{\detokenize{6_feature_selectors::doc}}

\section{mastml.feature\_selectors Module}
\label{\detokenize{6_feature_selectors:module-mastml.feature_selectors}}\label{\detokenize{6_feature_selectors:mastml-feature-selectors-module}}\index{mastml.feature\_selectors (module)@\spxentry{mastml.feature\_selectors}\spxextra{module}}
This module contains a collection of routines to perform feature selection.
\begin{description}
\item[{BaseSelector:}] \leavevmode
Base class to have MAST-ML like workflow functionality for feature selectors. All feature selection routines
should inherit this base class

\item[{SklearnFeatureSelector:}] \leavevmode
Class to wrap feature selectors from the scikit-learn package and make them have functionality from
BaseSelector. Any scikit-learn feature selector from sklearn.feature\_selection can be used by providing the
name of the selector class as a string.

\item[{NoSelect:}] \leavevmode
Class that performs no feature selection and just uses all features in the dataset. Needed as a placeholder
when evaluating data splits in a MAST-ML run where feature selection is not performed.

\item[{EnsembleModelFeatureSelector:}] \leavevmode
Class to selects features based on the feature importances scores obtained when fitting an ensemble-based model.
Any model with the {\color{red}\bfseries{}feature\_importances\_} attribute will work, e.g. sklearn’s RandomForestRegressor and
GradientBoostingRegressor.

\item[{PearsonSelector:}] \leavevmode
Class that selects features based on their Pearson correlation score with the target data. Can also be used
to assess Pearson correlation between features for use to reduce dimensionality of the feature space.

\item[{MASTMLFeatureSelector:}] \leavevmode
Class written for MAST-ML to perform more flexible forward selection than what can be found in scikit-learn.
Allows the user to specify a particular model and cross validation routine for selecting features, as well as the
ability to forcibly select certain features on the outset.

\end{description}


\subsection{Classes}
\label{\detokenize{6_feature_selectors:classes}}

\begin{savenotes}\sphinxatlongtablestart\begin{longtable}[c]{\X{1}{2}\X{1}{2}}
\hline

\endfirsthead

\multicolumn{2}{c}%
{\makebox[0pt]{\sphinxtablecontinued{\tablename\ \thetable{} -- continued from previous page}}}\\
\hline

\endhead

\hline
\multicolumn{2}{r}{\makebox[0pt][r]{\sphinxtablecontinued{Continued on next page}}}\\
\endfoot

\endlastfoot

\sphinxcode{\sphinxupquote{BaseEstimator}}
&
Base class for all estimators in scikit-learn.
\\
\hline
{\hyperref[\detokenize{api/mastml.feature_selectors.BaseSelector:mastml.feature_selectors.BaseSelector}]{\sphinxcrossref{\sphinxcode{\sphinxupquote{BaseSelector}}}}}()
&
Base class that forms foundation of MAST-ML feature selectors
\\
\hline
{\hyperref[\detokenize{api/mastml.feature_selectors.EnsembleModelFeatureSelector:mastml.feature_selectors.EnsembleModelFeatureSelector}]{\sphinxcrossref{\sphinxcode{\sphinxupquote{EnsembleModelFeatureSelector}}}}}(model, …{[}, …{]})
&
Class custom-written for MAST-ML to conduct selection of features with ensemble model feature importances
\\
\hline
\sphinxcode{\sphinxupquote{KFold}}({[}n\_splits, shuffle, random\_state{]})
&
K-Folds cross-validator
\\
\hline
{\hyperref[\detokenize{api/mastml.feature_selectors.MASTMLFeatureSelector:mastml.feature_selectors.MASTMLFeatureSelector}]{\sphinxcrossref{\sphinxcode{\sphinxupquote{MASTMLFeatureSelector}}}}}(model, …{[}, cv, …{]})
&
Class custom-written for MAST-ML to conduct forward selection of features with flexible model and cv scheme
\\
\hline
{\hyperref[\detokenize{api/mastml.feature_selectors.NoSelect:mastml.feature_selectors.NoSelect}]{\sphinxcrossref{\sphinxcode{\sphinxupquote{NoSelect}}}}}()
&
Class for having a “null” transform where the output is the same as the input.
\\
\hline
{\hyperref[\detokenize{api/mastml.feature_selectors.PearsonSelector:mastml.feature_selectors.PearsonSelector}]{\sphinxcrossref{\sphinxcode{\sphinxupquote{PearsonSelector}}}}}(threshold\_between\_features, …)
&
Class custom-written for MAST-ML to conduct selection of features based on Pearson correlation coefficent between features and target.
\\
\hline
{\hyperref[\detokenize{api/mastml.feature_selectors.SklearnFeatureSelector:mastml.feature_selectors.SklearnFeatureSelector}]{\sphinxcrossref{\sphinxcode{\sphinxupquote{SklearnFeatureSelector}}}}}(selector, **kwargs)
&
Class that wraps scikit-learn feature selection methods with some new MAST-ML functionality
\\
\hline
\sphinxcode{\sphinxupquote{StandardScaler}}(*{[}, copy, with\_mean, with\_std{]})
&
Standardize features by removing the mean and scaling to unit variance
\\
\hline
\sphinxcode{\sphinxupquote{TransformerMixin}}
&
Mixin class for all transformers in scikit-learn.
\\
\hline
\sphinxcode{\sphinxupquote{datetime}}(year, month, day{[}, hour{[}, minute{[}, …)
&
The year, month and day arguments are required.
\\
\hline
\end{longtable}\sphinxatlongtableend\end{savenotes}


\subsubsection{BaseSelector}
\label{\detokenize{api/mastml.feature_selectors.BaseSelector:baseselector}}\label{\detokenize{api/mastml.feature_selectors.BaseSelector::doc}}\index{BaseSelector (class in mastml.feature\_selectors)@\spxentry{BaseSelector}\spxextra{class in mastml.feature\_selectors}}

\begin{fulllineitems}
\phantomsection\label{\detokenize{api/mastml.feature_selectors.BaseSelector:mastml.feature_selectors.BaseSelector}}\pysigline{\sphinxbfcode{\sphinxupquote{class }}\sphinxcode{\sphinxupquote{mastml.feature\_selectors.}}\sphinxbfcode{\sphinxupquote{BaseSelector}}}
Bases: \sphinxcode{\sphinxupquote{sklearn.base.BaseEstimator}}, \sphinxcode{\sphinxupquote{sklearn.base.TransformerMixin}}

Base class that forms foundation of MAST-ML feature selectors
\begin{description}
\item[{Args:}] \leavevmode
None. See individual selector types for input arguments

\item[{Methods:}] \leavevmode\begin{description}
\item[{fit: Does nothing, present for compatibility}] \leavevmode\begin{description}
\item[{Args:}] \leavevmode
X: (dataframe), dataframe of X features

y: (dataframe), dataframe of y data

\item[{Returns:}] \leavevmode
None

\end{description}

\item[{transform: Does nothing, present for compatibility}] \leavevmode\begin{description}
\item[{Args:}] \leavevmode
X: (dataframe), dataframe of X features

\item[{Returns:}] \leavevmode
X: (dataframe), dataframe of X features

\end{description}

\end{description}

evaluate: runs the fit and transform functions to select features, saves selector-specific files and saves
list of selected features
\begin{quote}
\begin{description}
\item[{Args:}] \leavevmode
X: (dataframe), dataframe of X features

y: (dataframe), dataframe of y data

savepath: (str), string denoting savepath to save selected features and associated files (if
applicable) to.

\end{description}

Returns:
\begin{quote}

X\_select (dataframe), dataframe of selected X features
\end{quote}
\end{quote}

\end{description}
\subsubsection*{Methods Summary}


\begin{savenotes}\sphinxatlongtablestart\begin{longtable}[c]{\X{1}{2}\X{1}{2}}
\hline

\endfirsthead

\multicolumn{2}{c}%
{\makebox[0pt]{\sphinxtablecontinued{\tablename\ \thetable{} -- continued from previous page}}}\\
\hline

\endhead

\hline
\multicolumn{2}{r}{\makebox[0pt][r]{\sphinxtablecontinued{Continued on next page}}}\\
\endfoot

\endlastfoot

{\hyperref[\detokenize{api/mastml.feature_selectors.BaseSelector:mastml.feature_selectors.BaseSelector.evaluate}]{\sphinxcrossref{\sphinxcode{\sphinxupquote{evaluate}}}}}(X, y{[}, savepath, make\_new\_dir{]})
&

\\
\hline
{\hyperref[\detokenize{api/mastml.feature_selectors.BaseSelector:mastml.feature_selectors.BaseSelector.fit}]{\sphinxcrossref{\sphinxcode{\sphinxupquote{fit}}}}}(X, y)
&

\\
\hline
{\hyperref[\detokenize{api/mastml.feature_selectors.BaseSelector:mastml.feature_selectors.BaseSelector.transform}]{\sphinxcrossref{\sphinxcode{\sphinxupquote{transform}}}}}(X)
&

\\
\hline
\end{longtable}\sphinxatlongtableend\end{savenotes}
\subsubsection*{Methods Documentation}
\index{evaluate() (mastml.feature\_selectors.BaseSelector method)@\spxentry{evaluate()}\spxextra{mastml.feature\_selectors.BaseSelector method}}

\begin{fulllineitems}
\phantomsection\label{\detokenize{api/mastml.feature_selectors.BaseSelector:mastml.feature_selectors.BaseSelector.evaluate}}\pysiglinewithargsret{\sphinxbfcode{\sphinxupquote{evaluate}}}{\emph{X}, \emph{y}, \emph{savepath=None}, \emph{make\_new\_dir=False}}{}
\end{fulllineitems}

\index{fit() (mastml.feature\_selectors.BaseSelector method)@\spxentry{fit()}\spxextra{mastml.feature\_selectors.BaseSelector method}}

\begin{fulllineitems}
\phantomsection\label{\detokenize{api/mastml.feature_selectors.BaseSelector:mastml.feature_selectors.BaseSelector.fit}}\pysiglinewithargsret{\sphinxbfcode{\sphinxupquote{fit}}}{\emph{X}, \emph{y}}{}
\end{fulllineitems}

\index{transform() (mastml.feature\_selectors.BaseSelector method)@\spxentry{transform()}\spxextra{mastml.feature\_selectors.BaseSelector method}}

\begin{fulllineitems}
\phantomsection\label{\detokenize{api/mastml.feature_selectors.BaseSelector:mastml.feature_selectors.BaseSelector.transform}}\pysiglinewithargsret{\sphinxbfcode{\sphinxupquote{transform}}}{\emph{X}}{}
\end{fulllineitems}


\end{fulllineitems}



\subsubsection{EnsembleModelFeatureSelector}
\label{\detokenize{api/mastml.feature_selectors.EnsembleModelFeatureSelector:ensemblemodelfeatureselector}}\label{\detokenize{api/mastml.feature_selectors.EnsembleModelFeatureSelector::doc}}\index{EnsembleModelFeatureSelector (class in mastml.feature\_selectors)@\spxentry{EnsembleModelFeatureSelector}\spxextra{class in mastml.feature\_selectors}}

\begin{fulllineitems}
\phantomsection\label{\detokenize{api/mastml.feature_selectors.EnsembleModelFeatureSelector:mastml.feature_selectors.EnsembleModelFeatureSelector}}\pysiglinewithargsret{\sphinxbfcode{\sphinxupquote{class }}\sphinxcode{\sphinxupquote{mastml.feature\_selectors.}}\sphinxbfcode{\sphinxupquote{EnsembleModelFeatureSelector}}}{\emph{model}, \emph{n\_features\_to\_select}, \emph{n\_random\_dummy=0}, \emph{n\_permuted\_dummy=0}}{}
Bases: {\hyperref[\detokenize{api/mastml.feature_selectors.BaseSelector:mastml.feature_selectors.BaseSelector}]{\sphinxcrossref{\sphinxcode{\sphinxupquote{mastml.feature\_selectors.BaseSelector}}}}}

Class custom-written for MAST-ML to conduct selection of features with ensemble model feature importances
\begin{description}
\item[{Args:}] \leavevmode
model: (mastml.models object), a MAST-ML compatable model

n\_features\_to\_select: (int), the number of features to select

n\_random\_dummy: (int), the number of random dummy variable to use. default is 0 if not used

n\_permuted\_dummy: (int), the number of permuted dummy variable to use. default is 0 if not used

\item[{Methods:}] \leavevmode\begin{description}
\item[{fit: performs feature selection}] \leavevmode\begin{description}
\item[{Args:}] \leavevmode
X: (dataframe), dataframe of X features

y: (dataframe), dataframe of y data

\item[{Returns:}] \leavevmode
None

\end{description}

\item[{transform: performs the transform to generate output of only selected features}] \leavevmode\begin{description}
\item[{Args:}] \leavevmode
X: (dataframe), dataframe of X features

\item[{Returns:}] \leavevmode
dataframe: (dataframe), dataframe of selected X features

\end{description}

\item[{create\_dummy\_variable: Inserts n\_dummy\_variable of dummy variables with the same standard deviation and mean of}] \leavevmode\begin{quote}

of the whole dataframe
\end{quote}
\begin{description}
\item[{Args:}] \leavevmode
X: (dataframe), dataframe of X features

\item[{Returns:}] \leavevmode
X: dataframe that includes dummy variables and scaled with standard scaler

\end{description}

\item[{check\_dummy\_ranking: If dummy variable is used, prints warning when number of features selected}] \leavevmode\begin{quote}

is not optimal (numbers of features selected ranks below the dummy variable)
\end{quote}
\begin{description}
\item[{Args:}] \leavevmode
feature\_importances\_sorted: list of features sorted based on their importances

\end{description}

\end{description}

\end{description}
\subsubsection*{Methods Summary}


\begin{savenotes}\sphinxatlongtablestart\begin{longtable}[c]{\X{1}{2}\X{1}{2}}
\hline

\endfirsthead

\multicolumn{2}{c}%
{\makebox[0pt]{\sphinxtablecontinued{\tablename\ \thetable{} -- continued from previous page}}}\\
\hline

\endhead

\hline
\multicolumn{2}{r}{\makebox[0pt][r]{\sphinxtablecontinued{Continued on next page}}}\\
\endfoot

\endlastfoot

{\hyperref[\detokenize{api/mastml.feature_selectors.EnsembleModelFeatureSelector:mastml.feature_selectors.EnsembleModelFeatureSelector.check_dummy_ranking}]{\sphinxcrossref{\sphinxcode{\sphinxupquote{check\_dummy\_ranking}}}}}(feature\_importances\_sorted)
&

\\
\hline
{\hyperref[\detokenize{api/mastml.feature_selectors.EnsembleModelFeatureSelector:mastml.feature_selectors.EnsembleModelFeatureSelector.create_dummy_variable}]{\sphinxcrossref{\sphinxcode{\sphinxupquote{create\_dummy\_variable}}}}}(X)
&

\\
\hline
{\hyperref[\detokenize{api/mastml.feature_selectors.EnsembleModelFeatureSelector:mastml.feature_selectors.EnsembleModelFeatureSelector.fit}]{\sphinxcrossref{\sphinxcode{\sphinxupquote{fit}}}}}(X, y)
&

\\
\hline
{\hyperref[\detokenize{api/mastml.feature_selectors.EnsembleModelFeatureSelector:mastml.feature_selectors.EnsembleModelFeatureSelector.transform}]{\sphinxcrossref{\sphinxcode{\sphinxupquote{transform}}}}}(X)
&

\\
\hline
\end{longtable}\sphinxatlongtableend\end{savenotes}
\subsubsection*{Methods Documentation}
\index{check\_dummy\_ranking() (mastml.feature\_selectors.EnsembleModelFeatureSelector method)@\spxentry{check\_dummy\_ranking()}\spxextra{mastml.feature\_selectors.EnsembleModelFeatureSelector method}}

\begin{fulllineitems}
\phantomsection\label{\detokenize{api/mastml.feature_selectors.EnsembleModelFeatureSelector:mastml.feature_selectors.EnsembleModelFeatureSelector.check_dummy_ranking}}\pysiglinewithargsret{\sphinxbfcode{\sphinxupquote{check\_dummy\_ranking}}}{\emph{feature\_importances\_sorted}}{}
\end{fulllineitems}

\index{create\_dummy\_variable() (mastml.feature\_selectors.EnsembleModelFeatureSelector method)@\spxentry{create\_dummy\_variable()}\spxextra{mastml.feature\_selectors.EnsembleModelFeatureSelector method}}

\begin{fulllineitems}
\phantomsection\label{\detokenize{api/mastml.feature_selectors.EnsembleModelFeatureSelector:mastml.feature_selectors.EnsembleModelFeatureSelector.create_dummy_variable}}\pysiglinewithargsret{\sphinxbfcode{\sphinxupquote{create\_dummy\_variable}}}{\emph{X}}{}
\end{fulllineitems}

\index{fit() (mastml.feature\_selectors.EnsembleModelFeatureSelector method)@\spxentry{fit()}\spxextra{mastml.feature\_selectors.EnsembleModelFeatureSelector method}}

\begin{fulllineitems}
\phantomsection\label{\detokenize{api/mastml.feature_selectors.EnsembleModelFeatureSelector:mastml.feature_selectors.EnsembleModelFeatureSelector.fit}}\pysiglinewithargsret{\sphinxbfcode{\sphinxupquote{fit}}}{\emph{X}, \emph{y}}{}
\end{fulllineitems}

\index{transform() (mastml.feature\_selectors.EnsembleModelFeatureSelector method)@\spxentry{transform()}\spxextra{mastml.feature\_selectors.EnsembleModelFeatureSelector method}}

\begin{fulllineitems}
\phantomsection\label{\detokenize{api/mastml.feature_selectors.EnsembleModelFeatureSelector:mastml.feature_selectors.EnsembleModelFeatureSelector.transform}}\pysiglinewithargsret{\sphinxbfcode{\sphinxupquote{transform}}}{\emph{X}}{}
\end{fulllineitems}


\end{fulllineitems}



\subsubsection{MASTMLFeatureSelector}
\label{\detokenize{api/mastml.feature_selectors.MASTMLFeatureSelector:mastmlfeatureselector}}\label{\detokenize{api/mastml.feature_selectors.MASTMLFeatureSelector::doc}}\index{MASTMLFeatureSelector (class in mastml.feature\_selectors)@\spxentry{MASTMLFeatureSelector}\spxextra{class in mastml.feature\_selectors}}

\begin{fulllineitems}
\phantomsection\label{\detokenize{api/mastml.feature_selectors.MASTMLFeatureSelector:mastml.feature_selectors.MASTMLFeatureSelector}}\pysiglinewithargsret{\sphinxbfcode{\sphinxupquote{class }}\sphinxcode{\sphinxupquote{mastml.feature\_selectors.}}\sphinxbfcode{\sphinxupquote{MASTMLFeatureSelector}}}{\emph{model}, \emph{n\_features\_to\_select}, \emph{cv=None}, \emph{manually\_selected\_features={[}{]}}}{}
Bases: {\hyperref[\detokenize{api/mastml.feature_selectors.BaseSelector:mastml.feature_selectors.BaseSelector}]{\sphinxcrossref{\sphinxcode{\sphinxupquote{mastml.feature\_selectors.BaseSelector}}}}}

Class custom-written for MAST-ML to conduct forward selection of features with flexible model and cv scheme
\begin{description}
\item[{Args:}] \leavevmode
estimator: (scikit-learn model/estimator object), a scikit-learn model/estimator

n\_features\_to\_select: (int), the number of features to select

cv: (scikit-learn cross-validation object), a scikit-learn cross-validation object

manually\_selected\_features: (list), a list of features manually set by the user. The feature selector will
first start from this list of features and sequentially add features until n\_features\_to\_select is met.

\item[{Methods:}] \leavevmode\begin{description}
\item[{fit: performs feature selection}] \leavevmode\begin{description}
\item[{Args:}] \leavevmode
X: (dataframe), dataframe of X features

y: (dataframe), dataframe of y data

Xgroups: (dataframe), dataframe of group labels

\item[{Returns:}] \leavevmode
None

\end{description}

\item[{transform: performs the transform to generate output of only selected features}] \leavevmode\begin{description}
\item[{Args:}] \leavevmode
X: (dataframe), dataframe of X features

\item[{Returns:}] \leavevmode
dataframe: (dataframe), dataframe of selected X features

\end{description}

\end{description}

\end{description}
\subsubsection*{Methods Summary}


\begin{savenotes}\sphinxatlongtablestart\begin{longtable}[c]{\X{1}{2}\X{1}{2}}
\hline

\endfirsthead

\multicolumn{2}{c}%
{\makebox[0pt]{\sphinxtablecontinued{\tablename\ \thetable{} -- continued from previous page}}}\\
\hline

\endhead

\hline
\multicolumn{2}{r}{\makebox[0pt][r]{\sphinxtablecontinued{Continued on next page}}}\\
\endfoot

\endlastfoot

{\hyperref[\detokenize{api/mastml.feature_selectors.MASTMLFeatureSelector:mastml.feature_selectors.MASTMLFeatureSelector.fit}]{\sphinxcrossref{\sphinxcode{\sphinxupquote{fit}}}}}(X, y{[}, Xgroups{]})
&

\\
\hline
{\hyperref[\detokenize{api/mastml.feature_selectors.MASTMLFeatureSelector:mastml.feature_selectors.MASTMLFeatureSelector.transform}]{\sphinxcrossref{\sphinxcode{\sphinxupquote{transform}}}}}(X)
&

\\
\hline
\end{longtable}\sphinxatlongtableend\end{savenotes}
\subsubsection*{Methods Documentation}
\index{fit() (mastml.feature\_selectors.MASTMLFeatureSelector method)@\spxentry{fit()}\spxextra{mastml.feature\_selectors.MASTMLFeatureSelector method}}

\begin{fulllineitems}
\phantomsection\label{\detokenize{api/mastml.feature_selectors.MASTMLFeatureSelector:mastml.feature_selectors.MASTMLFeatureSelector.fit}}\pysiglinewithargsret{\sphinxbfcode{\sphinxupquote{fit}}}{\emph{X}, \emph{y}, \emph{Xgroups=None}}{}
\end{fulllineitems}

\index{transform() (mastml.feature\_selectors.MASTMLFeatureSelector method)@\spxentry{transform()}\spxextra{mastml.feature\_selectors.MASTMLFeatureSelector method}}

\begin{fulllineitems}
\phantomsection\label{\detokenize{api/mastml.feature_selectors.MASTMLFeatureSelector:mastml.feature_selectors.MASTMLFeatureSelector.transform}}\pysiglinewithargsret{\sphinxbfcode{\sphinxupquote{transform}}}{\emph{X}}{}
\end{fulllineitems}


\end{fulllineitems}



\subsubsection{NoSelect}
\label{\detokenize{api/mastml.feature_selectors.NoSelect:noselect}}\label{\detokenize{api/mastml.feature_selectors.NoSelect::doc}}\index{NoSelect (class in mastml.feature\_selectors)@\spxentry{NoSelect}\spxextra{class in mastml.feature\_selectors}}

\begin{fulllineitems}
\phantomsection\label{\detokenize{api/mastml.feature_selectors.NoSelect:mastml.feature_selectors.NoSelect}}\pysigline{\sphinxbfcode{\sphinxupquote{class }}\sphinxcode{\sphinxupquote{mastml.feature\_selectors.}}\sphinxbfcode{\sphinxupquote{NoSelect}}}
Bases: {\hyperref[\detokenize{api/mastml.feature_selectors.BaseSelector:mastml.feature_selectors.BaseSelector}]{\sphinxcrossref{\sphinxcode{\sphinxupquote{mastml.feature\_selectors.BaseSelector}}}}}

Class for having a “null” transform where the output is the same as the input. Needed by MAST-ML as a placeholder if
certain workflow aspects are not performed.

See BaseSelector for information on args and methods

\end{fulllineitems}



\subsubsection{PearsonSelector}
\label{\detokenize{api/mastml.feature_selectors.PearsonSelector:pearsonselector}}\label{\detokenize{api/mastml.feature_selectors.PearsonSelector::doc}}\index{PearsonSelector (class in mastml.feature\_selectors)@\spxentry{PearsonSelector}\spxextra{class in mastml.feature\_selectors}}

\begin{fulllineitems}
\phantomsection\label{\detokenize{api/mastml.feature_selectors.PearsonSelector:mastml.feature_selectors.PearsonSelector}}\pysiglinewithargsret{\sphinxbfcode{\sphinxupquote{class }}\sphinxcode{\sphinxupquote{mastml.feature\_selectors.}}\sphinxbfcode{\sphinxupquote{PearsonSelector}}}{\emph{threshold\_between\_features}, \emph{threshold\_with\_target}, \emph{flag\_highly\_correlated\_features}, \emph{n\_features\_to\_select}}{}
Bases: {\hyperref[\detokenize{api/mastml.feature_selectors.BaseSelector:mastml.feature_selectors.BaseSelector}]{\sphinxcrossref{\sphinxcode{\sphinxupquote{mastml.feature\_selectors.BaseSelector}}}}}

Class custom-written for MAST-ML to conduct selection of features based on Pearson correlation coefficent between
features and target. Can also be used for dimensionality reduction by removing redundant features highly correlated
with each other.
\begin{description}
\item[{Args:}] \leavevmode
threshold\_between\_features: (float), the threshold to decide whether redundant features are removed. Should
be a decimal value between 0 and 1. Only used if remove\_highly\_correlated\_features is True

threshold\_with\_target: (float), the threshold to decide whether a given feature is sufficiently correlated
with the target feature and thus kept as a selected feature. Should be a decimal value between 0 and 1.

remove\_highly\_correlated\_features: (bool), whether to remove features highly correlated with each other

n\_features\_to\_select: (int), the number of features to select

\item[{Methods:}] \leavevmode\begin{description}
\item[{fit: performs feature selection}] \leavevmode\begin{description}
\item[{Args:}] \leavevmode
X: (dataframe), dataframe of X features

y: (dataframe), dataframe of y data

\item[{Returns:}] \leavevmode
None

\end{description}

\item[{transform: performs the transform to generate output of only selected features}] \leavevmode\begin{description}
\item[{Args:}] \leavevmode
X: (dataframe), dataframe of X features

\item[{Returns:}] \leavevmode
dataframe: (dataframe), dataframe of selected X features

\end{description}

\end{description}

\end{description}
\subsubsection*{Methods Summary}


\begin{savenotes}\sphinxatlongtablestart\begin{longtable}[c]{\X{1}{2}\X{1}{2}}
\hline

\endfirsthead

\multicolumn{2}{c}%
{\makebox[0pt]{\sphinxtablecontinued{\tablename\ \thetable{} -- continued from previous page}}}\\
\hline

\endhead

\hline
\multicolumn{2}{r}{\makebox[0pt][r]{\sphinxtablecontinued{Continued on next page}}}\\
\endfoot

\endlastfoot

{\hyperref[\detokenize{api/mastml.feature_selectors.PearsonSelector:mastml.feature_selectors.PearsonSelector.fit}]{\sphinxcrossref{\sphinxcode{\sphinxupquote{fit}}}}}(X, y)
&

\\
\hline
{\hyperref[\detokenize{api/mastml.feature_selectors.PearsonSelector:mastml.feature_selectors.PearsonSelector.transform}]{\sphinxcrossref{\sphinxcode{\sphinxupquote{transform}}}}}(X)
&

\\
\hline
\end{longtable}\sphinxatlongtableend\end{savenotes}
\subsubsection*{Methods Documentation}
\index{fit() (mastml.feature\_selectors.PearsonSelector method)@\spxentry{fit()}\spxextra{mastml.feature\_selectors.PearsonSelector method}}

\begin{fulllineitems}
\phantomsection\label{\detokenize{api/mastml.feature_selectors.PearsonSelector:mastml.feature_selectors.PearsonSelector.fit}}\pysiglinewithargsret{\sphinxbfcode{\sphinxupquote{fit}}}{\emph{X}, \emph{y}}{}
\end{fulllineitems}

\index{transform() (mastml.feature\_selectors.PearsonSelector method)@\spxentry{transform()}\spxextra{mastml.feature\_selectors.PearsonSelector method}}

\begin{fulllineitems}
\phantomsection\label{\detokenize{api/mastml.feature_selectors.PearsonSelector:mastml.feature_selectors.PearsonSelector.transform}}\pysiglinewithargsret{\sphinxbfcode{\sphinxupquote{transform}}}{\emph{X}}{}
\end{fulllineitems}


\end{fulllineitems}



\subsubsection{SklearnFeatureSelector}
\label{\detokenize{api/mastml.feature_selectors.SklearnFeatureSelector:sklearnfeatureselector}}\label{\detokenize{api/mastml.feature_selectors.SklearnFeatureSelector::doc}}\index{SklearnFeatureSelector (class in mastml.feature\_selectors)@\spxentry{SklearnFeatureSelector}\spxextra{class in mastml.feature\_selectors}}

\begin{fulllineitems}
\phantomsection\label{\detokenize{api/mastml.feature_selectors.SklearnFeatureSelector:mastml.feature_selectors.SklearnFeatureSelector}}\pysiglinewithargsret{\sphinxbfcode{\sphinxupquote{class }}\sphinxcode{\sphinxupquote{mastml.feature\_selectors.}}\sphinxbfcode{\sphinxupquote{SklearnFeatureSelector}}}{\emph{selector}, \emph{**kwargs}}{}
Bases: {\hyperref[\detokenize{api/mastml.feature_selectors.BaseSelector:mastml.feature_selectors.BaseSelector}]{\sphinxcrossref{\sphinxcode{\sphinxupquote{mastml.feature\_selectors.BaseSelector}}}}}

Class that wraps scikit-learn feature selection methods with some new MAST-ML functionality
\begin{description}
\item[{Args:}] \leavevmode
selector (str) : a string denoting the name of a sklearn.feature\_selection object

{\color{red}\bfseries{}**}kwargs: the key word arguments of the designated sklearn.feature\_selection object

\item[{Methods:}] \leavevmode\begin{description}
\item[{fit: performs feature selection}] \leavevmode\begin{description}
\item[{Args:}] \leavevmode
X: (dataframe), dataframe of X features

y: (dataframe), dataframe of y data

\item[{Returns:}] \leavevmode
None

\end{description}

\item[{transform: performs the transform to generate output of only selected features}] \leavevmode\begin{description}
\item[{Args:}] \leavevmode
X: (dataframe), dataframe of X features

\item[{Returns:}] \leavevmode
X\_select: (dataframe), dataframe of selected X features

\end{description}

\end{description}

\end{description}
\subsubsection*{Methods Summary}


\begin{savenotes}\sphinxatlongtablestart\begin{longtable}[c]{\X{1}{2}\X{1}{2}}
\hline

\endfirsthead

\multicolumn{2}{c}%
{\makebox[0pt]{\sphinxtablecontinued{\tablename\ \thetable{} -- continued from previous page}}}\\
\hline

\endhead

\hline
\multicolumn{2}{r}{\makebox[0pt][r]{\sphinxtablecontinued{Continued on next page}}}\\
\endfoot

\endlastfoot

{\hyperref[\detokenize{api/mastml.feature_selectors.SklearnFeatureSelector:mastml.feature_selectors.SklearnFeatureSelector.fit}]{\sphinxcrossref{\sphinxcode{\sphinxupquote{fit}}}}}(X, y)
&

\\
\hline
{\hyperref[\detokenize{api/mastml.feature_selectors.SklearnFeatureSelector:mastml.feature_selectors.SklearnFeatureSelector.transform}]{\sphinxcrossref{\sphinxcode{\sphinxupquote{transform}}}}}(X)
&

\\
\hline
\end{longtable}\sphinxatlongtableend\end{savenotes}
\subsubsection*{Methods Documentation}
\index{fit() (mastml.feature\_selectors.SklearnFeatureSelector method)@\spxentry{fit()}\spxextra{mastml.feature\_selectors.SklearnFeatureSelector method}}

\begin{fulllineitems}
\phantomsection\label{\detokenize{api/mastml.feature_selectors.SklearnFeatureSelector:mastml.feature_selectors.SklearnFeatureSelector.fit}}\pysiglinewithargsret{\sphinxbfcode{\sphinxupquote{fit}}}{\emph{X}, \emph{y}}{}
\end{fulllineitems}

\index{transform() (mastml.feature\_selectors.SklearnFeatureSelector method)@\spxentry{transform()}\spxextra{mastml.feature\_selectors.SklearnFeatureSelector method}}

\begin{fulllineitems}
\phantomsection\label{\detokenize{api/mastml.feature_selectors.SklearnFeatureSelector:mastml.feature_selectors.SklearnFeatureSelector.transform}}\pysiglinewithargsret{\sphinxbfcode{\sphinxupquote{transform}}}{\emph{X}}{}
\end{fulllineitems}


\end{fulllineitems}



\subsection{Class Inheritance Diagram}
\label{\detokenize{6_feature_selectors:class-inheritance-diagram}}
\sphinxincludegraphics[]{None}


\chapter{Code Documentation: Hyperparameter Optimization}
\label{\detokenize{7_hyper_opt:code-documentation-hyperparameter-optimization}}\label{\detokenize{7_hyper_opt::doc}}

\section{mastml.hyper\_opt Module}
\label{\detokenize{7_hyper_opt:module-mastml.hyper_opt}}\label{\detokenize{7_hyper_opt:mastml-hyper-opt-module}}\index{mastml.hyper\_opt (module)@\spxentry{mastml.hyper\_opt}\spxextra{module}}
This module contains methods for optimizing hyperparameters of models
\begin{description}
\item[{HyperOptUtils:}] \leavevmode
This class contains various helper utilities for setting up and running hyperparameter optimization

\item[{GridSearch:}] \leavevmode
This class performs a basic grid search over the parameters and value ranges of interest to find the best
set of model hyperparameters in the provided grid of values

\item[{RandomizedSearch:}] \leavevmode
This class performs a randomized search over the parameters and value ranges of interest to find the best
set of model hyperparameters in the provided grid of values. Often faster than GridSearch. Instead of a grid
of values, it takes a probability distribution name as input (e.g. “norm”)

\item[{BayesianSearch:}] \leavevmode
This class performs a Bayesian search over the parameters and value ranges of interest to find the best
set of model hyperparameters in the provided grid of values. Often faster than GridSearch.

\end{description}


\subsection{Classes}
\label{\detokenize{7_hyper_opt:classes}}

\begin{savenotes}\sphinxatlongtablestart\begin{longtable}[c]{\X{1}{2}\X{1}{2}}
\hline

\endfirsthead

\multicolumn{2}{c}%
{\makebox[0pt]{\sphinxtablecontinued{\tablename\ \thetable{} -- continued from previous page}}}\\
\hline

\endhead

\hline
\multicolumn{2}{r}{\makebox[0pt][r]{\sphinxtablecontinued{Continued on next page}}}\\
\endfoot

\endlastfoot

\sphinxcode{\sphinxupquote{BayesSearchCV}}(estimator, search\_spaces{[}, …{]})
&
Bayesian optimization over hyper parameters.
\\
\hline
{\hyperref[\detokenize{api/mastml.hyper_opt.BayesianSearch:mastml.hyper_opt.BayesianSearch}]{\sphinxcrossref{\sphinxcode{\sphinxupquote{BayesianSearch}}}}}(param\_names, param\_values{[}, …{]})
&
Class to conduct a Bayesian search to find optimized model hyperparameter values
\\
\hline
\sphinxcode{\sphinxupquote{Categorical}}(categories{[}, prior, transform, name{]})
&
Search space dimension that can take on categorical values.
\\
\hline
{\hyperref[\detokenize{api/mastml.hyper_opt.GridSearch:mastml.hyper_opt.GridSearch}]{\sphinxcrossref{\sphinxcode{\sphinxupquote{GridSearch}}}}}(param\_names, param\_values{[}, …{]})
&
Class to conduct a grid search to find optimized model hyperparameter values
\\
\hline
\sphinxcode{\sphinxupquote{GridSearchCV}}(estimator, param\_grid, *{[}, …{]})
&
Exhaustive search over specified parameter values for an estimator.
\\
\hline
{\hyperref[\detokenize{api/mastml.hyper_opt.HyperOptUtils:mastml.hyper_opt.HyperOptUtils}]{\sphinxcrossref{\sphinxcode{\sphinxupquote{HyperOptUtils}}}}}(param\_names, param\_values)
&
Helper class providing useful methods for other hyperparameter optimization classes.
\\
\hline
\sphinxcode{\sphinxupquote{Integer}}(low, high{[}, prior, base, transform, …{]})
&
Search space dimension that can take on integer values.
\\
\hline
\sphinxcode{\sphinxupquote{Metrics}}(metrics\_list{[}, metrics\_type{]})
&
Class containing access to a wide range of metrics from scikit-learn and a number of MAST-ML custom-written metrics
\\
\hline
{\hyperref[\detokenize{api/mastml.hyper_opt.RandomizedSearch:mastml.hyper_opt.RandomizedSearch}]{\sphinxcrossref{\sphinxcode{\sphinxupquote{RandomizedSearch}}}}}(param\_names, param\_values)
&
Class to conduct a randomized search to find optimized model hyperparameter values
\\
\hline
\sphinxcode{\sphinxupquote{RandomizedSearchCV}}(estimator, …{[}, n\_iter, …{]})
&
Randomized search on hyper parameters.
\\
\hline
\sphinxcode{\sphinxupquote{Real}}(low, high{[}, prior, base, transform, …{]})
&
Search space dimension that can take on any real value.
\\
\hline
\sphinxcode{\sphinxupquote{SklearnModel}}(model, **kwargs)
&
Class to wrap any sklearn estimator, and provide some new dataframe functionality
\\
\hline
\end{longtable}\sphinxatlongtableend\end{savenotes}


\subsubsection{BayesianSearch}
\label{\detokenize{api/mastml.hyper_opt.BayesianSearch:bayesiansearch}}\label{\detokenize{api/mastml.hyper_opt.BayesianSearch::doc}}\index{BayesianSearch (class in mastml.hyper\_opt)@\spxentry{BayesianSearch}\spxextra{class in mastml.hyper\_opt}}

\begin{fulllineitems}
\phantomsection\label{\detokenize{api/mastml.hyper_opt.BayesianSearch:mastml.hyper_opt.BayesianSearch}}\pysiglinewithargsret{\sphinxbfcode{\sphinxupquote{class }}\sphinxcode{\sphinxupquote{mastml.hyper\_opt.}}\sphinxbfcode{\sphinxupquote{BayesianSearch}}}{\emph{param\_names}, \emph{param\_values}, \emph{scoring=None}, \emph{n\_iter=50}, \emph{n\_jobs=1}}{}
Bases: {\hyperref[\detokenize{api/mastml.hyper_opt.HyperOptUtils:mastml.hyper_opt.HyperOptUtils}]{\sphinxcrossref{\sphinxcode{\sphinxupquote{mastml.hyper\_opt.HyperOptUtils}}}}}

Class to conduct a Bayesian search to find optimized model hyperparameter values
\begin{description}
\item[{Args:}] \leavevmode
param\_names: (list), list containing names of hyperparams to optimize

param\_values: (list), list containing values of hyperparams to optimize

scoring: (str), string denoting name of regression metric to evaluate learning curves. See mastml.metrics.Metrics.\_metric\_zoo for full list

n\_iter: (int), number denoting the number of evaluations in the search space to perform. Higher numbers will take longer but will be more accurate

n\_jobs: (int), number of jobs to run in parallel. Can speed up calculation when using multiple cores

\item[{Methods:}] \leavevmode\begin{description}
\item[{fit}] \leavevmode{[}optimizes hyperparameters{]}\begin{description}
\item[{Args:}] \leavevmode
X: (pd.DataFrame), dataframe of X feature data

y: (pd.Series), series of target y data

model: (mastml.models object), a MAST-ML model, e.g. SklearnModel or EnsembleModel

cv: (scikit-learn cross-validation object), a scikit-learn cross-validation object

savepath: (str), path of output directory

\item[{Returns:}] \leavevmode
best\_estimator (mastml.models object) : the optimized MAST-ML model

\end{description}

\end{description}

\end{description}
\subsubsection*{Methods Summary}


\begin{savenotes}\sphinxatlongtablestart\begin{longtable}[c]{\X{1}{2}\X{1}{2}}
\hline

\endfirsthead

\multicolumn{2}{c}%
{\makebox[0pt]{\sphinxtablecontinued{\tablename\ \thetable{} -- continued from previous page}}}\\
\hline

\endhead

\hline
\multicolumn{2}{r}{\makebox[0pt][r]{\sphinxtablecontinued{Continued on next page}}}\\
\endfoot

\endlastfoot

{\hyperref[\detokenize{api/mastml.hyper_opt.BayesianSearch:mastml.hyper_opt.BayesianSearch.fit}]{\sphinxcrossref{\sphinxcode{\sphinxupquote{fit}}}}}(X, y, model, cv{[}, savepath{]})
&

\\
\hline
\end{longtable}\sphinxatlongtableend\end{savenotes}
\subsubsection*{Methods Documentation}
\index{fit() (mastml.hyper\_opt.BayesianSearch method)@\spxentry{fit()}\spxextra{mastml.hyper\_opt.BayesianSearch method}}

\begin{fulllineitems}
\phantomsection\label{\detokenize{api/mastml.hyper_opt.BayesianSearch:mastml.hyper_opt.BayesianSearch.fit}}\pysiglinewithargsret{\sphinxbfcode{\sphinxupquote{fit}}}{\emph{X}, \emph{y}, \emph{model}, \emph{cv}, \emph{savepath=None}}{}
\end{fulllineitems}


\end{fulllineitems}



\subsubsection{GridSearch}
\label{\detokenize{api/mastml.hyper_opt.GridSearch:gridsearch}}\label{\detokenize{api/mastml.hyper_opt.GridSearch::doc}}\index{GridSearch (class in mastml.hyper\_opt)@\spxentry{GridSearch}\spxextra{class in mastml.hyper\_opt}}

\begin{fulllineitems}
\phantomsection\label{\detokenize{api/mastml.hyper_opt.GridSearch:mastml.hyper_opt.GridSearch}}\pysiglinewithargsret{\sphinxbfcode{\sphinxupquote{class }}\sphinxcode{\sphinxupquote{mastml.hyper\_opt.}}\sphinxbfcode{\sphinxupquote{GridSearch}}}{\emph{param\_names}, \emph{param\_values}, \emph{scoring=None}, \emph{n\_jobs=1}}{}
Bases: {\hyperref[\detokenize{api/mastml.hyper_opt.HyperOptUtils:mastml.hyper_opt.HyperOptUtils}]{\sphinxcrossref{\sphinxcode{\sphinxupquote{mastml.hyper\_opt.HyperOptUtils}}}}}

Class to conduct a grid search to find optimized model hyperparameter values
\begin{description}
\item[{Args:}] \leavevmode
param\_names: (list), list containing names of hyperparams to optimize

param\_values: (list), list containing values of hyperparams to optimize

scoring: (str), string denoting name of regression metric to evaluate learning curves. See mastml.metrics.Metrics.\_metric\_zoo for full list

n\_jobs: (int), number of jobs to run in parallel. Can speed up calculation when using multiple cores

\item[{Methods:}] \leavevmode\begin{description}
\item[{fit}] \leavevmode{[}optimizes hyperparameters{]}\begin{description}
\item[{Args:}] \leavevmode
X: (pd.DataFrame), dataframe of X feature data

y: (pd.Series), series of target y data

model: (mastml.models object), a MAST-ML model, e.g. SklearnModel or EnsembleModel

cv: (scikit-learn cross-validation object), a scikit-learn cross-validation object

savepath: (str), path of output directory

\item[{Returns:}] \leavevmode
best\_estimator (mastml.models object) : the optimized MAST-ML model

\end{description}

\end{description}

\end{description}
\subsubsection*{Methods Summary}


\begin{savenotes}\sphinxatlongtablestart\begin{longtable}[c]{\X{1}{2}\X{1}{2}}
\hline

\endfirsthead

\multicolumn{2}{c}%
{\makebox[0pt]{\sphinxtablecontinued{\tablename\ \thetable{} -- continued from previous page}}}\\
\hline

\endhead

\hline
\multicolumn{2}{r}{\makebox[0pt][r]{\sphinxtablecontinued{Continued on next page}}}\\
\endfoot

\endlastfoot

{\hyperref[\detokenize{api/mastml.hyper_opt.GridSearch:mastml.hyper_opt.GridSearch.fit}]{\sphinxcrossref{\sphinxcode{\sphinxupquote{fit}}}}}(X, y, model{[}, cv, savepath{]})
&

\\
\hline
\end{longtable}\sphinxatlongtableend\end{savenotes}
\subsubsection*{Methods Documentation}
\index{fit() (mastml.hyper\_opt.GridSearch method)@\spxentry{fit()}\spxextra{mastml.hyper\_opt.GridSearch method}}

\begin{fulllineitems}
\phantomsection\label{\detokenize{api/mastml.hyper_opt.GridSearch:mastml.hyper_opt.GridSearch.fit}}\pysiglinewithargsret{\sphinxbfcode{\sphinxupquote{fit}}}{\emph{X}, \emph{y}, \emph{model}, \emph{cv=None}, \emph{savepath=None}}{}
\end{fulllineitems}


\end{fulllineitems}



\subsubsection{HyperOptUtils}
\label{\detokenize{api/mastml.hyper_opt.HyperOptUtils:hyperoptutils}}\label{\detokenize{api/mastml.hyper_opt.HyperOptUtils::doc}}\index{HyperOptUtils (class in mastml.hyper\_opt)@\spxentry{HyperOptUtils}\spxextra{class in mastml.hyper\_opt}}

\begin{fulllineitems}
\phantomsection\label{\detokenize{api/mastml.hyper_opt.HyperOptUtils:mastml.hyper_opt.HyperOptUtils}}\pysiglinewithargsret{\sphinxbfcode{\sphinxupquote{class }}\sphinxcode{\sphinxupquote{mastml.hyper\_opt.}}\sphinxbfcode{\sphinxupquote{HyperOptUtils}}}{\emph{param\_names}, \emph{param\_values}}{}
Bases: \sphinxcode{\sphinxupquote{object}}

Helper class providing useful methods for other hyperparameter optimization classes.
\begin{description}
\item[{Args:}] \leavevmode
param\_names: (list), list containing names of hyperparams to optimize

param\_values: (list), list containing values of hyperparams to optimize

\item[{Methods:}] \leavevmode\begin{description}
\item[{\_search\_space\_generator}] \leavevmode{[}parses GridSearch param\_dict and checks values{]}\begin{description}
\item[{Args:}] \leavevmode
params: (dict), dict of \{param\_name : param\_value\} pairs.

\item[{Returns:}] \leavevmode
{\color{red}\bfseries{}params\_}: (dict), dict of \{param\_name : param\_value\} pairs.

\end{description}

\item[{\_save\_output}] \leavevmode{[}saves hyperparameter optimization output and best values to csv file{]}\begin{description}
\item[{Args:}] \leavevmode
savepath: (str), path of output directory

data: (dict), dict of \{estimator\_name : hyper\_opt.GridSearch.fit()\} object, or equivalent

\item[{Returns:}] \leavevmode
None

\end{description}

\item[{\_get\_grid\_param\_dict}] \leavevmode{[}configures the param\_dict for GridSearch{]}\begin{description}
\item[{Args:}] \leavevmode
None

\item[{Returns:}] \leavevmode
param\_dict: (dict), dict of \{param\_name : param\_value\} pairs.

\end{description}

\item[{\_get\_randomized\_param\_dict}] \leavevmode{[}configures the param\_dict for RandomSearch{]}\begin{description}
\item[{Args:}] \leavevmode
None

\item[{Returns:}] \leavevmode
param\_dict: (dict), dict of \{param\_name : param\_value\} pairs.

\end{description}

\item[{\_get\_bayesian\_param\_dict}] \leavevmode{[}configures the param\_dict for BayesianSearch{]}\begin{description}
\item[{Args:}] \leavevmode
None

\item[{Returns:}] \leavevmode
param\_dict: (dict), dict of \{param\_name : param\_value\} pairs.

\end{description}

\end{description}

\end{description}

\end{fulllineitems}



\subsubsection{RandomizedSearch}
\label{\detokenize{api/mastml.hyper_opt.RandomizedSearch:randomizedsearch}}\label{\detokenize{api/mastml.hyper_opt.RandomizedSearch::doc}}\index{RandomizedSearch (class in mastml.hyper\_opt)@\spxentry{RandomizedSearch}\spxextra{class in mastml.hyper\_opt}}

\begin{fulllineitems}
\phantomsection\label{\detokenize{api/mastml.hyper_opt.RandomizedSearch:mastml.hyper_opt.RandomizedSearch}}\pysiglinewithargsret{\sphinxbfcode{\sphinxupquote{class }}\sphinxcode{\sphinxupquote{mastml.hyper\_opt.}}\sphinxbfcode{\sphinxupquote{RandomizedSearch}}}{\emph{param\_names}, \emph{param\_values}, \emph{scoring=None}, \emph{n\_iter=50}, \emph{n\_jobs=1}}{}
Bases: {\hyperref[\detokenize{api/mastml.hyper_opt.HyperOptUtils:mastml.hyper_opt.HyperOptUtils}]{\sphinxcrossref{\sphinxcode{\sphinxupquote{mastml.hyper\_opt.HyperOptUtils}}}}}

Class to conduct a randomized search to find optimized model hyperparameter values
\begin{description}
\item[{Args:}] \leavevmode
param\_names: (list), list containing names of hyperparams to optimize

param\_values: (list), list containing values of hyperparams to optimize

scoring: (str), string denoting name of regression metric to evaluate learning curves. See mastml.metrics.Metrics.\_metric\_zoo for full list

n\_iter: (int), number denoting the number of evaluations in the search space to perform. Higher numbers will take longer but will be more accurate

n\_jobs: (int), number of jobs to run in parallel. Can speed up calculation when using multiple cores

\item[{Methods:}] \leavevmode\begin{description}
\item[{fit}] \leavevmode{[}optimizes hyperparameters{]}\begin{description}
\item[{Args:}] \leavevmode
X: (pd.DataFrame), dataframe of X feature data

y: (pd.Series), series of target y data

model: (mastml.models object), a MAST-ML model, e.g. SklearnModel or EnsembleModel

cv: (scikit-learn cross-validation object), a scikit-learn cross-validation object

savepath: (str), path of output directory

\item[{Returns:}] \leavevmode
best\_estimator (mastml.models object) : the optimized MAST-ML model

\end{description}

\end{description}

\end{description}
\subsubsection*{Methods Summary}


\begin{savenotes}\sphinxatlongtablestart\begin{longtable}[c]{\X{1}{2}\X{1}{2}}
\hline

\endfirsthead

\multicolumn{2}{c}%
{\makebox[0pt]{\sphinxtablecontinued{\tablename\ \thetable{} -- continued from previous page}}}\\
\hline

\endhead

\hline
\multicolumn{2}{r}{\makebox[0pt][r]{\sphinxtablecontinued{Continued on next page}}}\\
\endfoot

\endlastfoot

{\hyperref[\detokenize{api/mastml.hyper_opt.RandomizedSearch:mastml.hyper_opt.RandomizedSearch.fit}]{\sphinxcrossref{\sphinxcode{\sphinxupquote{fit}}}}}(X, y, model{[}, cv, savepath, refit{]})
&

\\
\hline
\end{longtable}\sphinxatlongtableend\end{savenotes}
\subsubsection*{Methods Documentation}
\index{fit() (mastml.hyper\_opt.RandomizedSearch method)@\spxentry{fit()}\spxextra{mastml.hyper\_opt.RandomizedSearch method}}

\begin{fulllineitems}
\phantomsection\label{\detokenize{api/mastml.hyper_opt.RandomizedSearch:mastml.hyper_opt.RandomizedSearch.fit}}\pysiglinewithargsret{\sphinxbfcode{\sphinxupquote{fit}}}{\emph{X}, \emph{y}, \emph{model}, \emph{cv=None}, \emph{savepath=None}, \emph{refit=True}}{}
\end{fulllineitems}


\end{fulllineitems}



\subsection{Class Inheritance Diagram}
\label{\detokenize{7_hyper_opt:class-inheritance-diagram}}
\sphinxincludegraphics[]{None}


\chapter{Code Documentation: Learning Curve}
\label{\detokenize{8_learning_curve:code-documentation-learning-curve}}\label{\detokenize{8_learning_curve::doc}}

\section{mastml.learning\_curve Module}
\label{\detokenize{8_learning_curve:module-mastml.learning_curve}}\label{\detokenize{8_learning_curve:mastml-learning-curve-module}}\index{mastml.learning\_curve (module)@\spxentry{mastml.learning\_curve}\spxextra{module}}
This module contains methods to construct learning curves, which evaluate some cross-validation performance metric
(e.g. RMSE) as a function of amount of training data (i.e. a data learning curve) or as a function of the number of
features used in the fitting (i.e. a feature learning curve).
\begin{description}
\item[{LearningCurve:}] \leavevmode
Class used to construct data learning curves and feature learning curves

\end{description}


\subsection{Classes}
\label{\detokenize{8_learning_curve:classes}}

\begin{savenotes}\sphinxatlongtablestart\begin{longtable}[c]{\X{1}{2}\X{1}{2}}
\hline

\endfirsthead

\multicolumn{2}{c}%
{\makebox[0pt]{\sphinxtablecontinued{\tablename\ \thetable{} -- continued from previous page}}}\\
\hline

\endhead

\hline
\multicolumn{2}{r}{\makebox[0pt][r]{\sphinxtablecontinued{Continued on next page}}}\\
\endfoot

\endlastfoot

\sphinxcode{\sphinxupquote{KFold}}({[}n\_splits, shuffle, random\_state{]})
&
K-Folds cross-validator
\\
\hline
{\hyperref[\detokenize{api/mastml.learning_curve.LearningCurve:mastml.learning_curve.LearningCurve}]{\sphinxcrossref{\sphinxcode{\sphinxupquote{LearningCurve}}}}}()
&
This class is used to construct learning curves, both in the form of model performance vs.
\\
\hline
\sphinxcode{\sphinxupquote{Line}}
&
Class containing methods for constructing line plots
\\
\hline
\sphinxcode{\sphinxupquote{Metrics}}(metrics\_list{[}, metrics\_type{]})
&
Class containing access to a wide range of metrics from scikit-learn and a number of MAST-ML custom-written metrics
\\
\hline
\sphinxcode{\sphinxupquote{SklearnFeatureSelector}}(selector, **kwargs)
&
Class that wraps scikit-learn feature selection methods with some new MAST-ML functionality
\\
\hline
\sphinxcode{\sphinxupquote{datetime}}(year, month, day{[}, hour{[}, minute{[}, …)
&
The year, month and day arguments are required.
\\
\hline
\end{longtable}\sphinxatlongtableend\end{savenotes}


\subsubsection{LearningCurve}
\label{\detokenize{api/mastml.learning_curve.LearningCurve:learningcurve}}\label{\detokenize{api/mastml.learning_curve.LearningCurve::doc}}\index{LearningCurve (class in mastml.learning\_curve)@\spxentry{LearningCurve}\spxextra{class in mastml.learning\_curve}}

\begin{fulllineitems}
\phantomsection\label{\detokenize{api/mastml.learning_curve.LearningCurve:mastml.learning_curve.LearningCurve}}\pysigline{\sphinxbfcode{\sphinxupquote{class }}\sphinxcode{\sphinxupquote{mastml.learning\_curve.}}\sphinxbfcode{\sphinxupquote{LearningCurve}}}
Bases: \sphinxcode{\sphinxupquote{object}}

This class is used to construct learning curves, both in the form of model performance vs. amount of training
data and model performance vs. number of features used in the fit.
\begin{description}
\item[{Args:}] \leavevmode
None

\item[{Methods:}] \leavevmode\begin{description}
\item[{evaluate: Sets up a save directory and performs both the data and feature-based learning curves}] \leavevmode\begin{description}
\item[{Args:}] \leavevmode
model: (SklearnModel or EnsembleModel), a model made in MAST-ML

X: (pd.DataFrame), dataframe containing the X feature matrix

y: (pd.Series), series containing the target y data

savepath: (str), string denoting the savepath to save the learning curve output

groups: (pd.Series), series of group designation

train\_sizes: (list or np.array), list or array of floats denoting fractions of training data to evaluate for data learning curve

cv: (scikit-learn cross-validation object), a scikit-learn cross-validation object

scoring: (str), string denoting name of regression metric to evaluate learning curves. See mastml.metrics.Metrics.\_metric\_zoo for full list

selector: (mastml.feature\_selector), a mastml.feature\_selectors instance

make\_plot: (bool), whether or not to make the learning curve plots

\end{description}

\item[{data\_learning\_curve: Method that calculates the model CV score as a function of amount of training data used}] \leavevmode\begin{description}
\item[{Args:}] \leavevmode
model: (SklearnModel or EnsembleModel), a model made in MAST-ML

X: (pd.DataFrame), dataframe containing the X feature matrix

y: (pd.Series), series containing the target y data

savepath: (str), string denoting the savepath to save the learning curve output

groups: (pd.Series), series of group designation

train\_sizes: (list or np.array), list or array of floats denoting fractions of training data to evaluate for data learning curve

cv: (scikit-learn cross-validation object), a scikit-learn cross-validation object

scoring: (str), string denoting name of regression metric to evaluate learning curves. See mastml.metrics.Metrics.\_metric\_zoo for full list

make\_plot: (bool), whether or not to make the learning curve plots

\item[{Returns:}] \leavevmode
None

\end{description}

\item[{feature\_learning\_curve: Method that calculates the model CV score as a function of the number of features used}] \leavevmode\begin{description}
\item[{Args:}] \leavevmode
model: (SklearnModel or EnsembleModel), a model made in MAST-ML

X: (pd.DataFrame), dataframe containing the X feature matrix

y: (pd.Series), series containing the target y data

savepath: (str), string denoting the savepath to save the learning curve output

groups: (pd.Series), series of group designation

cv: (scikit-learn cross-validation object), a scikit-learn cross-validation object

scoring: (str), string denoting name of regression metric to evaluate learning curves. See mastml.metrics.Metrics.\_metric\_zoo for full list

selector: (mastml.feature\_selector), a mastml.feature\_selectors instance

make\_plot: (bool), whether or not to make the learning curve plots

\item[{Returns:}] \leavevmode
None

\end{description}

\item[{\_setup\_savedir: Method to create the output save directory for learning curve data}] \leavevmode\begin{description}
\item[{Args:}] \leavevmode
savepath: (str), string denoting the base path to save the output to

\item[{Returns:}] \leavevmode
splitdir: (str), path where learning curve data will be saved to

\end{description}

\end{description}

\end{description}
\subsubsection*{Methods Summary}


\begin{savenotes}\sphinxatlongtablestart\begin{longtable}[c]{\X{1}{2}\X{1}{2}}
\hline

\endfirsthead

\multicolumn{2}{c}%
{\makebox[0pt]{\sphinxtablecontinued{\tablename\ \thetable{} -- continued from previous page}}}\\
\hline

\endhead

\hline
\multicolumn{2}{r}{\makebox[0pt][r]{\sphinxtablecontinued{Continued on next page}}}\\
\endfoot

\endlastfoot

{\hyperref[\detokenize{api/mastml.learning_curve.LearningCurve:mastml.learning_curve.LearningCurve.data_learning_curve}]{\sphinxcrossref{\sphinxcode{\sphinxupquote{data\_learning\_curve}}}}}(model, X, y{[}, savepath, …{]})
&

\\
\hline
{\hyperref[\detokenize{api/mastml.learning_curve.LearningCurve:mastml.learning_curve.LearningCurve.evaluate}]{\sphinxcrossref{\sphinxcode{\sphinxupquote{evaluate}}}}}(model, X, y{[}, savepath, groups, …{]})
&

\\
\hline
{\hyperref[\detokenize{api/mastml.learning_curve.LearningCurve:mastml.learning_curve.LearningCurve.feature_learning_curve}]{\sphinxcrossref{\sphinxcode{\sphinxupquote{feature\_learning\_curve}}}}}(model, X, y{[}, …{]})
&

\\
\hline
\end{longtable}\sphinxatlongtableend\end{savenotes}
\subsubsection*{Methods Documentation}
\index{data\_learning\_curve() (mastml.learning\_curve.LearningCurve method)@\spxentry{data\_learning\_curve()}\spxextra{mastml.learning\_curve.LearningCurve method}}

\begin{fulllineitems}
\phantomsection\label{\detokenize{api/mastml.learning_curve.LearningCurve:mastml.learning_curve.LearningCurve.data_learning_curve}}\pysiglinewithargsret{\sphinxbfcode{\sphinxupquote{data\_learning\_curve}}}{\emph{model}, \emph{X}, \emph{y}, \emph{savepath=None}, \emph{groups=None}, \emph{train\_sizes=None}, \emph{cv=None}, \emph{scoring=None}, \emph{make\_plot=True}}{}
\end{fulllineitems}

\index{evaluate() (mastml.learning\_curve.LearningCurve method)@\spxentry{evaluate()}\spxextra{mastml.learning\_curve.LearningCurve method}}

\begin{fulllineitems}
\phantomsection\label{\detokenize{api/mastml.learning_curve.LearningCurve:mastml.learning_curve.LearningCurve.evaluate}}\pysiglinewithargsret{\sphinxbfcode{\sphinxupquote{evaluate}}}{\emph{model}, \emph{X}, \emph{y}, \emph{savepath=None}, \emph{groups=None}, \emph{train\_sizes=None}, \emph{cv=None}, \emph{scoring=None}, \emph{selector=None}, \emph{make\_plot=True}, \emph{make\_new\_dir=True}}{}
\end{fulllineitems}

\index{feature\_learning\_curve() (mastml.learning\_curve.LearningCurve method)@\spxentry{feature\_learning\_curve()}\spxextra{mastml.learning\_curve.LearningCurve method}}

\begin{fulllineitems}
\phantomsection\label{\detokenize{api/mastml.learning_curve.LearningCurve:mastml.learning_curve.LearningCurve.feature_learning_curve}}\pysiglinewithargsret{\sphinxbfcode{\sphinxupquote{feature\_learning\_curve}}}{\emph{model}, \emph{X}, \emph{y}, \emph{savepath=None}, \emph{groups=None}, \emph{cv=None}, \emph{scoring=None}, \emph{selector=None}, \emph{make\_plot=True}}{}
\end{fulllineitems}


\end{fulllineitems}



\subsection{Class Inheritance Diagram}
\label{\detokenize{8_learning_curve:class-inheritance-diagram}}
\sphinxincludegraphics[]{None}


\chapter{Code Documentation: Mastml}
\label{\detokenize{9_mastml:code-documentation-mastml}}\label{\detokenize{9_mastml::doc}}

\section{mastml.mastml Module}
\label{\detokenize{9_mastml:module-mastml.mastml}}\label{\detokenize{9_mastml:mastml-mastml-module}}\index{mastml.mastml (module)@\spxentry{mastml.mastml}\spxextra{module}}
This module contains routines to set up and manage the metadata for a MAST-ML run
\begin{description}
\item[{Mastml:}] \leavevmode
Class to set up directories for saving the output of a MAST-ML run, and for constructing and updating a
metadata summary file.

\end{description}


\subsection{Classes}
\label{\detokenize{9_mastml:classes}}

\begin{savenotes}\sphinxatlongtablestart\begin{longtable}[c]{\X{1}{2}\X{1}{2}}
\hline

\endfirsthead

\multicolumn{2}{c}%
{\makebox[0pt]{\sphinxtablecontinued{\tablename\ \thetable{} -- continued from previous page}}}\\
\hline

\endhead

\hline
\multicolumn{2}{r}{\makebox[0pt][r]{\sphinxtablecontinued{Continued on next page}}}\\
\endfoot

\endlastfoot

{\hyperref[\detokenize{api/mastml.mastml.Mastml:mastml.mastml.Mastml}]{\sphinxcrossref{\sphinxcode{\sphinxupquote{Mastml}}}}}(savepath{[}, mastml\_metadata{]})
&
Main helper class to initialize mastml runs and create and manage run metadata
\\
\hline
\sphinxcode{\sphinxupquote{OrderedDict}}
&
Dictionary that remembers insertion order
\\
\hline
\sphinxcode{\sphinxupquote{datetime}}(year, month, day{[}, hour{[}, minute{[}, …)
&
The year, month and day arguments are required.
\\
\hline
\end{longtable}\sphinxatlongtableend\end{savenotes}


\subsubsection{Mastml}
\label{\detokenize{api/mastml.mastml.Mastml:mastml}}\label{\detokenize{api/mastml.mastml.Mastml::doc}}\index{Mastml (class in mastml.mastml)@\spxentry{Mastml}\spxextra{class in mastml.mastml}}

\begin{fulllineitems}
\phantomsection\label{\detokenize{api/mastml.mastml.Mastml:mastml.mastml.Mastml}}\pysiglinewithargsret{\sphinxbfcode{\sphinxupquote{class }}\sphinxcode{\sphinxupquote{mastml.mastml.}}\sphinxbfcode{\sphinxupquote{Mastml}}}{\emph{savepath}, \emph{mastml\_metadata=None}}{}
Bases: \sphinxcode{\sphinxupquote{object}}

Main helper class to initialize mastml runs and create and manage run metadata
\begin{description}
\item[{Args:}] \leavevmode
savepath: (str), string specifing the savepath name for the mastml run

mastml\_metdata: (dict), dict of mastml metadata. If none, a new dict will be created

\item[{Methods:}] \leavevmode\begin{description}
\item[{\_initialize\_run: initializes run by making new metadata file or updating existing one, and initializing the output directory.}] \leavevmode\begin{description}
\item[{Args:}] \leavevmode
None

\item[{Returns:}] \leavevmode
None

\end{description}

\item[{\_initialize\_output: creates the output folder based on specified savepath and datetime information}] \leavevmode\begin{description}
\item[{Args:}] \leavevmode
None

\item[{Returns:}] \leavevmode
None

\end{description}

\item[{\_initialize\_metadata: creates a new metadata file and saves the savepath info to it}] \leavevmode\begin{description}
\item[{Args:}] \leavevmode
None

\item[{Returns:}] \leavevmode
None

\end{description}

\item[{\_update\_metadata: placeholder for updating the metadata file with new run information}] \leavevmode\begin{description}
\item[{Args:}] \leavevmode
None

\item[{Returns:}] \leavevmode
None

\end{description}

\item[{\_save\_mastml\_metadata: saves the metadata dict as a json file}] \leavevmode\begin{description}
\item[{Args:}] \leavevmode
None

\item[{Returns:}] \leavevmode
None

\end{description}

\item[{get\_savepath: returns the savepath}] \leavevmode\begin{description}
\item[{Args:}] \leavevmode
None

\item[{Returns:}] \leavevmode
string specifying the savepath of the mastml run

\end{description}

\item[{get\_mastml\_metadata: returns the metadata file}] \leavevmode\begin{description}
\item[{Args:}] \leavevmode
None

\item[{Returns:}] \leavevmode
mastml metadata object (ordered dict)

\end{description}

\end{description}

\end{description}
\subsubsection*{Attributes Summary}


\begin{savenotes}\sphinxatlongtablestart\begin{longtable}[c]{\X{1}{2}\X{1}{2}}
\hline

\endfirsthead

\multicolumn{2}{c}%
{\makebox[0pt]{\sphinxtablecontinued{\tablename\ \thetable{} -- continued from previous page}}}\\
\hline

\endhead

\hline
\multicolumn{2}{r}{\makebox[0pt][r]{\sphinxtablecontinued{Continued on next page}}}\\
\endfoot

\endlastfoot

{\hyperref[\detokenize{api/mastml.mastml.Mastml:mastml.mastml.Mastml.get_mastml_metadata}]{\sphinxcrossref{\sphinxcode{\sphinxupquote{get\_mastml\_metadata}}}}}
&

\\
\hline
{\hyperref[\detokenize{api/mastml.mastml.Mastml:mastml.mastml.Mastml.get_savepath}]{\sphinxcrossref{\sphinxcode{\sphinxupquote{get\_savepath}}}}}
&

\\
\hline
\end{longtable}\sphinxatlongtableend\end{savenotes}
\subsubsection*{Attributes Documentation}
\index{get\_mastml\_metadata (mastml.mastml.Mastml attribute)@\spxentry{get\_mastml\_metadata}\spxextra{mastml.mastml.Mastml attribute}}

\begin{fulllineitems}
\phantomsection\label{\detokenize{api/mastml.mastml.Mastml:mastml.mastml.Mastml.get_mastml_metadata}}\pysigline{\sphinxbfcode{\sphinxupquote{get\_mastml\_metadata}}}
\end{fulllineitems}

\index{get\_savepath (mastml.mastml.Mastml attribute)@\spxentry{get\_savepath}\spxextra{mastml.mastml.Mastml attribute}}

\begin{fulllineitems}
\phantomsection\label{\detokenize{api/mastml.mastml.Mastml:mastml.mastml.Mastml.get_savepath}}\pysigline{\sphinxbfcode{\sphinxupquote{get\_savepath}}}
\end{fulllineitems}


\end{fulllineitems}



\subsection{Class Inheritance Diagram}
\label{\detokenize{9_mastml:class-inheritance-diagram}}
\sphinxincludegraphics[]{None}


\chapter{Code Documentation: Metrics}
\label{\detokenize{10_metrics:code-documentation-metrics}}\label{\detokenize{10_metrics::doc}}

\section{mastml.metrics Module}
\label{\detokenize{10_metrics:module-mastml.metrics}}\label{\detokenize{10_metrics:mastml-metrics-module}}\index{mastml.metrics (module)@\spxentry{mastml.metrics}\spxextra{module}}
This module contains a metrics class for construction and evaluation of various regression score metrics between
true and model predicted data.
\begin{description}
\item[{Metrics:}] \leavevmode
Class to construct and evaluate a list of regression metrics of interest. The full list of available metrics
can be obtained from Metrics().\_metric\_zoo()

\end{description}


\subsection{Functions}
\label{\detokenize{10_metrics:functions}}

\begin{savenotes}\sphinxatlongtablestart\begin{longtable}[c]{\X{1}{2}\X{1}{2}}
\hline

\endfirsthead

\multicolumn{2}{c}%
{\makebox[0pt]{\sphinxtablecontinued{\tablename\ \thetable{} -- continued from previous page}}}\\
\hline

\endhead

\hline
\multicolumn{2}{r}{\makebox[0pt][r]{\sphinxtablecontinued{Continued on next page}}}\\
\endfoot

\endlastfoot

{\hyperref[\detokenize{api/mastml.metrics.r2_score_adjusted:mastml.metrics.r2_score_adjusted}]{\sphinxcrossref{\sphinxcode{\sphinxupquote{r2\_score\_adjusted}}}}}(y\_true, y\_pred{[}, n\_features{]})
&
Method that calculates the adjusted R\textasciicircum{}2 value
\\
\hline
{\hyperref[\detokenize{api/mastml.metrics.r2_score_fitted:mastml.metrics.r2_score_fitted}]{\sphinxcrossref{\sphinxcode{\sphinxupquote{r2\_score\_fitted}}}}}(y\_true, y\_pred)
&
Method that calculates the R\textasciicircum{}2 value
\\
\hline
{\hyperref[\detokenize{api/mastml.metrics.r2_score_noint:mastml.metrics.r2_score_noint}]{\sphinxcrossref{\sphinxcode{\sphinxupquote{r2\_score\_noint}}}}}(y\_true, y\_pred)
&
Method that calculates the R\textasciicircum{}2 value without fitting the y-intercept
\\
\hline
{\hyperref[\detokenize{api/mastml.metrics.rmse_over_stdev:mastml.metrics.rmse_over_stdev}]{\sphinxcrossref{\sphinxcode{\sphinxupquote{rmse\_over\_stdev}}}}}(y\_true, y\_pred{[}, train\_y{]})
&
Method that calculates the root mean squared error (RMSE) of a set of data, divided by the standard deviation of the training data set.
\\
\hline
{\hyperref[\detokenize{api/mastml.metrics.root_mean_squared_error:mastml.metrics.root_mean_squared_error}]{\sphinxcrossref{\sphinxcode{\sphinxupquote{root\_mean\_squared\_error}}}}}(y\_true, y\_pred)
&
Method that calculates the root mean squared error (RMSE)
\\
\hline
\end{longtable}\sphinxatlongtableend\end{savenotes}


\subsubsection{r2\_score\_adjusted}
\label{\detokenize{api/mastml.metrics.r2_score_adjusted:r2-score-adjusted}}\label{\detokenize{api/mastml.metrics.r2_score_adjusted::doc}}\index{r2\_score\_adjusted() (in module mastml.metrics)@\spxentry{r2\_score\_adjusted()}\spxextra{in module mastml.metrics}}

\begin{fulllineitems}
\phantomsection\label{\detokenize{api/mastml.metrics.r2_score_adjusted:mastml.metrics.r2_score_adjusted}}\pysiglinewithargsret{\sphinxcode{\sphinxupquote{mastml.metrics.}}\sphinxbfcode{\sphinxupquote{r2\_score\_adjusted}}}{\emph{y\_true}, \emph{y\_pred}, \emph{n\_features=None}}{}
Method that calculates the adjusted R\textasciicircum{}2 value
\begin{description}
\item[{Args:}] \leavevmode
y\_true: (numpy array), array of true y data values

y\_pred: (numpy array), array of predicted y data values

n\_features: (int), number of features used in the fit

\item[{Returns:}] \leavevmode
(float): score of adjusted R\textasciicircum{}2

\end{description}

\end{fulllineitems}



\subsubsection{r2\_score\_fitted}
\label{\detokenize{api/mastml.metrics.r2_score_fitted:r2-score-fitted}}\label{\detokenize{api/mastml.metrics.r2_score_fitted::doc}}\index{r2\_score\_fitted() (in module mastml.metrics)@\spxentry{r2\_score\_fitted()}\spxextra{in module mastml.metrics}}

\begin{fulllineitems}
\phantomsection\label{\detokenize{api/mastml.metrics.r2_score_fitted:mastml.metrics.r2_score_fitted}}\pysiglinewithargsret{\sphinxcode{\sphinxupquote{mastml.metrics.}}\sphinxbfcode{\sphinxupquote{r2\_score\_fitted}}}{\emph{y\_true}, \emph{y\_pred}}{}
Method that calculates the R\textasciicircum{}2 value
\begin{description}
\item[{Args:}] \leavevmode
y\_true: (numpy array), array of true y data values

y\_pred: (numpy array), array of predicted y data values

\item[{Returns:}] \leavevmode
(float): score of R\textasciicircum{}2

\end{description}

\end{fulllineitems}



\subsubsection{r2\_score\_noint}
\label{\detokenize{api/mastml.metrics.r2_score_noint:r2-score-noint}}\label{\detokenize{api/mastml.metrics.r2_score_noint::doc}}\index{r2\_score\_noint() (in module mastml.metrics)@\spxentry{r2\_score\_noint()}\spxextra{in module mastml.metrics}}

\begin{fulllineitems}
\phantomsection\label{\detokenize{api/mastml.metrics.r2_score_noint:mastml.metrics.r2_score_noint}}\pysiglinewithargsret{\sphinxcode{\sphinxupquote{mastml.metrics.}}\sphinxbfcode{\sphinxupquote{r2\_score\_noint}}}{\emph{y\_true}, \emph{y\_pred}}{}
Method that calculates the R\textasciicircum{}2 value without fitting the y-intercept
\begin{description}
\item[{Args:}] \leavevmode
y\_true: (numpy array), array of true y data values

y\_pred: (numpy array), array of predicted y data values

\item[{Returns:}] \leavevmode
(float): score of R\textasciicircum{}2 with no y-intercept

\end{description}

\end{fulllineitems}



\subsubsection{rmse\_over\_stdev}
\label{\detokenize{api/mastml.metrics.rmse_over_stdev:rmse-over-stdev}}\label{\detokenize{api/mastml.metrics.rmse_over_stdev::doc}}\index{rmse\_over\_stdev() (in module mastml.metrics)@\spxentry{rmse\_over\_stdev()}\spxextra{in module mastml.metrics}}

\begin{fulllineitems}
\phantomsection\label{\detokenize{api/mastml.metrics.rmse_over_stdev:mastml.metrics.rmse_over_stdev}}\pysiglinewithargsret{\sphinxcode{\sphinxupquote{mastml.metrics.}}\sphinxbfcode{\sphinxupquote{rmse\_over\_stdev}}}{\emph{y\_true}, \emph{y\_pred}, \emph{train\_y=None}}{}
Method that calculates the root mean squared error (RMSE) of a set of data, divided by the standard deviation of
the training data set.
\begin{description}
\item[{Args:}] \leavevmode
y\_true: (numpy array), array of true y data values

y\_pred: (numpy array), array of predicted y data values

train\_y: (numpy array), array of training y data values

\item[{Returns:}] \leavevmode
(float): score of RMSE divided by standard deviation of training data

\end{description}

\end{fulllineitems}



\subsubsection{root\_mean\_squared\_error}
\label{\detokenize{api/mastml.metrics.root_mean_squared_error:root-mean-squared-error}}\label{\detokenize{api/mastml.metrics.root_mean_squared_error::doc}}\index{root\_mean\_squared\_error() (in module mastml.metrics)@\spxentry{root\_mean\_squared\_error()}\spxextra{in module mastml.metrics}}

\begin{fulllineitems}
\phantomsection\label{\detokenize{api/mastml.metrics.root_mean_squared_error:mastml.metrics.root_mean_squared_error}}\pysiglinewithargsret{\sphinxcode{\sphinxupquote{mastml.metrics.}}\sphinxbfcode{\sphinxupquote{root\_mean\_squared\_error}}}{\emph{y\_true}, \emph{y\_pred}}{}
Method that calculates the root mean squared error (RMSE)
\begin{description}
\item[{Args:}] \leavevmode
y\_true: (numpy array), array of true y data values

y\_pred: (numpy array), array of predicted y data values

\item[{Returns:}] \leavevmode
(float): score of RMSE

\end{description}

\end{fulllineitems}



\subsection{Classes}
\label{\detokenize{10_metrics:classes}}

\begin{savenotes}\sphinxatlongtablestart\begin{longtable}[c]{\X{1}{2}\X{1}{2}}
\hline

\endfirsthead

\multicolumn{2}{c}%
{\makebox[0pt]{\sphinxtablecontinued{\tablename\ \thetable{} -- continued from previous page}}}\\
\hline

\endhead

\hline
\multicolumn{2}{r}{\makebox[0pt][r]{\sphinxtablecontinued{Continued on next page}}}\\
\endfoot

\endlastfoot

\sphinxcode{\sphinxupquote{LinearRegression}}(*{[}, fit\_intercept, …{]})
&
Ordinary least squares Linear Regression.
\\
\hline
{\hyperref[\detokenize{api/mastml.metrics.Metrics:mastml.metrics.Metrics}]{\sphinxcrossref{\sphinxcode{\sphinxupquote{Metrics}}}}}(metrics\_list{[}, metrics\_type{]})
&
Class containing access to a wide range of metrics from scikit-learn and a number of MAST-ML custom-written metrics
\\
\hline
\end{longtable}\sphinxatlongtableend\end{savenotes}


\subsubsection{Metrics}
\label{\detokenize{api/mastml.metrics.Metrics:metrics}}\label{\detokenize{api/mastml.metrics.Metrics::doc}}\index{Metrics (class in mastml.metrics)@\spxentry{Metrics}\spxextra{class in mastml.metrics}}

\begin{fulllineitems}
\phantomsection\label{\detokenize{api/mastml.metrics.Metrics:mastml.metrics.Metrics}}\pysiglinewithargsret{\sphinxbfcode{\sphinxupquote{class }}\sphinxcode{\sphinxupquote{mastml.metrics.}}\sphinxbfcode{\sphinxupquote{Metrics}}}{\emph{metrics\_list}, \emph{metrics\_type='regression'}}{}
Bases: \sphinxcode{\sphinxupquote{object}}

Class containing access to a wide range of metrics from scikit-learn and a number of MAST-ML custom-written metrics
\begin{description}
\item[{Args:}] \leavevmode
metrics\_list: (list), list of strings of metric names to use

metrics\_type: (str), one of ‘regression’ or ‘classification’: whether to use set of common regression/classifier metrics

\item[{Methods:}] \leavevmode\begin{description}
\item[{evaluate: main method to evaluate the specified metrics and the provided true and pred data}] \leavevmode\begin{description}
\item[{Args:}] \leavevmode
y\_true: (pd.Series), series of true y data

y\_pred: (pd.Series), series of predicted y data

\item[{Returns:}] \leavevmode
stats\_dict: (dict), dictionary of calculated statistics for each metric

\end{description}

\item[{\_get\_metrics: builds the metrics dict of metric names}] \leavevmode{[}metric instances based on the metrics specified in metrics\_list{]}\begin{description}
\item[{Args:}] \leavevmode
None

\item[{Returns:}] \leavevmode
None

\end{description}

\item[{\_metric\_zoo: method to retrieve full dict of metric names}] \leavevmode{[}metric instance pairs{]}\begin{description}
\item[{Args:}] \leavevmode
None

\item[{Returns:}] \leavevmode
all\_metrics (dict), dictionary of all metric names and instances

\end{description}

\end{description}

\end{description}
\subsubsection*{Methods Summary}


\begin{savenotes}\sphinxatlongtablestart\begin{longtable}[c]{\X{1}{2}\X{1}{2}}
\hline

\endfirsthead

\multicolumn{2}{c}%
{\makebox[0pt]{\sphinxtablecontinued{\tablename\ \thetable{} -- continued from previous page}}}\\
\hline

\endhead

\hline
\multicolumn{2}{r}{\makebox[0pt][r]{\sphinxtablecontinued{Continued on next page}}}\\
\endfoot

\endlastfoot

{\hyperref[\detokenize{api/mastml.metrics.Metrics:mastml.metrics.Metrics.evaluate}]{\sphinxcrossref{\sphinxcode{\sphinxupquote{evaluate}}}}}(y\_true, y\_pred)
&

\\
\hline
\end{longtable}\sphinxatlongtableend\end{savenotes}
\subsubsection*{Methods Documentation}
\index{evaluate() (mastml.metrics.Metrics method)@\spxentry{evaluate()}\spxextra{mastml.metrics.Metrics method}}

\begin{fulllineitems}
\phantomsection\label{\detokenize{api/mastml.metrics.Metrics:mastml.metrics.Metrics.evaluate}}\pysiglinewithargsret{\sphinxbfcode{\sphinxupquote{evaluate}}}{\emph{y\_true}, \emph{y\_pred}}{}
\end{fulllineitems}


\end{fulllineitems}



\subsection{Class Inheritance Diagram}
\label{\detokenize{10_metrics:class-inheritance-diagram}}
\sphinxincludegraphics[]{None}


\chapter{Code Documentation: Models}
\label{\detokenize{11_models:code-documentation-models}}\label{\detokenize{11_models::doc}}

\section{mastml.models Module}
\label{\detokenize{11_models:module-mastml.models}}\label{\detokenize{11_models:mastml-models-module}}\index{mastml.models (module)@\spxentry{mastml.models}\spxextra{module}}
Module for constructing models for use in MAST-ML.
\begin{description}
\item[{SklearnModel:}] \leavevmode
Class that wraps scikit-learn models to have MAST-ML type functionality. Providing the model name as a string
and the keyword arguments for the model parameters will construct the model. Note that this class also supports
construction of XGBoost models and Keras neural network models via Keras’ keras.wrappers.scikit\_learn.KerasRegressor
model.

\item[{EnsembleModel:}] \leavevmode
Class that constructs a model which is an ensemble of many base models (sometimes called weak learners). This
class supports construction of ensembles of most scikit-learn regression models as well as ensembles of neural
networks that are made via Keras’ keras.wrappers.scikit\_learn.KerasRegressor class.

\end{description}


\subsection{Classes}
\label{\detokenize{11_models:classes}}

\begin{savenotes}\sphinxatlongtablestart\begin{longtable}[c]{\X{1}{2}\X{1}{2}}
\hline

\endfirsthead

\multicolumn{2}{c}%
{\makebox[0pt]{\sphinxtablecontinued{\tablename\ \thetable{} -- continued from previous page}}}\\
\hline

\endhead

\hline
\multicolumn{2}{r}{\makebox[0pt][r]{\sphinxtablecontinued{Continued on next page}}}\\
\endfoot

\endlastfoot

\sphinxcode{\sphinxupquote{BaggingRegressor}}({[}base\_estimator, …{]})
&
A Bagging regressor.
\\
\hline
\sphinxcode{\sphinxupquote{BaseEstimator}}
&
Base class for all estimators in scikit-learn.
\\
\hline
{\hyperref[\detokenize{api/mastml.models.EnsembleModel:mastml.models.EnsembleModel}]{\sphinxcrossref{\sphinxcode{\sphinxupquote{EnsembleModel}}}}}(model, n\_estimators, **kwargs)
&
Class used to construct ensemble models with a particular number and type of weak learner (base model).
\\
\hline
\sphinxcode{\sphinxupquote{GaussianProcessRegressor}}({[}kernel, alpha, …{]})
&
Gaussian process regression (GPR).
\\
\hline
{\hyperref[\detokenize{api/mastml.models.SklearnModel:mastml.models.SklearnModel}]{\sphinxcrossref{\sphinxcode{\sphinxupquote{SklearnModel}}}}}(model, **kwargs)
&
Class to wrap any sklearn estimator, and provide some new dataframe functionality
\\
\hline
\sphinxcode{\sphinxupquote{TransformerMixin}}
&
Mixin class for all transformers in scikit-learn.
\\
\hline
\end{longtable}\sphinxatlongtableend\end{savenotes}


\subsubsection{EnsembleModel}
\label{\detokenize{api/mastml.models.EnsembleModel:ensemblemodel}}\label{\detokenize{api/mastml.models.EnsembleModel::doc}}\index{EnsembleModel (class in mastml.models)@\spxentry{EnsembleModel}\spxextra{class in mastml.models}}

\begin{fulllineitems}
\phantomsection\label{\detokenize{api/mastml.models.EnsembleModel:mastml.models.EnsembleModel}}\pysiglinewithargsret{\sphinxbfcode{\sphinxupquote{class }}\sphinxcode{\sphinxupquote{mastml.models.}}\sphinxbfcode{\sphinxupquote{EnsembleModel}}}{\emph{model}, \emph{n\_estimators}, \emph{**kwargs}}{}
Bases: \sphinxcode{\sphinxupquote{sklearn.base.BaseEstimator}}, \sphinxcode{\sphinxupquote{sklearn.base.TransformerMixin}}

Class used to construct ensemble models with a particular number and type of weak learner (base model). The
ensemble model is compatible with most scikit-learn regressor models and KerasRegressor models
\begin{description}
\item[{Args:}] \leavevmode
model: (str), string name denoting the name of the model type to use as the base model

n\_estimators: (int), the number of base models to include in the ensemble

kwargs: keyword arguments for the base model parameter names and values

\item[{Methods:}] \leavevmode\begin{description}
\item[{fit: method that fits the model parameters to the provided training data}] \leavevmode\begin{description}
\item[{Args:}] \leavevmode
X: (pd.DataFrame), dataframe of X features

y: (pd.Series), series of y target data

\item[{Returns:}] \leavevmode
fitted model

\end{description}

\item[{predict: method that evaluates model on new data to give predictions}] \leavevmode\begin{description}
\item[{Args:}] \leavevmode
X: (pd.DataFrame), dataframe of X features

as\_frame: (bool), whether to return data as pandas dataframe (else numpy array)

\item[{Returns:}] \leavevmode
series or array of predicted values

\end{description}

\item[{get\_params: method to output key model parameters}] \leavevmode\begin{description}
\item[{Args:}] \leavevmode
deep: (bool), determines the extent of information returned, default True

\item[{Returns:}] \leavevmode
information on model parameters

\end{description}

\end{description}

\end{description}
\subsubsection*{Methods Summary}


\begin{savenotes}\sphinxatlongtablestart\begin{longtable}[c]{\X{1}{2}\X{1}{2}}
\hline

\endfirsthead

\multicolumn{2}{c}%
{\makebox[0pt]{\sphinxtablecontinued{\tablename\ \thetable{} -- continued from previous page}}}\\
\hline

\endhead

\hline
\multicolumn{2}{r}{\makebox[0pt][r]{\sphinxtablecontinued{Continued on next page}}}\\
\endfoot

\endlastfoot

{\hyperref[\detokenize{api/mastml.models.EnsembleModel:mastml.models.EnsembleModel.fit}]{\sphinxcrossref{\sphinxcode{\sphinxupquote{fit}}}}}(X, y)
&

\\
\hline
{\hyperref[\detokenize{api/mastml.models.EnsembleModel:mastml.models.EnsembleModel.get_params}]{\sphinxcrossref{\sphinxcode{\sphinxupquote{get\_params}}}}}({[}deep{]})
&
Get parameters for this estimator.
\\
\hline
{\hyperref[\detokenize{api/mastml.models.EnsembleModel:mastml.models.EnsembleModel.predict}]{\sphinxcrossref{\sphinxcode{\sphinxupquote{predict}}}}}(X{[}, as\_frame{]})
&

\\
\hline
\end{longtable}\sphinxatlongtableend\end{savenotes}
\subsubsection*{Methods Documentation}
\index{fit() (mastml.models.EnsembleModel method)@\spxentry{fit()}\spxextra{mastml.models.EnsembleModel method}}

\begin{fulllineitems}
\phantomsection\label{\detokenize{api/mastml.models.EnsembleModel:mastml.models.EnsembleModel.fit}}\pysiglinewithargsret{\sphinxbfcode{\sphinxupquote{fit}}}{\emph{X}, \emph{y}}{}
\end{fulllineitems}

\index{get\_params() (mastml.models.EnsembleModel method)@\spxentry{get\_params()}\spxextra{mastml.models.EnsembleModel method}}

\begin{fulllineitems}
\phantomsection\label{\detokenize{api/mastml.models.EnsembleModel:mastml.models.EnsembleModel.get_params}}\pysiglinewithargsret{\sphinxbfcode{\sphinxupquote{get\_params}}}{\emph{deep=True}}{}
Get parameters for this estimator.
\begin{description}
\item[{deep}] \leavevmode{[}bool, default=True{]}
If True, will return the parameters for this estimator and
contained subobjects that are estimators.

\end{description}
\begin{description}
\item[{params}] \leavevmode{[}dict{]}
Parameter names mapped to their values.

\end{description}

\end{fulllineitems}

\index{predict() (mastml.models.EnsembleModel method)@\spxentry{predict()}\spxextra{mastml.models.EnsembleModel method}}

\begin{fulllineitems}
\phantomsection\label{\detokenize{api/mastml.models.EnsembleModel:mastml.models.EnsembleModel.predict}}\pysiglinewithargsret{\sphinxbfcode{\sphinxupquote{predict}}}{\emph{X}, \emph{as\_frame=True}}{}
\end{fulllineitems}


\end{fulllineitems}



\subsubsection{SklearnModel}
\label{\detokenize{api/mastml.models.SklearnModel:sklearnmodel}}\label{\detokenize{api/mastml.models.SklearnModel::doc}}\index{SklearnModel (class in mastml.models)@\spxentry{SklearnModel}\spxextra{class in mastml.models}}

\begin{fulllineitems}
\phantomsection\label{\detokenize{api/mastml.models.SklearnModel:mastml.models.SklearnModel}}\pysiglinewithargsret{\sphinxbfcode{\sphinxupquote{class }}\sphinxcode{\sphinxupquote{mastml.models.}}\sphinxbfcode{\sphinxupquote{SklearnModel}}}{\emph{model}, \emph{**kwargs}}{}
Bases: \sphinxcode{\sphinxupquote{sklearn.base.BaseEstimator}}, \sphinxcode{\sphinxupquote{sklearn.base.TransformerMixin}}

Class to wrap any sklearn estimator, and provide some new dataframe functionality
\begin{description}
\item[{Args:}] \leavevmode
model: (str), string denoting the name of an sklearn estimator object, e.g. KernelRidge

kwargs: keyword pairs of values to include for model, e.g. for KernelRidge can specify kernel, alpha, gamma values

\item[{Methods:}] \leavevmode\begin{description}
\item[{fit: method that fits the model parameters to the provided training data}] \leavevmode\begin{description}
\item[{Args:}] \leavevmode
X: (pd.DataFrame), dataframe of X features

y: (pd.Series), series of y target data

\item[{Returns:}] \leavevmode
fitted model

\end{description}

\item[{predict: method that evaluates model on new data to give predictions}] \leavevmode\begin{description}
\item[{Args:}] \leavevmode
X: (pd.DataFrame), dataframe of X features

as\_frame: (bool), whether to return data as pandas dataframe (else numpy array)

\item[{Returns:}] \leavevmode
series or array of predicted values

\end{description}

\item[{help: method to output key information on class use, e.g. methods and parameters}] \leavevmode\begin{description}
\item[{Args:}] \leavevmode
None

\item[{Returns:}] \leavevmode
None, but outputs help to screen

\end{description}

\end{description}

\end{description}
\subsubsection*{Methods Summary}


\begin{savenotes}\sphinxatlongtablestart\begin{longtable}[c]{\X{1}{2}\X{1}{2}}
\hline

\endfirsthead

\multicolumn{2}{c}%
{\makebox[0pt]{\sphinxtablecontinued{\tablename\ \thetable{} -- continued from previous page}}}\\
\hline

\endhead

\hline
\multicolumn{2}{r}{\makebox[0pt][r]{\sphinxtablecontinued{Continued on next page}}}\\
\endfoot

\endlastfoot

{\hyperref[\detokenize{api/mastml.models.SklearnModel:mastml.models.SklearnModel.fit}]{\sphinxcrossref{\sphinxcode{\sphinxupquote{fit}}}}}(X, y)
&

\\
\hline
{\hyperref[\detokenize{api/mastml.models.SklearnModel:mastml.models.SklearnModel.get_params}]{\sphinxcrossref{\sphinxcode{\sphinxupquote{get\_params}}}}}({[}deep{]})
&
Get parameters for this estimator.
\\
\hline
{\hyperref[\detokenize{api/mastml.models.SklearnModel:mastml.models.SklearnModel.help}]{\sphinxcrossref{\sphinxcode{\sphinxupquote{help}}}}}()
&

\\
\hline
{\hyperref[\detokenize{api/mastml.models.SklearnModel:mastml.models.SklearnModel.predict}]{\sphinxcrossref{\sphinxcode{\sphinxupquote{predict}}}}}(X{[}, as\_frame{]})
&

\\
\hline
\end{longtable}\sphinxatlongtableend\end{savenotes}
\subsubsection*{Methods Documentation}
\index{fit() (mastml.models.SklearnModel method)@\spxentry{fit()}\spxextra{mastml.models.SklearnModel method}}

\begin{fulllineitems}
\phantomsection\label{\detokenize{api/mastml.models.SklearnModel:mastml.models.SklearnModel.fit}}\pysiglinewithargsret{\sphinxbfcode{\sphinxupquote{fit}}}{\emph{X}, \emph{y}}{}
\end{fulllineitems}

\index{get\_params() (mastml.models.SklearnModel method)@\spxentry{get\_params()}\spxextra{mastml.models.SklearnModel method}}

\begin{fulllineitems}
\phantomsection\label{\detokenize{api/mastml.models.SklearnModel:mastml.models.SklearnModel.get_params}}\pysiglinewithargsret{\sphinxbfcode{\sphinxupquote{get\_params}}}{\emph{deep=True}}{}
Get parameters for this estimator.
\begin{description}
\item[{deep}] \leavevmode{[}bool, default=True{]}
If True, will return the parameters for this estimator and
contained subobjects that are estimators.

\end{description}
\begin{description}
\item[{params}] \leavevmode{[}dict{]}
Parameter names mapped to their values.

\end{description}

\end{fulllineitems}

\index{help() (mastml.models.SklearnModel method)@\spxentry{help()}\spxextra{mastml.models.SklearnModel method}}

\begin{fulllineitems}
\phantomsection\label{\detokenize{api/mastml.models.SklearnModel:mastml.models.SklearnModel.help}}\pysiglinewithargsret{\sphinxbfcode{\sphinxupquote{help}}}{}{}
\end{fulllineitems}

\index{predict() (mastml.models.SklearnModel method)@\spxentry{predict()}\spxextra{mastml.models.SklearnModel method}}

\begin{fulllineitems}
\phantomsection\label{\detokenize{api/mastml.models.SklearnModel:mastml.models.SklearnModel.predict}}\pysiglinewithargsret{\sphinxbfcode{\sphinxupquote{predict}}}{\emph{X}, \emph{as\_frame=True}}{}
\end{fulllineitems}


\end{fulllineitems}



\subsection{Class Inheritance Diagram}
\label{\detokenize{11_models:class-inheritance-diagram}}
\sphinxincludegraphics[]{None}


\chapter{Code Documentation: Plots}
\label{\detokenize{12_plots:code-documentation-plots}}\label{\detokenize{12_plots::doc}}

\section{mastml.plots Module}
\label{\detokenize{12_plots:module-mastml.plots}}\label{\detokenize{12_plots:mastml-plots-module}}\index{mastml.plots (module)@\spxentry{mastml.plots}\spxextra{module}}
This module contains classes used for generating different types of analysis plots
\begin{description}
\item[{Scatter:}] \leavevmode
This class contains a variety of scatter plot types, e.g. parity (predicted vs. true) plots

\item[{Error:}] \leavevmode
This class contains plotting methods used to better quantify the model errors and uncertainty quantification.

\item[{Histogram:}] \leavevmode
This class contains methods for constructing histograms of data distributions and visualization of model residuals.

\item[{Line:}] \leavevmode
This class contains methods for making line plots, e.g. for constructing learning curves of model performance vs.
amount of data or number of features.

\end{description}


\subsection{Functions}
\label{\detokenize{12_plots:functions}}

\begin{savenotes}\sphinxatlongtablestart\begin{longtable}[c]{\X{1}{2}\X{1}{2}}
\hline

\endfirsthead

\multicolumn{2}{c}%
{\makebox[0pt]{\sphinxtablecontinued{\tablename\ \thetable{} -- continued from previous page}}}\\
\hline

\endhead

\hline
\multicolumn{2}{r}{\makebox[0pt][r]{\sphinxtablecontinued{Continued on next page}}}\\
\endfoot

\endlastfoot

\sphinxcode{\sphinxupquote{ceil}}
&
Return the ceiling of x as an Integral.
\\
\hline
{\hyperref[\detokenize{api/mastml.plots.check_dimensions:mastml.plots.check_dimensions}]{\sphinxcrossref{\sphinxcode{\sphinxupquote{check\_dimensions}}}}}(y)
&
Method to check the dimensions of supplied data.
\\
\hline
\sphinxcode{\sphinxupquote{figaspect}}(arg)
&
Calculate the width and height for a figure with a specified aspect ratio.
\\
\hline
{\hyperref[\detokenize{api/mastml.plots.get_divisor:mastml.plots.get_divisor}]{\sphinxcrossref{\sphinxcode{\sphinxupquote{get\_divisor}}}}}(high, low)
&
Method to obtain a sensible divisor based on range of two values
\\
\hline
\sphinxcode{\sphinxupquote{log}}(x, {[}base=math.e{]})
&
Return the logarithm of x to the given base.
\\
\hline
\sphinxcode{\sphinxupquote{make\_axes\_locatable}}(axes)
&

\\
\hline
{\hyperref[\detokenize{api/mastml.plots.make_axis_same:mastml.plots.make_axis_same}]{\sphinxcrossref{\sphinxcode{\sphinxupquote{make\_axis\_same}}}}}(ax, max1, min1)
&
Method to make the x and y ticks for each axis the same.
\\
\hline
{\hyperref[\detokenize{api/mastml.plots.make_fig_ax:mastml.plots.make_fig_ax}]{\sphinxcrossref{\sphinxcode{\sphinxupquote{make\_fig\_ax}}}}}({[}aspect\_ratio, x\_align, left{]})
&
Method to make matplotlib figure and axes objects.
\\
\hline
{\hyperref[\detokenize{api/mastml.plots.make_fig_ax_square:mastml.plots.make_fig_ax_square}]{\sphinxcrossref{\sphinxcode{\sphinxupquote{make\_fig\_ax\_square}}}}}({[}aspect, aspect\_ratio{]})
&
Method to make square shaped matplotlib figure and axes objects.
\\
\hline
{\hyperref[\detokenize{api/mastml.plots.make_plots:mastml.plots.make_plots}]{\sphinxcrossref{\sphinxcode{\sphinxupquote{make\_plots}}}}}(plots, y\_true, y\_pred, groups, …)
&
Helper function to make collections of different types of plots after a single or multiple data splits are evaluated.
\\
\hline
\sphinxcode{\sphinxupquote{mark\_inset}}(parent\_axes, inset\_axes, loc1, …)
&
Draw a box to mark the location of an area represented by an inset axes.
\\
\hline
{\hyperref[\detokenize{api/mastml.plots.nice_mean:mastml.plots.nice_mean}]{\sphinxcrossref{\sphinxcode{\sphinxupquote{nice\_mean}}}}}(ls)
&
Method to return mean of a list or equivalent array with NaN values
\\
\hline
{\hyperref[\detokenize{api/mastml.plots.nice_names:mastml.plots.nice_names}]{\sphinxcrossref{\sphinxcode{\sphinxupquote{nice\_names}}}}}()
&

\\
\hline
{\hyperref[\detokenize{api/mastml.plots.nice_range:mastml.plots.nice_range}]{\sphinxcrossref{\sphinxcode{\sphinxupquote{nice\_range}}}}}(lower, upper)
&
Method to create a range of values, including the specified start and end points, with nicely spaced intervals
\\
\hline
{\hyperref[\detokenize{api/mastml.plots.nice_std:mastml.plots.nice_std}]{\sphinxcrossref{\sphinxcode{\sphinxupquote{nice\_std}}}}}(ls)
&
Method to return standard deviation of a list or equivalent array with NaN values
\\
\hline
{\hyperref[\detokenize{api/mastml.plots.plot_stats:mastml.plots.plot_stats}]{\sphinxcrossref{\sphinxcode{\sphinxupquote{plot\_stats}}}}}(fig, stats{[}, x\_align, y\_align, …{]})
&
Method that prints stats onto the plot.
\\
\hline
\sphinxcode{\sphinxupquote{r2\_score}}(y\_true, y\_pred, *{[}, sample\_weight, …{]})
&
R\textasciicircum{}2 (coefficient of determination) regression score function.
\\
\hline
{\hyperref[\detokenize{api/mastml.plots.recursive_max:mastml.plots.recursive_max}]{\sphinxcrossref{\sphinxcode{\sphinxupquote{recursive\_max}}}}}(arr)
&
Method to recursively find the max value of an array of iterables.
\\
\hline
{\hyperref[\detokenize{api/mastml.plots.recursive_max_and_min:mastml.plots.recursive_max_and_min}]{\sphinxcrossref{\sphinxcode{\sphinxupquote{recursive\_max\_and\_min}}}}}(arr)
&
Method to recursively return max and min of values or iterables in array
\\
\hline
{\hyperref[\detokenize{api/mastml.plots.recursive_min:mastml.plots.recursive_min}]{\sphinxcrossref{\sphinxcode{\sphinxupquote{recursive\_min}}}}}(arr)
&
Method to recursively find the min value of an array of iterables.
\\
\hline
{\hyperref[\detokenize{api/mastml.plots.reset_index:mastml.plots.reset_index}]{\sphinxcrossref{\sphinxcode{\sphinxupquote{reset\_index}}}}}(y)
&

\\
\hline
{\hyperref[\detokenize{api/mastml.plots.round_down:mastml.plots.round_down}]{\sphinxcrossref{\sphinxcode{\sphinxupquote{round\_down}}}}}(num, divisor)
&
Method to return a rounded down number
\\
\hline
{\hyperref[\detokenize{api/mastml.plots.round_up:mastml.plots.round_up}]{\sphinxcrossref{\sphinxcode{\sphinxupquote{round\_up}}}}}(num, divisor)
&
Method to return a rounded up number
\\
\hline
{\hyperref[\detokenize{api/mastml.plots.rounder:mastml.plots.rounder}]{\sphinxcrossref{\sphinxcode{\sphinxupquote{rounder}}}}}(delta)
&
Method to obtain number of decimal places to report on plots
\\
\hline
{\hyperref[\detokenize{api/mastml.plots.stat_to_string:mastml.plots.stat_to_string}]{\sphinxcrossref{\sphinxcode{\sphinxupquote{stat\_to\_string}}}}}(name, value, nice\_names)
&
Method that converts a metric object into a string for displaying on a plot
\\
\hline
{\hyperref[\detokenize{api/mastml.plots.trim_array:mastml.plots.trim_array}]{\sphinxcrossref{\sphinxcode{\sphinxupquote{trim\_array}}}}}(arr\_list)
&
Method used to trim a set of arrays to make all arrays the same shape
\\
\hline
\sphinxcode{\sphinxupquote{zoomed\_inset\_axes}}(parent\_axes, zoom{[}, loc, …{]})
&
Create an anchored inset axes by scaling a parent axes.
\\
\hline
\end{longtable}\sphinxatlongtableend\end{savenotes}


\subsubsection{check\_dimensions}
\label{\detokenize{api/mastml.plots.check_dimensions:check-dimensions}}\label{\detokenize{api/mastml.plots.check_dimensions::doc}}\index{check\_dimensions() (in module mastml.plots)@\spxentry{check\_dimensions()}\spxextra{in module mastml.plots}}

\begin{fulllineitems}
\phantomsection\label{\detokenize{api/mastml.plots.check_dimensions:mastml.plots.check_dimensions}}\pysiglinewithargsret{\sphinxcode{\sphinxupquote{mastml.plots.}}\sphinxbfcode{\sphinxupquote{check\_dimensions}}}{\emph{y}}{}
Method to check the dimensions of supplied data. Plotters need data to be 1D and often data is passed in as 2D

Args:
\begin{quote}

y: (numpy array or pd.DataFrame), array or dataframe of data used for plotting
\end{quote}

Returns:
\begin{quote}

y: (pd.Series), series that is now 1D
\end{quote}

\end{fulllineitems}



\subsubsection{get\_divisor}
\label{\detokenize{api/mastml.plots.get_divisor:get-divisor}}\label{\detokenize{api/mastml.plots.get_divisor::doc}}\index{get\_divisor() (in module mastml.plots)@\spxentry{get\_divisor()}\spxextra{in module mastml.plots}}

\begin{fulllineitems}
\phantomsection\label{\detokenize{api/mastml.plots.get_divisor:mastml.plots.get_divisor}}\pysiglinewithargsret{\sphinxcode{\sphinxupquote{mastml.plots.}}\sphinxbfcode{\sphinxupquote{get\_divisor}}}{\emph{high}, \emph{low}}{}
Method to obtain a sensible divisor based on range of two values

Args:
\begin{quote}

high: (float), a max data value

low: (float), a min data value
\end{quote}

Returns:
\begin{quote}

divisor: (float), a number used to make sensible axis ticks
\end{quote}

\end{fulllineitems}



\subsubsection{make\_axis\_same}
\label{\detokenize{api/mastml.plots.make_axis_same:make-axis-same}}\label{\detokenize{api/mastml.plots.make_axis_same::doc}}\index{make\_axis\_same() (in module mastml.plots)@\spxentry{make\_axis\_same()}\spxextra{in module mastml.plots}}

\begin{fulllineitems}
\phantomsection\label{\detokenize{api/mastml.plots.make_axis_same:mastml.plots.make_axis_same}}\pysiglinewithargsret{\sphinxcode{\sphinxupquote{mastml.plots.}}\sphinxbfcode{\sphinxupquote{make\_axis\_same}}}{\emph{ax}, \emph{max1}, \emph{min1}}{}
Method to make the x and y ticks for each axis the same. Useful for parity plots

Args:
\begin{quote}

ax: (matplotlib axis object), a matplotlib axes object

max1: (float), the maximum value of a particular axis

min1: (float), the minimum value of a particular axis
\end{quote}

Returns:
\begin{quote}

None
\end{quote}

\end{fulllineitems}



\subsubsection{make\_fig\_ax}
\label{\detokenize{api/mastml.plots.make_fig_ax:make-fig-ax}}\label{\detokenize{api/mastml.plots.make_fig_ax::doc}}\index{make\_fig\_ax() (in module mastml.plots)@\spxentry{make\_fig\_ax()}\spxextra{in module mastml.plots}}

\begin{fulllineitems}
\phantomsection\label{\detokenize{api/mastml.plots.make_fig_ax:mastml.plots.make_fig_ax}}\pysiglinewithargsret{\sphinxcode{\sphinxupquote{mastml.plots.}}\sphinxbfcode{\sphinxupquote{make\_fig\_ax}}}{\emph{aspect\_ratio=0.5}, \emph{x\_align=0.65}, \emph{left=0.1}}{}
Method to make matplotlib figure and axes objects. Using Object Oriented interface from \sphinxurl{https://matplotlib.org/gallery/api/agg\_oo\_sgskip.html}

Args:
\begin{quote}

aspect\_ratio: (float), aspect ratio for figure and axes creation

x\_align: (float), x position to draw edge of figure. Needed so can display stats alongside plot

left: (float), the leftmost position to draw edge of figure
\end{quote}

Returns:
\begin{quote}

fig: (matplotlib fig object), a matplotlib figure object with the specified aspect ratio

ax: (matplotlib ax object), a matplotlib axes object with the specified aspect ratio
\end{quote}

\end{fulllineitems}



\subsubsection{make\_fig\_ax\_square}
\label{\detokenize{api/mastml.plots.make_fig_ax_square:make-fig-ax-square}}\label{\detokenize{api/mastml.plots.make_fig_ax_square::doc}}\index{make\_fig\_ax\_square() (in module mastml.plots)@\spxentry{make\_fig\_ax\_square()}\spxextra{in module mastml.plots}}

\begin{fulllineitems}
\phantomsection\label{\detokenize{api/mastml.plots.make_fig_ax_square:mastml.plots.make_fig_ax_square}}\pysiglinewithargsret{\sphinxcode{\sphinxupquote{mastml.plots.}}\sphinxbfcode{\sphinxupquote{make\_fig\_ax\_square}}}{\emph{aspect='equal'}, \emph{aspect\_ratio=1}}{}
Method to make square shaped matplotlib figure and axes objects. Using Object Oriented interface from

\sphinxurl{https://matplotlib.org/gallery/api/agg\_oo\_sgskip.html}

Args:
\begin{quote}

aspect: (str), ‘equal’ denotes x and y aspect will be equal (i.e. square)

aspect\_ratio: (float), aspect ratio for figure and axes creation
\end{quote}

Returns:
\begin{quote}

fig: (matplotlib fig object), a matplotlib figure object with the specified aspect ratio

ax: (matplotlib ax object), a matplotlib axes object with the specified aspect ratio
\end{quote}

\end{fulllineitems}



\subsubsection{make\_plots}
\label{\detokenize{api/mastml.plots.make_plots:make-plots}}\label{\detokenize{api/mastml.plots.make_plots::doc}}\index{make\_plots() (in module mastml.plots)@\spxentry{make\_plots()}\spxextra{in module mastml.plots}}

\begin{fulllineitems}
\phantomsection\label{\detokenize{api/mastml.plots.make_plots:mastml.plots.make_plots}}\pysiglinewithargsret{\sphinxcode{\sphinxupquote{mastml.plots.}}\sphinxbfcode{\sphinxupquote{make\_plots}}}{\emph{plots}, \emph{y\_true}, \emph{y\_pred}, \emph{groups}, \emph{dataset\_stdev}, \emph{metrics}, \emph{model}, \emph{residuals}, \emph{model\_errors}, \emph{has\_model\_errors}, \emph{savepath}, \emph{data\_type}, \emph{show\_figure=False}, \emph{recalibrate\_errors=False}, \emph{model\_errors\_cal=None}, \emph{splits\_summary=False}}{}
Helper function to make collections of different types of plots after a single or multiple data splits are evaluated.
\begin{description}
\item[{Args:}] \leavevmode
plots: (list of str), list denoting which types of plots to make. Viable entries are “Scatter”, “Histogram”, “Error”

y\_true: (pd.Series), series containing the true y data

y\_pred: (pd.Series), series containing the predicted y data

groups: (list), list denoting the group label for each data point

dataset\_stdev: (float), the dataset standard deviation

metrics: (list of str), list denoting the metric names to evaluate. See {\color{red}\bfseries{}mastml.metrics.Metrics.metrics\_zoo\_} for full list

model: (mastml.models object), a MAST-ML model object, e.g. SklearnModel or EnsembleModel

residuals: (pd.Series), series containing the residuals (true model errors)

model\_errors: (pd.Series), series containing the as-obtained uncalibrated model errors

has\_model\_errors: (bool), whether the model type used can be subject to UQ and thus have model errors calculated

savepath: (str), string denoting the path to save output to

data\_type: (str), string denoting the data type analyzed, e.g. train, test, leftout

show\_figure: (bool), whether or not the generated figure is output to the notebook screen (default False)

recalibrate\_errors: (bool), whether or not the model errors have been recalibrated (default False)

model\_errors\_cal: (pd.Series), series containing the calibrated predicted model errors

splits\_summary: (bool), whether or not the data used in the plots comes from a collection of many splits (default False), False denotes a single split folder

\item[{Returns:}] \leavevmode
None.

\end{description}

\end{fulllineitems}



\subsubsection{nice\_mean}
\label{\detokenize{api/mastml.plots.nice_mean:nice-mean}}\label{\detokenize{api/mastml.plots.nice_mean::doc}}\index{nice\_mean() (in module mastml.plots)@\spxentry{nice\_mean()}\spxextra{in module mastml.plots}}

\begin{fulllineitems}
\phantomsection\label{\detokenize{api/mastml.plots.nice_mean:mastml.plots.nice_mean}}\pysiglinewithargsret{\sphinxcode{\sphinxupquote{mastml.plots.}}\sphinxbfcode{\sphinxupquote{nice\_mean}}}{\emph{ls}}{}
Method to return mean of a list or equivalent array with NaN values

Args:
\begin{quote}

ls: (list), list of values
\end{quote}

Returns:
\begin{quote}

(numpy array), array containing mean of list of values or NaN if list has no values
\end{quote}

\end{fulllineitems}



\subsubsection{nice\_names}
\label{\detokenize{api/mastml.plots.nice_names:nice-names}}\label{\detokenize{api/mastml.plots.nice_names::doc}}\index{nice\_names() (in module mastml.plots)@\spxentry{nice\_names()}\spxextra{in module mastml.plots}}

\begin{fulllineitems}
\phantomsection\label{\detokenize{api/mastml.plots.nice_names:mastml.plots.nice_names}}\pysiglinewithargsret{\sphinxcode{\sphinxupquote{mastml.plots.}}\sphinxbfcode{\sphinxupquote{nice\_names}}}{}{}
\end{fulllineitems}



\subsubsection{nice\_range}
\label{\detokenize{api/mastml.plots.nice_range:nice-range}}\label{\detokenize{api/mastml.plots.nice_range::doc}}\index{nice\_range() (in module mastml.plots)@\spxentry{nice\_range()}\spxextra{in module mastml.plots}}

\begin{fulllineitems}
\phantomsection\label{\detokenize{api/mastml.plots.nice_range:mastml.plots.nice_range}}\pysiglinewithargsret{\sphinxcode{\sphinxupquote{mastml.plots.}}\sphinxbfcode{\sphinxupquote{nice\_range}}}{\emph{lower}, \emph{upper}}{}
Method to create a range of values, including the specified start and end points, with nicely spaced intervals

Args:
\begin{quote}

lower: (float or int), lower bound of range to create

upper: (float or int), upper bound of range to create
\end{quote}

Returns:
\begin{quote}

(list), list of numerical values in established range
\end{quote}

\end{fulllineitems}



\subsubsection{nice\_std}
\label{\detokenize{api/mastml.plots.nice_std:nice-std}}\label{\detokenize{api/mastml.plots.nice_std::doc}}\index{nice\_std() (in module mastml.plots)@\spxentry{nice\_std()}\spxextra{in module mastml.plots}}

\begin{fulllineitems}
\phantomsection\label{\detokenize{api/mastml.plots.nice_std:mastml.plots.nice_std}}\pysiglinewithargsret{\sphinxcode{\sphinxupquote{mastml.plots.}}\sphinxbfcode{\sphinxupquote{nice\_std}}}{\emph{ls}}{}
Method to return standard deviation of a list or equivalent array with NaN values

Args:
\begin{quote}

ls: (list), list of values
\end{quote}

Returns:
\begin{quote}

(numpy array), array containing standard deviation of list of values or NaN if list has no values
\end{quote}

\end{fulllineitems}



\subsubsection{plot\_stats}
\label{\detokenize{api/mastml.plots.plot_stats:plot-stats}}\label{\detokenize{api/mastml.plots.plot_stats::doc}}\index{plot\_stats() (in module mastml.plots)@\spxentry{plot\_stats()}\spxextra{in module mastml.plots}}

\begin{fulllineitems}
\phantomsection\label{\detokenize{api/mastml.plots.plot_stats:mastml.plots.plot_stats}}\pysiglinewithargsret{\sphinxcode{\sphinxupquote{mastml.plots.}}\sphinxbfcode{\sphinxupquote{plot\_stats}}}{\emph{fig}, \emph{stats}, \emph{x\_align=0.65}, \emph{y\_align=0.9}, \emph{font\_dict=\{\}}, \emph{fontsize=14}}{}
Method that prints stats onto the plot. Goes off screen if they are too long or too many in number.

Args:
\begin{quote}

fig: (matplotlib figure object), a matplotlib figure object

stats: (dict), dict of statistics to be included with a plot

x\_align: (float), float denoting x position of where to align display of stats on a plot

y\_align: (float), float denoting y position of where to align display of stats on a plot

font\_dict: (dict), dict of matplotlib font options to alter display of stats on plot

fontsize: (int), the fontsize of stats to display on plot
\end{quote}

Returns:
\begin{quote}

None
\end{quote}

\end{fulllineitems}



\subsubsection{recursive\_max}
\label{\detokenize{api/mastml.plots.recursive_max:recursive-max}}\label{\detokenize{api/mastml.plots.recursive_max::doc}}\index{recursive\_max() (in module mastml.plots)@\spxentry{recursive\_max()}\spxextra{in module mastml.plots}}

\begin{fulllineitems}
\phantomsection\label{\detokenize{api/mastml.plots.recursive_max:mastml.plots.recursive_max}}\pysiglinewithargsret{\sphinxcode{\sphinxupquote{mastml.plots.}}\sphinxbfcode{\sphinxupquote{recursive\_max}}}{\emph{arr}}{}
Method to recursively find the max value of an array of iterables.

Credit: \sphinxurl{https://www.linkedin.com/pulse/ask-recursion-during-coding-interviews-identify-good-talent-veteanu/}

Args:
\begin{quote}

arr: (numpy array), an array of values or iterables
\end{quote}

Returns:
\begin{quote}

(float), max value in arr
\end{quote}

\end{fulllineitems}



\subsubsection{recursive\_max\_and\_min}
\label{\detokenize{api/mastml.plots.recursive_max_and_min:recursive-max-and-min}}\label{\detokenize{api/mastml.plots.recursive_max_and_min::doc}}\index{recursive\_max\_and\_min() (in module mastml.plots)@\spxentry{recursive\_max\_and\_min()}\spxextra{in module mastml.plots}}

\begin{fulllineitems}
\phantomsection\label{\detokenize{api/mastml.plots.recursive_max_and_min:mastml.plots.recursive_max_and_min}}\pysiglinewithargsret{\sphinxcode{\sphinxupquote{mastml.plots.}}\sphinxbfcode{\sphinxupquote{recursive\_max\_and\_min}}}{\emph{arr}}{}
Method to recursively return max and min of values or iterables in array

Args:
\begin{quote}

arr: (numpy array), an array of values or iterables
\end{quote}

Returns:
\begin{quote}

(tuple), tuple containing max and min of arr
\end{quote}

\end{fulllineitems}



\subsubsection{recursive\_min}
\label{\detokenize{api/mastml.plots.recursive_min:recursive-min}}\label{\detokenize{api/mastml.plots.recursive_min::doc}}\index{recursive\_min() (in module mastml.plots)@\spxentry{recursive\_min()}\spxextra{in module mastml.plots}}

\begin{fulllineitems}
\phantomsection\label{\detokenize{api/mastml.plots.recursive_min:mastml.plots.recursive_min}}\pysiglinewithargsret{\sphinxcode{\sphinxupquote{mastml.plots.}}\sphinxbfcode{\sphinxupquote{recursive\_min}}}{\emph{arr}}{}
Method to recursively find the min value of an array of iterables.

Credit: \sphinxurl{https://www.linkedin.com/pulse/ask-recursion-during-coding-interviews-identify-good-talent-veteanu/}

Args:
\begin{quote}

arr: (numpy array), an array of values or iterables
\end{quote}

Returns:
\begin{quote}

(float), min value in arr
\end{quote}

\end{fulllineitems}



\subsubsection{reset\_index}
\label{\detokenize{api/mastml.plots.reset_index:reset-index}}\label{\detokenize{api/mastml.plots.reset_index::doc}}\index{reset\_index() (in module mastml.plots)@\spxentry{reset\_index()}\spxextra{in module mastml.plots}}

\begin{fulllineitems}
\phantomsection\label{\detokenize{api/mastml.plots.reset_index:mastml.plots.reset_index}}\pysiglinewithargsret{\sphinxcode{\sphinxupquote{mastml.plots.}}\sphinxbfcode{\sphinxupquote{reset\_index}}}{\emph{y}}{}
\end{fulllineitems}



\subsubsection{round\_down}
\label{\detokenize{api/mastml.plots.round_down:round-down}}\label{\detokenize{api/mastml.plots.round_down::doc}}\index{round\_down() (in module mastml.plots)@\spxentry{round\_down()}\spxextra{in module mastml.plots}}

\begin{fulllineitems}
\phantomsection\label{\detokenize{api/mastml.plots.round_down:mastml.plots.round_down}}\pysiglinewithargsret{\sphinxcode{\sphinxupquote{mastml.plots.}}\sphinxbfcode{\sphinxupquote{round\_down}}}{\emph{num}, \emph{divisor}}{}
Method to return a rounded down number

Args:
\begin{quote}

num: (float), a number to round down

divisor: (int), divisor to denote how to round down
\end{quote}

Returns:
\begin{quote}

(float), the rounded-down number
\end{quote}

\end{fulllineitems}



\subsubsection{round\_up}
\label{\detokenize{api/mastml.plots.round_up:round-up}}\label{\detokenize{api/mastml.plots.round_up::doc}}\index{round\_up() (in module mastml.plots)@\spxentry{round\_up()}\spxextra{in module mastml.plots}}

\begin{fulllineitems}
\phantomsection\label{\detokenize{api/mastml.plots.round_up:mastml.plots.round_up}}\pysiglinewithargsret{\sphinxcode{\sphinxupquote{mastml.plots.}}\sphinxbfcode{\sphinxupquote{round\_up}}}{\emph{num}, \emph{divisor}}{}
Method to return a rounded up number

Args:
\begin{quote}

num: (float), a number to round up

divisor: (int), divisor to denote how to round up
\end{quote}

Returns:
\begin{quote}

(float), the rounded-up number
\end{quote}

\end{fulllineitems}



\subsubsection{rounder}
\label{\detokenize{api/mastml.plots.rounder:rounder}}\label{\detokenize{api/mastml.plots.rounder::doc}}\index{rounder() (in module mastml.plots)@\spxentry{rounder()}\spxextra{in module mastml.plots}}

\begin{fulllineitems}
\phantomsection\label{\detokenize{api/mastml.plots.rounder:mastml.plots.rounder}}\pysiglinewithargsret{\sphinxcode{\sphinxupquote{mastml.plots.}}\sphinxbfcode{\sphinxupquote{rounder}}}{\emph{delta}}{}
Method to obtain number of decimal places to report on plots

Args:
\begin{quote}

delta: (float), a float representing the change in two y values on a plot, used to obtain the plot axis spacing size
\end{quote}

Return:
\begin{quote}

(int), an integer denoting the number of decimal places to use
\end{quote}

\end{fulllineitems}



\subsubsection{stat\_to\_string}
\label{\detokenize{api/mastml.plots.stat_to_string:stat-to-string}}\label{\detokenize{api/mastml.plots.stat_to_string::doc}}\index{stat\_to\_string() (in module mastml.plots)@\spxentry{stat\_to\_string()}\spxextra{in module mastml.plots}}

\begin{fulllineitems}
\phantomsection\label{\detokenize{api/mastml.plots.stat_to_string:mastml.plots.stat_to_string}}\pysiglinewithargsret{\sphinxcode{\sphinxupquote{mastml.plots.}}\sphinxbfcode{\sphinxupquote{stat\_to\_string}}}{\emph{name}, \emph{value}, \emph{nice\_names}}{}
Method that converts a metric object into a string for displaying on a plot

Args:
\begin{quote}

name: (str), long name of a stat metric or quantity

value: (float), value of the metric or quantity
\end{quote}

Return:
\begin{quote}

(str), a string of the metric name, adjusted to look nicer for inclusion on a plot
\end{quote}

\end{fulllineitems}



\subsubsection{trim\_array}
\label{\detokenize{api/mastml.plots.trim_array:trim-array}}\label{\detokenize{api/mastml.plots.trim_array::doc}}\index{trim\_array() (in module mastml.plots)@\spxentry{trim\_array()}\spxextra{in module mastml.plots}}

\begin{fulllineitems}
\phantomsection\label{\detokenize{api/mastml.plots.trim_array:mastml.plots.trim_array}}\pysiglinewithargsret{\sphinxcode{\sphinxupquote{mastml.plots.}}\sphinxbfcode{\sphinxupquote{trim\_array}}}{\emph{arr\_list}}{}
Method used to trim a set of arrays to make all arrays the same shape

Args:
\begin{quote}

arr\_list: (list), list of numpy arrays, where arrays are different sizes
\end{quote}

Returns:
\begin{quote}

arr\_list: (), list of trimmed numpy arrays, where arrays are same size
\end{quote}

\end{fulllineitems}



\subsection{Classes}
\label{\detokenize{12_plots:classes}}

\begin{savenotes}\sphinxatlongtablestart\begin{longtable}[c]{\X{1}{2}\X{1}{2}}
\hline

\endfirsthead

\multicolumn{2}{c}%
{\makebox[0pt]{\sphinxtablecontinued{\tablename\ \thetable{} -- continued from previous page}}}\\
\hline

\endhead

\hline
\multicolumn{2}{r}{\makebox[0pt][r]{\sphinxtablecontinued{Continued on next page}}}\\
\endfoot

\endlastfoot

{\hyperref[\detokenize{api/mastml.plots.Error:mastml.plots.Error}]{\sphinxcrossref{\sphinxcode{\sphinxupquote{Error}}}}}
&
Class to make plots related to model error assessment and uncertainty quantification
\\
\hline
\sphinxcode{\sphinxupquote{ErrorUtils}}
&
Collection of functions to conduct error analysis on certain types of models (uncertainty quantification), and prepare residual and model error data for plotting, as well as recalibrate model errors with various methods
\\
\hline
\sphinxcode{\sphinxupquote{Figure}}({[}figsize, dpi, facecolor, edgecolor, …{]})
&
The top level container for all the plot elements.
\\
\hline
\sphinxcode{\sphinxupquote{FigureCanvas}}
&
alias of \sphinxcode{\sphinxupquote{matplotlib.backends.backend\_agg.FigureCanvasAgg}}
\\
\hline
\sphinxcode{\sphinxupquote{FontProperties}}({[}family, style, variant, …{]})
&
A class for storing and manipulating font properties.
\\
\hline
{\hyperref[\detokenize{api/mastml.plots.Histogram:mastml.plots.Histogram}]{\sphinxcrossref{\sphinxcode{\sphinxupquote{Histogram}}}}}
&
Class to generate histogram plots, such as histograms of residual values
\\
\hline
\sphinxcode{\sphinxupquote{Iterable}}
&

\\
\hline
{\hyperref[\detokenize{api/mastml.plots.Line:mastml.plots.Line}]{\sphinxcrossref{\sphinxcode{\sphinxupquote{Line}}}}}
&
Class containing methods for constructing line plots
\\
\hline
\sphinxcode{\sphinxupquote{LinearRegression}}(*{[}, fit\_intercept, …{]})
&
Ordinary least squares Linear Regression.
\\
\hline
\sphinxcode{\sphinxupquote{Metrics}}(metrics\_list{[}, metrics\_type{]})
&
Class containing access to a wide range of metrics from scikit-learn and a number of MAST-ML custom-written metrics
\\
\hline
{\hyperref[\detokenize{api/mastml.plots.Scatter:mastml.plots.Scatter}]{\sphinxcrossref{\sphinxcode{\sphinxupquote{Scatter}}}}}
&
Class to generate scatter plots, such as parity plots showing true vs.
\\
\hline
\sphinxcode{\sphinxupquote{gaussian\_kde}}(dataset{[}, bw\_method, weights{]})
&
Representation of a kernel-density estimate using Gaussian kernels.
\\
\hline
\end{longtable}\sphinxatlongtableend\end{savenotes}


\subsubsection{Error}
\label{\detokenize{api/mastml.plots.Error:error}}\label{\detokenize{api/mastml.plots.Error::doc}}\index{Error (class in mastml.plots)@\spxentry{Error}\spxextra{class in mastml.plots}}

\begin{fulllineitems}
\phantomsection\label{\detokenize{api/mastml.plots.Error:mastml.plots.Error}}\pysigline{\sphinxbfcode{\sphinxupquote{class }}\sphinxcode{\sphinxupquote{mastml.plots.}}\sphinxbfcode{\sphinxupquote{Error}}}
Bases: \sphinxcode{\sphinxupquote{object}}

Class to make plots related to model error assessment and uncertainty quantification
\begin{description}
\item[{Args:}] \leavevmode
None

\item[{Methods:}] \leavevmode\begin{description}
\item[{plot\_normalized\_error: Method to plot the normalized residual errors of a model prediction}] \leavevmode\begin{description}
\item[{Args:}] \leavevmode
residuals: (pd.Series), series containing the true errors (model residuals)

savepath: (str), string denoting the save path to save the figure to

data\_type: (str), string denoting the data type, e.g. train, test, leftout

model\_errors: (pd.Series), series containing the predicted model errors (optional, default None)

show\_figure: (bool), whether or not the generated figure is output to the notebook screen (default False)

\item[{Returns:}] \leavevmode
None

\end{description}

\item[{plot\_cumulative\_normalized\_error: Method to plot the cumulative normalized residual errors of a model prediction}] \leavevmode\begin{description}
\item[{Args:}] \leavevmode
residuals: (pd.Series), series containing the true errors (model residuals)

savepath: (str), string denoting the save path to save the figure to

data\_type: (str), string denoting the data type, e.g. train, test, leftout

model\_errors: (pd.Series), series containing the predicted model errors (optional, default None)

show\_figure: (bool), whether or not the generated figure is output to the notebook screen (default False)

\item[{Returns:}] \leavevmode
None

\end{description}

\item[{plot\_rstat: Method for plotting the r-statistic distribution (true divided by predicted error)}] \leavevmode\begin{description}
\item[{Args:}] \leavevmode
savepath: (str), string denoting the save path to save the figure to

data\_type: (str), string denoting the data type, e.g. train, test, leftout

residuals: (pd.Series), series containing the true errors (model residuals)

model\_errors: (pd.Series), series containing the predicted model errors

show\_figure: (bool), whether or not the generated figure is output to the notebook screen (default False)

is\_calibrated: (bool), whether or not the model errors have been recalibrated (default False)

\item[{Returns:}] \leavevmode
None

\end{description}

\item[{plot\_rstat\_uncal\_cal\_overlay: Method for plotting the r-statistic distribution for two cases together: the as-obtained uncalibrated model errors and calibrated errors}] \leavevmode\begin{description}
\item[{Args:}] \leavevmode
savepath: (str), string denoting the save path to save the figure to

data\_type: (str), string denoting the data type, e.g. train, test, leftout

residuals: (pd.Series), series containing the true errors (model residuals)

model\_errors: (pd.Series), series containing the predicted model errors

model\_errors\_cal: (pd.Series), series containing the calibrated predicted model errors

show\_figure: (bool), whether or not the generated figure is output to the notebook screen (default False)

\item[{Returns:}] \leavevmode
None

\end{description}

\item[{plot\_real\_vs\_predicted\_error: Sometimes called the RvE plot, or residual vs. error plot, this method plots the binned RMS residuals as a function of the binned model errors}] \leavevmode\begin{description}
\item[{Args:}] \leavevmode
savepath: (str), string denoting the save path to save the figure to

model: (mastml.models object), a MAST-ML model object, e.g. SklearnModel or EnsembleModel

data\_type: (str), string denoting the data type, e.g. train, test, leftout

model\_errors: (pd.Series), series containing the predicted model errors

residuals: (pd.Series), series containing the true errors (model residuals)

dataset\_stdev: (float), the standard deviation of the training dataset

show\_figure: (bool), whether or not the generated figure is output to the notebook screen (default False)

is\_calibrated: (bool), whether or not the model errors have been recalibrated (default False)

well\_sampled\_fraction: (float), number denoting whether a bin qualifies as well-sampled or not. Default to 0.025 (2.5\% of total samples). Only affects visuals, not fitting

\item[{Returns:}] \leavevmode
None

\end{description}

\item[{plot\_real\_vs\_predicted\_error\_uncal\_cal\_overlay: Method for making the residual vs. error plot for two cases together: using the as-obtained uncalibrated model errors and calibrated errors}] \leavevmode\begin{description}
\item[{Args:}] \leavevmode
savepath: (str), string denoting the save path to save the figure to

model: (mastml.models object), a MAST-ML model object, e.g. SklearnModel or EnsembleModel

data\_type: (str), string denoting the data type, e.g. train, test, leftout

model\_errors: (pd.Series), series containing the predicted model errors

model\_errors\_cal: (pd.Series), series containing the calibrated predicted model errors

residuals: (pd.Series), series containing the true errors (model residuals)

dataset\_stdev: (float), the standard deviation of the training dataset

show\_figure: (bool), whether or not the generated figure is output to the notebook screen (default False)

well\_sampled\_fraction: (float), number denoting whether a bin qualifies as well-sampled or not. Default to 0.025 (2.5\% of total samples). Only affects visuals, not fitting

\item[{Returns:}] \leavevmode
None

\end{description}

\end{description}

\end{description}
\subsubsection*{Methods Summary}


\begin{savenotes}\sphinxatlongtablestart\begin{longtable}[c]{\X{1}{2}\X{1}{2}}
\hline

\endfirsthead

\multicolumn{2}{c}%
{\makebox[0pt]{\sphinxtablecontinued{\tablename\ \thetable{} -- continued from previous page}}}\\
\hline

\endhead

\hline
\multicolumn{2}{r}{\makebox[0pt][r]{\sphinxtablecontinued{Continued on next page}}}\\
\endfoot

\endlastfoot

{\hyperref[\detokenize{api/mastml.plots.Error:mastml.plots.Error.plot_cumulative_normalized_error}]{\sphinxcrossref{\sphinxcode{\sphinxupquote{plot\_cumulative\_normalized\_error}}}}}(residuals, …)
&

\\
\hline
{\hyperref[\detokenize{api/mastml.plots.Error:mastml.plots.Error.plot_normalized_error}]{\sphinxcrossref{\sphinxcode{\sphinxupquote{plot\_normalized\_error}}}}}(residuals, savepath, …)
&

\\
\hline
{\hyperref[\detokenize{api/mastml.plots.Error:mastml.plots.Error.plot_real_vs_predicted_error}]{\sphinxcrossref{\sphinxcode{\sphinxupquote{plot\_real\_vs\_predicted\_error}}}}}(savepath, …)
&

\\
\hline
{\hyperref[\detokenize{api/mastml.plots.Error:mastml.plots.Error.plot_real_vs_predicted_error_uncal_cal_overlay}]{\sphinxcrossref{\sphinxcode{\sphinxupquote{plot\_real\_vs\_predicted\_error\_uncal\_cal\_overlay}}}}}(…)
&

\\
\hline
{\hyperref[\detokenize{api/mastml.plots.Error:mastml.plots.Error.plot_rstat}]{\sphinxcrossref{\sphinxcode{\sphinxupquote{plot\_rstat}}}}}(savepath, data\_type, residuals, …)
&

\\
\hline
{\hyperref[\detokenize{api/mastml.plots.Error:mastml.plots.Error.plot_rstat_uncal_cal_overlay}]{\sphinxcrossref{\sphinxcode{\sphinxupquote{plot\_rstat\_uncal\_cal\_overlay}}}}}(savepath, …)
&

\\
\hline
\end{longtable}\sphinxatlongtableend\end{savenotes}
\subsubsection*{Methods Documentation}
\index{plot\_cumulative\_normalized\_error() (mastml.plots.Error class method)@\spxentry{plot\_cumulative\_normalized\_error()}\spxextra{mastml.plots.Error class method}}

\begin{fulllineitems}
\phantomsection\label{\detokenize{api/mastml.plots.Error:mastml.plots.Error.plot_cumulative_normalized_error}}\pysiglinewithargsret{\sphinxbfcode{\sphinxupquote{classmethod }}\sphinxbfcode{\sphinxupquote{plot\_cumulative\_normalized\_error}}}{\emph{residuals}, \emph{savepath}, \emph{data\_type}, \emph{model\_errors=None}, \emph{show\_figure=False}}{}
\end{fulllineitems}

\index{plot\_normalized\_error() (mastml.plots.Error class method)@\spxentry{plot\_normalized\_error()}\spxextra{mastml.plots.Error class method}}

\begin{fulllineitems}
\phantomsection\label{\detokenize{api/mastml.plots.Error:mastml.plots.Error.plot_normalized_error}}\pysiglinewithargsret{\sphinxbfcode{\sphinxupquote{classmethod }}\sphinxbfcode{\sphinxupquote{plot\_normalized\_error}}}{\emph{residuals}, \emph{savepath}, \emph{data\_type}, \emph{model\_errors=None}, \emph{show\_figure=False}}{}
\end{fulllineitems}

\index{plot\_real\_vs\_predicted\_error() (mastml.plots.Error class method)@\spxentry{plot\_real\_vs\_predicted\_error()}\spxextra{mastml.plots.Error class method}}

\begin{fulllineitems}
\phantomsection\label{\detokenize{api/mastml.plots.Error:mastml.plots.Error.plot_real_vs_predicted_error}}\pysiglinewithargsret{\sphinxbfcode{\sphinxupquote{classmethod }}\sphinxbfcode{\sphinxupquote{plot\_real\_vs\_predicted\_error}}}{\emph{savepath}, \emph{model}, \emph{data\_type}, \emph{model\_errors}, \emph{residuals}, \emph{dataset\_stdev}, \emph{show\_figure=False}, \emph{is\_calibrated=False}, \emph{well\_sampled\_fraction=0.025}}{}
\end{fulllineitems}

\index{plot\_real\_vs\_predicted\_error\_uncal\_cal\_overlay() (mastml.plots.Error class method)@\spxentry{plot\_real\_vs\_predicted\_error\_uncal\_cal\_overlay()}\spxextra{mastml.plots.Error class method}}

\begin{fulllineitems}
\phantomsection\label{\detokenize{api/mastml.plots.Error:mastml.plots.Error.plot_real_vs_predicted_error_uncal_cal_overlay}}\pysiglinewithargsret{\sphinxbfcode{\sphinxupquote{classmethod }}\sphinxbfcode{\sphinxupquote{plot\_real\_vs\_predicted\_error\_uncal\_cal\_overlay}}}{\emph{savepath}, \emph{model}, \emph{data\_type}, \emph{model\_errors}, \emph{model\_errors\_cal}, \emph{residuals}, \emph{dataset\_stdev}, \emph{show\_figure=False}, \emph{well\_sampled\_fraction=0.025}}{}
\end{fulllineitems}

\index{plot\_rstat() (mastml.plots.Error class method)@\spxentry{plot\_rstat()}\spxextra{mastml.plots.Error class method}}

\begin{fulllineitems}
\phantomsection\label{\detokenize{api/mastml.plots.Error:mastml.plots.Error.plot_rstat}}\pysiglinewithargsret{\sphinxbfcode{\sphinxupquote{classmethod }}\sphinxbfcode{\sphinxupquote{plot\_rstat}}}{\emph{savepath}, \emph{data\_type}, \emph{residuals}, \emph{model\_errors}, \emph{show\_figure=False}, \emph{is\_calibrated=False}}{}
\end{fulllineitems}

\index{plot\_rstat\_uncal\_cal\_overlay() (mastml.plots.Error class method)@\spxentry{plot\_rstat\_uncal\_cal\_overlay()}\spxextra{mastml.plots.Error class method}}

\begin{fulllineitems}
\phantomsection\label{\detokenize{api/mastml.plots.Error:mastml.plots.Error.plot_rstat_uncal_cal_overlay}}\pysiglinewithargsret{\sphinxbfcode{\sphinxupquote{classmethod }}\sphinxbfcode{\sphinxupquote{plot\_rstat\_uncal\_cal\_overlay}}}{\emph{savepath}, \emph{data\_type}, \emph{residuals}, \emph{model\_errors}, \emph{model\_errors\_cal}, \emph{show\_figure=False}}{}
\end{fulllineitems}


\end{fulllineitems}



\subsubsection{Histogram}
\label{\detokenize{api/mastml.plots.Histogram:histogram}}\label{\detokenize{api/mastml.plots.Histogram::doc}}\index{Histogram (class in mastml.plots)@\spxentry{Histogram}\spxextra{class in mastml.plots}}

\begin{fulllineitems}
\phantomsection\label{\detokenize{api/mastml.plots.Histogram:mastml.plots.Histogram}}\pysigline{\sphinxbfcode{\sphinxupquote{class }}\sphinxcode{\sphinxupquote{mastml.plots.}}\sphinxbfcode{\sphinxupquote{Histogram}}}
Bases: \sphinxcode{\sphinxupquote{object}}

Class to generate histogram plots, such as histograms of residual values
\begin{description}
\item[{Args:}] \leavevmode
None

\item[{Methods:}] \leavevmode\begin{description}
\item[{plot\_histogram: method to plot a basic histogram of supplied data}] \leavevmode\begin{description}
\item[{Args:}] \leavevmode
df: (pd.DataFrame), dataframe or series of data to plot as a histogram

savepath: (str), string denoting the save path for the figure image

file\_name: (str), string denoting the character of the file name, e.g. train vs. test

x\_label: (str), string denoting the  property name

show\_figure: (bool), whether or not to show the figure output (e.g. when using Jupyter notebook)

\item[{Returns:}] \leavevmode
None

\end{description}

\item[{plot\_residuals\_histogram: method to plot a histogram of residual values}] \leavevmode\begin{description}
\item[{Args:}] \leavevmode
y\_true: (pd.Series), series of true y data

y\_pred: (pd.Series), series of predicted y data

savepath: (str), string denoting the save path for the figure image

file\_name: (str), string denoting the character of the file name, e.g. train vs. test

show\_figure: (bool), whether or not to show the figure output (e.g. when using Jupyter notebook)

\item[{Returns:}] \leavevmode
None

\end{description}

\item[{\_get\_histogram\_bins: Method to obtain the number of bins to use when plotting a histogram}] \leavevmode\begin{description}
\item[{Args:}] \leavevmode
df: (pandas Series or numpy array), array of y data used to construct histogram

\item[{Returns:}] \leavevmode
num\_bins: (int), the number of bins to use when plotting a histogram

\end{description}

\end{description}

\end{description}
\subsubsection*{Methods Summary}


\begin{savenotes}\sphinxatlongtablestart\begin{longtable}[c]{\X{1}{2}\X{1}{2}}
\hline

\endfirsthead

\multicolumn{2}{c}%
{\makebox[0pt]{\sphinxtablecontinued{\tablename\ \thetable{} -- continued from previous page}}}\\
\hline

\endhead

\hline
\multicolumn{2}{r}{\makebox[0pt][r]{\sphinxtablecontinued{Continued on next page}}}\\
\endfoot

\endlastfoot

{\hyperref[\detokenize{api/mastml.plots.Histogram:mastml.plots.Histogram.plot_histogram}]{\sphinxcrossref{\sphinxcode{\sphinxupquote{plot\_histogram}}}}}(df, savepath, file\_name, x\_label)
&

\\
\hline
{\hyperref[\detokenize{api/mastml.plots.Histogram:mastml.plots.Histogram.plot_residuals_histogram}]{\sphinxcrossref{\sphinxcode{\sphinxupquote{plot\_residuals\_histogram}}}}}(y\_true, y\_pred, …)
&

\\
\hline
\end{longtable}\sphinxatlongtableend\end{savenotes}
\subsubsection*{Methods Documentation}
\index{plot\_histogram() (mastml.plots.Histogram class method)@\spxentry{plot\_histogram()}\spxextra{mastml.plots.Histogram class method}}

\begin{fulllineitems}
\phantomsection\label{\detokenize{api/mastml.plots.Histogram:mastml.plots.Histogram.plot_histogram}}\pysiglinewithargsret{\sphinxbfcode{\sphinxupquote{classmethod }}\sphinxbfcode{\sphinxupquote{plot\_histogram}}}{\emph{df}, \emph{savepath}, \emph{file\_name}, \emph{x\_label}, \emph{show\_figure=False}}{}
\end{fulllineitems}

\index{plot\_residuals\_histogram() (mastml.plots.Histogram class method)@\spxentry{plot\_residuals\_histogram()}\spxextra{mastml.plots.Histogram class method}}

\begin{fulllineitems}
\phantomsection\label{\detokenize{api/mastml.plots.Histogram:mastml.plots.Histogram.plot_residuals_histogram}}\pysiglinewithargsret{\sphinxbfcode{\sphinxupquote{classmethod }}\sphinxbfcode{\sphinxupquote{plot\_residuals\_histogram}}}{\emph{y\_true}, \emph{y\_pred}, \emph{savepath}, \emph{show\_figure=False}, \emph{file\_name='residual\_histogram'}}{}
\end{fulllineitems}


\end{fulllineitems}



\subsubsection{Line}
\label{\detokenize{api/mastml.plots.Line:line}}\label{\detokenize{api/mastml.plots.Line::doc}}\index{Line (class in mastml.plots)@\spxentry{Line}\spxextra{class in mastml.plots}}

\begin{fulllineitems}
\phantomsection\label{\detokenize{api/mastml.plots.Line:mastml.plots.Line}}\pysigline{\sphinxbfcode{\sphinxupquote{class }}\sphinxcode{\sphinxupquote{mastml.plots.}}\sphinxbfcode{\sphinxupquote{Line}}}
Bases: \sphinxcode{\sphinxupquote{object}}

Class containing methods for constructing line plots
\begin{description}
\item[{Args:}] \leavevmode
None

\item[{Methods:}] \leavevmode
plot\_learning\_curve: Method used to plot both data and feature learning curves
\begin{description}
\item[{Args:}] \leavevmode
train\_sizes: (numpy array), array of x-axis values, such as fraction of data used or number of features

train\_mean: (numpy array), array of training data mean values, averaged over some type/number of CV splits

test\_mean: (numpy array), array of test data mean values, averaged over some type/number of CV splits

train\_stdev: (numpy array), array of training data standard deviation values, from some type/number of CV splits

test\_stdev: (numpy array), array of test data standard deviation values, from some type/number of CV splits

score\_name: (str), type of score metric for learning curve plotting; used in y-axis label

learning\_curve\_type: (str), type of learning curve employed: ‘sample\_learning\_curve’ or ‘feature\_learning\_curve’

savepath: (str), path to save the plotted learning curve to

\item[{Returns:}] \leavevmode
None

\end{description}

\end{description}
\subsubsection*{Methods Summary}


\begin{savenotes}\sphinxatlongtablestart\begin{longtable}[c]{\X{1}{2}\X{1}{2}}
\hline

\endfirsthead

\multicolumn{2}{c}%
{\makebox[0pt]{\sphinxtablecontinued{\tablename\ \thetable{} -- continued from previous page}}}\\
\hline

\endhead

\hline
\multicolumn{2}{r}{\makebox[0pt][r]{\sphinxtablecontinued{Continued on next page}}}\\
\endfoot

\endlastfoot

{\hyperref[\detokenize{api/mastml.plots.Line:mastml.plots.Line.plot_learning_curve}]{\sphinxcrossref{\sphinxcode{\sphinxupquote{plot\_learning\_curve}}}}}(train\_sizes, train\_mean, …)
&

\\
\hline
\end{longtable}\sphinxatlongtableend\end{savenotes}
\subsubsection*{Methods Documentation}
\index{plot\_learning\_curve() (mastml.plots.Line class method)@\spxentry{plot\_learning\_curve()}\spxextra{mastml.plots.Line class method}}

\begin{fulllineitems}
\phantomsection\label{\detokenize{api/mastml.plots.Line:mastml.plots.Line.plot_learning_curve}}\pysiglinewithargsret{\sphinxbfcode{\sphinxupquote{classmethod }}\sphinxbfcode{\sphinxupquote{plot\_learning\_curve}}}{\emph{train\_sizes}, \emph{train\_mean}, \emph{test\_mean}, \emph{train\_stdev}, \emph{test\_stdev}, \emph{score\_name}, \emph{learning\_curve\_type}, \emph{savepath}}{}
\end{fulllineitems}


\end{fulllineitems}



\subsubsection{Scatter}
\label{\detokenize{api/mastml.plots.Scatter:scatter}}\label{\detokenize{api/mastml.plots.Scatter::doc}}\index{Scatter (class in mastml.plots)@\spxentry{Scatter}\spxextra{class in mastml.plots}}

\begin{fulllineitems}
\phantomsection\label{\detokenize{api/mastml.plots.Scatter:mastml.plots.Scatter}}\pysigline{\sphinxbfcode{\sphinxupquote{class }}\sphinxcode{\sphinxupquote{mastml.plots.}}\sphinxbfcode{\sphinxupquote{Scatter}}}
Bases: \sphinxcode{\sphinxupquote{object}}

Class to generate scatter plots, such as parity plots showing true vs. predicted data values
\begin{description}
\item[{Args:}] \leavevmode
None

\end{description}

Methods:
\begin{quote}
\begin{description}
\item[{plot\_predicted\_vs\_true: method to plot a parity plot}] \leavevmode\begin{description}
\item[{Args:}] \leavevmode
y\_true: (pd.Series), series of true y data

y\_pred: (pd.Series), series of predicted y data

savepath: (str), string denoting the save path for the figure image

data\_type: (str), string denoting the data type (e.g. train, test, leaveout)

x\_label: (str), string denoting the true and predicted property name

metrics\_list: (list), list of strings of metric names to evaluate and include on the figure

show\_figure: (bool), whether or not to show the figure output (e.g. when using Jupyter notebook)

\item[{Returns:}] \leavevmode
None

\end{description}

\item[{plot\_best\_worst\_split: method to find the best and worst split in an evaluation set and plot them together}] \leavevmode\begin{description}
\item[{Args:}] \leavevmode
savepath: (str), string denoting the save path for the figure image

data\_type: (str), string denoting the data type (e.g. train, test, leaveout)

x\_label: (str), string denoting the true and predicted property name

metrics\_list: (list), list of strings of metric names to evaluate and include on the figure

show\_figure: (bool), whether or not to show the figure output (e.g. when using Jupyter notebook)

\item[{Returns:}] \leavevmode
None

\end{description}

\item[{plot\_best\_worst\_per\_point: method to find all of the best and worst data points from an evaluation set and plot them together}] \leavevmode\begin{description}
\item[{Args:}] \leavevmode
savepath: (str), string denoting the save path for the figure image

data\_type: (str), string denoting the data type (e.g. train, test, leaveout)

x\_label: (str), string denoting the true and predicted property name

metrics\_list: (list), list of strings of metric names to evaluate and include on the figure

show\_figure: (bool), whether or not to show the figure output (e.g. when using Jupyter notebook)

\item[{Returns:}] \leavevmode
None

\end{description}

\item[{plot\_predicted\_vs\_true\_bars: method to plot the average predicted value of each data point from an evaluation set with error bars denoting the standard deviation in predicted values}] \leavevmode\begin{description}
\item[{Args:}] \leavevmode
savepath: (str), string denoting the save path for the figure image

data\_type: (str), string denoting the data type (e.g. train, test, leaveout)

x\_label: (str), string denoting the true and predicted property name

metrics\_list: (list), list of strings of metric names to evaluate and include on the figure

show\_figure: (bool), whether or not to show the figure output (e.g. when using Jupyter notebook)

\item[{Returns:}] \leavevmode
None

\end{description}

\item[{plot\_metric\_vs\_group: method to plot the metric value for each group during e.g. a LeaveOneGroupOut data split}] \leavevmode\begin{description}
\item[{Args:}] \leavevmode
savepath: (str), string denoting the save path for the figure image

data\_type: (str), string denoting the data type (e.g. train, test, leaveout)

show\_figure: (bool), whether or not to show the figure output (e.g. when using Jupyter notebook)

\item[{Returns:}] \leavevmode
None

\end{description}

\end{description}
\end{quote}
\subsubsection*{Methods Summary}


\begin{savenotes}\sphinxatlongtablestart\begin{longtable}[c]{\X{1}{2}\X{1}{2}}
\hline

\endfirsthead

\multicolumn{2}{c}%
{\makebox[0pt]{\sphinxtablecontinued{\tablename\ \thetable{} -- continued from previous page}}}\\
\hline

\endhead

\hline
\multicolumn{2}{r}{\makebox[0pt][r]{\sphinxtablecontinued{Continued on next page}}}\\
\endfoot

\endlastfoot

{\hyperref[\detokenize{api/mastml.plots.Scatter:mastml.plots.Scatter.plot_best_worst_per_point}]{\sphinxcrossref{\sphinxcode{\sphinxupquote{plot\_best\_worst\_per\_point}}}}}(savepath, …{[}, …{]})
&

\\
\hline
{\hyperref[\detokenize{api/mastml.plots.Scatter:mastml.plots.Scatter.plot_best_worst_split}]{\sphinxcrossref{\sphinxcode{\sphinxupquote{plot\_best\_worst\_split}}}}}(savepath, data\_type, …)
&

\\
\hline
{\hyperref[\detokenize{api/mastml.plots.Scatter:mastml.plots.Scatter.plot_metric_vs_group}]{\sphinxcrossref{\sphinxcode{\sphinxupquote{plot\_metric\_vs\_group}}}}}(savepath, data\_type, …)
&

\\
\hline
{\hyperref[\detokenize{api/mastml.plots.Scatter:mastml.plots.Scatter.plot_predicted_vs_true}]{\sphinxcrossref{\sphinxcode{\sphinxupquote{plot\_predicted\_vs\_true}}}}}(y\_true, y\_pred, …)
&

\\
\hline
{\hyperref[\detokenize{api/mastml.plots.Scatter:mastml.plots.Scatter.plot_predicted_vs_true_bars}]{\sphinxcrossref{\sphinxcode{\sphinxupquote{plot\_predicted\_vs\_true\_bars}}}}}(savepath, …{[}, …{]})
&

\\
\hline
\end{longtable}\sphinxatlongtableend\end{savenotes}
\subsubsection*{Methods Documentation}
\index{plot\_best\_worst\_per\_point() (mastml.plots.Scatter class method)@\spxentry{plot\_best\_worst\_per\_point()}\spxextra{mastml.plots.Scatter class method}}

\begin{fulllineitems}
\phantomsection\label{\detokenize{api/mastml.plots.Scatter:mastml.plots.Scatter.plot_best_worst_per_point}}\pysiglinewithargsret{\sphinxbfcode{\sphinxupquote{classmethod }}\sphinxbfcode{\sphinxupquote{plot\_best\_worst\_per\_point}}}{\emph{savepath}, \emph{data\_type}, \emph{x\_label}, \emph{metrics\_list}, \emph{show\_figure=False}}{}
\end{fulllineitems}

\index{plot\_best\_worst\_split() (mastml.plots.Scatter class method)@\spxentry{plot\_best\_worst\_split()}\spxextra{mastml.plots.Scatter class method}}

\begin{fulllineitems}
\phantomsection\label{\detokenize{api/mastml.plots.Scatter:mastml.plots.Scatter.plot_best_worst_split}}\pysiglinewithargsret{\sphinxbfcode{\sphinxupquote{classmethod }}\sphinxbfcode{\sphinxupquote{plot\_best\_worst\_split}}}{\emph{savepath}, \emph{data\_type}, \emph{x\_label}, \emph{metrics\_list}, \emph{show\_figure=False}}{}
\end{fulllineitems}

\index{plot\_metric\_vs\_group() (mastml.plots.Scatter class method)@\spxentry{plot\_metric\_vs\_group()}\spxextra{mastml.plots.Scatter class method}}

\begin{fulllineitems}
\phantomsection\label{\detokenize{api/mastml.plots.Scatter:mastml.plots.Scatter.plot_metric_vs_group}}\pysiglinewithargsret{\sphinxbfcode{\sphinxupquote{classmethod }}\sphinxbfcode{\sphinxupquote{plot\_metric\_vs\_group}}}{\emph{savepath}, \emph{data\_type}, \emph{show\_figure}}{}
\end{fulllineitems}

\index{plot\_predicted\_vs\_true() (mastml.plots.Scatter class method)@\spxentry{plot\_predicted\_vs\_true()}\spxextra{mastml.plots.Scatter class method}}

\begin{fulllineitems}
\phantomsection\label{\detokenize{api/mastml.plots.Scatter:mastml.plots.Scatter.plot_predicted_vs_true}}\pysiglinewithargsret{\sphinxbfcode{\sphinxupquote{classmethod }}\sphinxbfcode{\sphinxupquote{plot\_predicted\_vs\_true}}}{\emph{y\_true}, \emph{y\_pred}, \emph{savepath}, \emph{data\_type}, \emph{x\_label}, \emph{metrics\_list=None}, \emph{show\_figure=False}}{}
\end{fulllineitems}

\index{plot\_predicted\_vs\_true\_bars() (mastml.plots.Scatter class method)@\spxentry{plot\_predicted\_vs\_true\_bars()}\spxextra{mastml.plots.Scatter class method}}

\begin{fulllineitems}
\phantomsection\label{\detokenize{api/mastml.plots.Scatter:mastml.plots.Scatter.plot_predicted_vs_true_bars}}\pysiglinewithargsret{\sphinxbfcode{\sphinxupquote{classmethod }}\sphinxbfcode{\sphinxupquote{plot\_predicted\_vs\_true\_bars}}}{\emph{savepath}, \emph{x\_label}, \emph{data\_type}, \emph{metrics\_list}, \emph{show\_figure=False}}{}
\end{fulllineitems}


\end{fulllineitems}



\subsection{Class Inheritance Diagram}
\label{\detokenize{12_plots:class-inheritance-diagram}}
\sphinxincludegraphics[]{None}


\chapter{Code Documentation: Preprocessing}
\label{\detokenize{13_preprocessing:code-documentation-preprocessing}}\label{\detokenize{13_preprocessing::doc}}

\section{mastml.preprocessing Module}
\label{\detokenize{13_preprocessing:module-mastml.preprocessing}}\label{\detokenize{13_preprocessing:mastml-preprocessing-module}}\index{mastml.preprocessing (module)@\spxentry{mastml.preprocessing}\spxextra{module}}
This module contains methods to perform data preprocessing, such as various standardization/normalization methods
\begin{description}
\item[{BasePreprocessor:}] \leavevmode
Base class that adds some MAST-ML type functionality to other preprocessors. Other preprocessor classes all inherit
this base class

\item[{SklearnPreprocessor:}] \leavevmode
Class that wraps any preprocessor method from scikit-learn (e.g. StandardScaler) to have MAST-ML type functionality

\item[{NoPreprocessor:}] \leavevmode
Class that performs no preprocessing. A preprocessor is needed in the MAST-ML evaluation of data splits. If no
preprocessing is desired, then this NoPreprocessor class is invoked by default

\item[{MeanStdevScaler:}] \leavevmode
Preprocessor class which extends scikit-learn’s StandardScaler to scale the dataset to a particular user-specified
mean and standard deviation value

\end{description}


\subsection{Classes}
\label{\detokenize{13_preprocessing:classes}}

\begin{savenotes}\sphinxatlongtablestart\begin{longtable}[c]{\X{1}{2}\X{1}{2}}
\hline

\endfirsthead

\multicolumn{2}{c}%
{\makebox[0pt]{\sphinxtablecontinued{\tablename\ \thetable{} -- continued from previous page}}}\\
\hline

\endhead

\hline
\multicolumn{2}{r}{\makebox[0pt][r]{\sphinxtablecontinued{Continued on next page}}}\\
\endfoot

\endlastfoot

\sphinxcode{\sphinxupquote{BaseEstimator}}
&
Base class for all estimators in scikit-learn.
\\
\hline
{\hyperref[\detokenize{api/mastml.preprocessing.BasePreprocessor:mastml.preprocessing.BasePreprocessor}]{\sphinxcrossref{\sphinxcode{\sphinxupquote{BasePreprocessor}}}}}(preprocessor{[}, as\_frame{]})
&
Base class to provide new methods beyond sklearn fit\_transform, such as dataframe support and directory management
\\
\hline
{\hyperref[\detokenize{api/mastml.preprocessing.MeanStdevScaler:mastml.preprocessing.MeanStdevScaler}]{\sphinxcrossref{\sphinxcode{\sphinxupquote{MeanStdevScaler}}}}}({[}mean, stdev, as\_frame{]})
&
Class designed to normalize input data to a specified mean and standard deviation
\\
\hline
{\hyperref[\detokenize{api/mastml.preprocessing.NoPreprocessor:mastml.preprocessing.NoPreprocessor}]{\sphinxcrossref{\sphinxcode{\sphinxupquote{NoPreprocessor}}}}}({[}preprocessor, as\_frame{]})
&
Class for having a “null” transform where the output is the same as the input.
\\
\hline
{\hyperref[\detokenize{api/mastml.preprocessing.SklearnPreprocessor:mastml.preprocessing.SklearnPreprocessor}]{\sphinxcrossref{\sphinxcode{\sphinxupquote{SklearnPreprocessor}}}}}(preprocessor{[}, as\_frame{]})
&
Class to wrap any scikit-learn preprocessor, e.g.
\\
\hline
\sphinxcode{\sphinxupquote{TransformerMixin}}
&
Mixin class for all transformers in scikit-learn.
\\
\hline
\sphinxcode{\sphinxupquote{datetime}}(year, month, day{[}, hour{[}, minute{[}, …)
&
The year, month and day arguments are required.
\\
\hline
\end{longtable}\sphinxatlongtableend\end{savenotes}


\subsubsection{BasePreprocessor}
\label{\detokenize{api/mastml.preprocessing.BasePreprocessor:basepreprocessor}}\label{\detokenize{api/mastml.preprocessing.BasePreprocessor::doc}}\index{BasePreprocessor (class in mastml.preprocessing)@\spxentry{BasePreprocessor}\spxextra{class in mastml.preprocessing}}

\begin{fulllineitems}
\phantomsection\label{\detokenize{api/mastml.preprocessing.BasePreprocessor:mastml.preprocessing.BasePreprocessor}}\pysiglinewithargsret{\sphinxbfcode{\sphinxupquote{class }}\sphinxcode{\sphinxupquote{mastml.preprocessing.}}\sphinxbfcode{\sphinxupquote{BasePreprocessor}}}{\emph{preprocessor}, \emph{as\_frame=False}}{}
Bases: \sphinxcode{\sphinxupquote{sklearn.base.BaseEstimator}}, \sphinxcode{\sphinxupquote{sklearn.base.TransformerMixin}}

Base class to provide new methods beyond sklearn fit\_transform, such as dataframe support and directory management
\begin{description}
\item[{Args:}] \leavevmode
preprocessor : a sklearn.preprocessor object, e.g. StandardScaler or mastml.preprocessing object

\item[{Methods:}] \leavevmode\begin{description}
\item[{fit\_transform: method that fits the data to the preprocessor, then transforms it to the preprocessed data}] \leavevmode\begin{description}
\item[{Args:}] \leavevmode
X: (pd.DataFrame), dataframe of X features

y: (pd.Series), series of y target data

\item[{Returns:}] \leavevmode
Transformed data (pd.DataFrame or numpy array based on self.as\_frame)

\end{description}

\item[{evaluate: main method to evaluate a preprocessor, build directory and save data output}] \leavevmode\begin{description}
\item[{Args:}] \leavevmode
X: (pd.DataFrame), dataframe of X features

y: (pd.Series), series of y target data

savepath: (str), string containing main savepath to construct splits for saving output

\item[{Returns:}] \leavevmode
Xnew (pd.DataFrame or numpy array), dataframe or array of the preprocessed X features

\end{description}

\item[{help: method to output key information on class use, e.g. methods and parameters}] \leavevmode\begin{description}
\item[{Args:}] \leavevmode
None

\item[{Returns:}] \leavevmode
None, but outputs help to screen

\end{description}

\item[{\_setup\_savedir: method to create a savedir based on the provided model, splitter, selector names and datetime}] \leavevmode\begin{description}
\item[{Args:}] \leavevmode
model: (mastml.models.SklearnModel or other estimator object), an estimator, e.g. KernelRidge

selector: (mastml.feature\_selectors or other selector object), a selector, e.g. EnsembleModelFeatureSelector

savepath: (str), string designating the savepath

\item[{Returns:}] \leavevmode
splitdir: (str), string containing the new subdirectory to save results to

\end{description}

\end{description}

\end{description}
\subsubsection*{Methods Summary}


\begin{savenotes}\sphinxatlongtablestart\begin{longtable}[c]{\X{1}{2}\X{1}{2}}
\hline

\endfirsthead

\multicolumn{2}{c}%
{\makebox[0pt]{\sphinxtablecontinued{\tablename\ \thetable{} -- continued from previous page}}}\\
\hline

\endhead

\hline
\multicolumn{2}{r}{\makebox[0pt][r]{\sphinxtablecontinued{Continued on next page}}}\\
\endfoot

\endlastfoot

{\hyperref[\detokenize{api/mastml.preprocessing.BasePreprocessor:mastml.preprocessing.BasePreprocessor.evaluate}]{\sphinxcrossref{\sphinxcode{\sphinxupquote{evaluate}}}}}(X{[}, y, savepath, file\_name, …{]})
&

\\
\hline
{\hyperref[\detokenize{api/mastml.preprocessing.BasePreprocessor:mastml.preprocessing.BasePreprocessor.fit}]{\sphinxcrossref{\sphinxcode{\sphinxupquote{fit}}}}}(X)
&

\\
\hline
{\hyperref[\detokenize{api/mastml.preprocessing.BasePreprocessor:mastml.preprocessing.BasePreprocessor.fit_transform}]{\sphinxcrossref{\sphinxcode{\sphinxupquote{fit\_transform}}}}}(X{[}, y{]})
&
Fit to data, then transform it.
\\
\hline
{\hyperref[\detokenize{api/mastml.preprocessing.BasePreprocessor:mastml.preprocessing.BasePreprocessor.help}]{\sphinxcrossref{\sphinxcode{\sphinxupquote{help}}}}}()
&

\\
\hline
{\hyperref[\detokenize{api/mastml.preprocessing.BasePreprocessor:mastml.preprocessing.BasePreprocessor.inverse_transform}]{\sphinxcrossref{\sphinxcode{\sphinxupquote{inverse\_transform}}}}}(X)
&

\\
\hline
{\hyperref[\detokenize{api/mastml.preprocessing.BasePreprocessor:mastml.preprocessing.BasePreprocessor.transform}]{\sphinxcrossref{\sphinxcode{\sphinxupquote{transform}}}}}(X)
&

\\
\hline
\end{longtable}\sphinxatlongtableend\end{savenotes}
\subsubsection*{Methods Documentation}
\index{evaluate() (mastml.preprocessing.BasePreprocessor method)@\spxentry{evaluate()}\spxextra{mastml.preprocessing.BasePreprocessor method}}

\begin{fulllineitems}
\phantomsection\label{\detokenize{api/mastml.preprocessing.BasePreprocessor:mastml.preprocessing.BasePreprocessor.evaluate}}\pysiglinewithargsret{\sphinxbfcode{\sphinxupquote{evaluate}}}{\emph{X}, \emph{y=None}, \emph{savepath=None}, \emph{file\_name=''}, \emph{make\_new\_dir=False}}{}
\end{fulllineitems}

\index{fit() (mastml.preprocessing.BasePreprocessor method)@\spxentry{fit()}\spxextra{mastml.preprocessing.BasePreprocessor method}}

\begin{fulllineitems}
\phantomsection\label{\detokenize{api/mastml.preprocessing.BasePreprocessor:mastml.preprocessing.BasePreprocessor.fit}}\pysiglinewithargsret{\sphinxbfcode{\sphinxupquote{fit}}}{\emph{X}}{}
\end{fulllineitems}

\index{fit\_transform() (mastml.preprocessing.BasePreprocessor method)@\spxentry{fit\_transform()}\spxextra{mastml.preprocessing.BasePreprocessor method}}

\begin{fulllineitems}
\phantomsection\label{\detokenize{api/mastml.preprocessing.BasePreprocessor:mastml.preprocessing.BasePreprocessor.fit_transform}}\pysiglinewithargsret{\sphinxbfcode{\sphinxupquote{fit\_transform}}}{\emph{X}, \emph{y=None}, \emph{**fit\_params}}{}
Fit to data, then transform it.

Fits transformer to \sphinxtitleref{X} and \sphinxtitleref{y} with optional parameters \sphinxtitleref{fit\_params}
and returns a transformed version of \sphinxtitleref{X}.
\begin{description}
\item[{X}] \leavevmode{[}array-like of shape (n\_samples, n\_features){]}
Input samples.

\item[{y}] \leavevmode{[}array-like of shape (n\_samples,) or (n\_samples, n\_outputs),                 default=None{]}
Target values (None for unsupervised transformations).

\item[{{\color{red}\bfseries{}**}fit\_params}] \leavevmode{[}dict{]}
Additional fit parameters.

\end{description}
\begin{description}
\item[{X\_new}] \leavevmode{[}ndarray array of shape (n\_samples, n\_features\_new){]}
Transformed array.

\end{description}

\end{fulllineitems}

\index{help() (mastml.preprocessing.BasePreprocessor method)@\spxentry{help()}\spxextra{mastml.preprocessing.BasePreprocessor method}}

\begin{fulllineitems}
\phantomsection\label{\detokenize{api/mastml.preprocessing.BasePreprocessor:mastml.preprocessing.BasePreprocessor.help}}\pysiglinewithargsret{\sphinxbfcode{\sphinxupquote{help}}}{}{}
\end{fulllineitems}

\index{inverse\_transform() (mastml.preprocessing.BasePreprocessor method)@\spxentry{inverse\_transform()}\spxextra{mastml.preprocessing.BasePreprocessor method}}

\begin{fulllineitems}
\phantomsection\label{\detokenize{api/mastml.preprocessing.BasePreprocessor:mastml.preprocessing.BasePreprocessor.inverse_transform}}\pysiglinewithargsret{\sphinxbfcode{\sphinxupquote{inverse\_transform}}}{\emph{X}}{}
\end{fulllineitems}

\index{transform() (mastml.preprocessing.BasePreprocessor method)@\spxentry{transform()}\spxextra{mastml.preprocessing.BasePreprocessor method}}

\begin{fulllineitems}
\phantomsection\label{\detokenize{api/mastml.preprocessing.BasePreprocessor:mastml.preprocessing.BasePreprocessor.transform}}\pysiglinewithargsret{\sphinxbfcode{\sphinxupquote{transform}}}{\emph{X}}{}
\end{fulllineitems}


\end{fulllineitems}



\subsubsection{MeanStdevScaler}
\label{\detokenize{api/mastml.preprocessing.MeanStdevScaler:meanstdevscaler}}\label{\detokenize{api/mastml.preprocessing.MeanStdevScaler::doc}}\index{MeanStdevScaler (class in mastml.preprocessing)@\spxentry{MeanStdevScaler}\spxextra{class in mastml.preprocessing}}

\begin{fulllineitems}
\phantomsection\label{\detokenize{api/mastml.preprocessing.MeanStdevScaler:mastml.preprocessing.MeanStdevScaler}}\pysiglinewithargsret{\sphinxbfcode{\sphinxupquote{class }}\sphinxcode{\sphinxupquote{mastml.preprocessing.}}\sphinxbfcode{\sphinxupquote{MeanStdevScaler}}}{\emph{mean=0}, \emph{stdev=1}, \emph{as\_frame=False}}{}
Bases: {\hyperref[\detokenize{api/mastml.preprocessing.BasePreprocessor:mastml.preprocessing.BasePreprocessor}]{\sphinxcrossref{\sphinxcode{\sphinxupquote{mastml.preprocessing.BasePreprocessor}}}}}

Class designed to normalize input data to a specified mean and standard deviation
\begin{description}
\item[{Args:}] \leavevmode
mean: (int/float), specified normalized mean of the data

stdev: (int/float), specified normalized standard deviation of the data

\item[{Methods:}] \leavevmode\begin{description}
\item[{fit: Obtains initial mean and stdev of data}] \leavevmode\begin{description}
\item[{Args:}] \leavevmode
df: (dataframe), dataframe of values to be normalized

\item[{Returns:}] \leavevmode
(self, the object instance)

\end{description}

\item[{transform: Normalizes the data to new mean and stdev values}] \leavevmode\begin{description}
\item[{Args:}] \leavevmode
df: (dataframe), dataframe of values to be normalized

\item[{Returns:}] \leavevmode
(dataframe), dataframe containing re-normalized data and any data that wasn’t normalized

\end{description}

\item[{inverse\_transform: Un-normalizes the data to the old mean and stdev values}] \leavevmode\begin{description}
\item[{Args:}] \leavevmode
df: (dataframe), dataframe of values to be un-normalized

\item[{Returns:}] \leavevmode
(dataframe), dataframe containing un-normalized data and any data that wasn’t normalized

\end{description}

\end{description}

\end{description}
\subsubsection*{Methods Summary}


\begin{savenotes}\sphinxatlongtablestart\begin{longtable}[c]{\X{1}{2}\X{1}{2}}
\hline

\endfirsthead

\multicolumn{2}{c}%
{\makebox[0pt]{\sphinxtablecontinued{\tablename\ \thetable{} -- continued from previous page}}}\\
\hline

\endhead

\hline
\multicolumn{2}{r}{\makebox[0pt][r]{\sphinxtablecontinued{Continued on next page}}}\\
\endfoot

\endlastfoot

{\hyperref[\detokenize{api/mastml.preprocessing.MeanStdevScaler:mastml.preprocessing.MeanStdevScaler.fit_transform}]{\sphinxcrossref{\sphinxcode{\sphinxupquote{fit\_transform}}}}}(X{[}, y{]})
&
Fit to data, then transform it.
\\
\hline
\end{longtable}\sphinxatlongtableend\end{savenotes}
\subsubsection*{Methods Documentation}
\index{fit\_transform() (mastml.preprocessing.MeanStdevScaler method)@\spxentry{fit\_transform()}\spxextra{mastml.preprocessing.MeanStdevScaler method}}

\begin{fulllineitems}
\phantomsection\label{\detokenize{api/mastml.preprocessing.MeanStdevScaler:mastml.preprocessing.MeanStdevScaler.fit_transform}}\pysiglinewithargsret{\sphinxbfcode{\sphinxupquote{fit\_transform}}}{\emph{X}, \emph{y=None}, \emph{**fit\_params}}{}
Fit to data, then transform it.

Fits transformer to \sphinxtitleref{X} and \sphinxtitleref{y} with optional parameters \sphinxtitleref{fit\_params}
and returns a transformed version of \sphinxtitleref{X}.
\begin{description}
\item[{X}] \leavevmode{[}array-like of shape (n\_samples, n\_features){]}
Input samples.

\item[{y}] \leavevmode{[}array-like of shape (n\_samples,) or (n\_samples, n\_outputs),                 default=None{]}
Target values (None for unsupervised transformations).

\item[{{\color{red}\bfseries{}**}fit\_params}] \leavevmode{[}dict{]}
Additional fit parameters.

\end{description}
\begin{description}
\item[{X\_new}] \leavevmode{[}ndarray array of shape (n\_samples, n\_features\_new){]}
Transformed array.

\end{description}

\end{fulllineitems}


\end{fulllineitems}



\subsubsection{NoPreprocessor}
\label{\detokenize{api/mastml.preprocessing.NoPreprocessor:nopreprocessor}}\label{\detokenize{api/mastml.preprocessing.NoPreprocessor::doc}}\index{NoPreprocessor (class in mastml.preprocessing)@\spxentry{NoPreprocessor}\spxextra{class in mastml.preprocessing}}

\begin{fulllineitems}
\phantomsection\label{\detokenize{api/mastml.preprocessing.NoPreprocessor:mastml.preprocessing.NoPreprocessor}}\pysiglinewithargsret{\sphinxbfcode{\sphinxupquote{class }}\sphinxcode{\sphinxupquote{mastml.preprocessing.}}\sphinxbfcode{\sphinxupquote{NoPreprocessor}}}{\emph{preprocessor=None}, \emph{as\_frame=False}}{}
Bases: {\hyperref[\detokenize{api/mastml.preprocessing.BasePreprocessor:mastml.preprocessing.BasePreprocessor}]{\sphinxcrossref{\sphinxcode{\sphinxupquote{mastml.preprocessing.BasePreprocessor}}}}}

Class for having a “null” transform where the output is the same as the input. Needed by MAST-ML as a placeholder if
certain workflow aspects are not performed.

See BasePreprocessor for information on args and methods
\subsubsection*{Methods Summary}


\begin{savenotes}\sphinxatlongtablestart\begin{longtable}[c]{\X{1}{2}\X{1}{2}}
\hline

\endfirsthead

\multicolumn{2}{c}%
{\makebox[0pt]{\sphinxtablecontinued{\tablename\ \thetable{} -- continued from previous page}}}\\
\hline

\endhead

\hline
\multicolumn{2}{r}{\makebox[0pt][r]{\sphinxtablecontinued{Continued on next page}}}\\
\endfoot

\endlastfoot

{\hyperref[\detokenize{api/mastml.preprocessing.NoPreprocessor:mastml.preprocessing.NoPreprocessor.fit}]{\sphinxcrossref{\sphinxcode{\sphinxupquote{fit}}}}}(X)
&

\\
\hline
{\hyperref[\detokenize{api/mastml.preprocessing.NoPreprocessor:mastml.preprocessing.NoPreprocessor.fit_transform}]{\sphinxcrossref{\sphinxcode{\sphinxupquote{fit\_transform}}}}}(X{[}, y{]})
&
Fit to data, then transform it.
\\
\hline
{\hyperref[\detokenize{api/mastml.preprocessing.NoPreprocessor:mastml.preprocessing.NoPreprocessor.transform}]{\sphinxcrossref{\sphinxcode{\sphinxupquote{transform}}}}}(X)
&

\\
\hline
\end{longtable}\sphinxatlongtableend\end{savenotes}
\subsubsection*{Methods Documentation}
\index{fit() (mastml.preprocessing.NoPreprocessor method)@\spxentry{fit()}\spxextra{mastml.preprocessing.NoPreprocessor method}}

\begin{fulllineitems}
\phantomsection\label{\detokenize{api/mastml.preprocessing.NoPreprocessor:mastml.preprocessing.NoPreprocessor.fit}}\pysiglinewithargsret{\sphinxbfcode{\sphinxupquote{fit}}}{\emph{X}}{}
\end{fulllineitems}

\index{fit\_transform() (mastml.preprocessing.NoPreprocessor method)@\spxentry{fit\_transform()}\spxextra{mastml.preprocessing.NoPreprocessor method}}

\begin{fulllineitems}
\phantomsection\label{\detokenize{api/mastml.preprocessing.NoPreprocessor:mastml.preprocessing.NoPreprocessor.fit_transform}}\pysiglinewithargsret{\sphinxbfcode{\sphinxupquote{fit\_transform}}}{\emph{X}, \emph{y=None}, \emph{**fit\_params}}{}
Fit to data, then transform it.

Fits transformer to \sphinxtitleref{X} and \sphinxtitleref{y} with optional parameters \sphinxtitleref{fit\_params}
and returns a transformed version of \sphinxtitleref{X}.
\begin{description}
\item[{X}] \leavevmode{[}array-like of shape (n\_samples, n\_features){]}
Input samples.

\item[{y}] \leavevmode{[}array-like of shape (n\_samples,) or (n\_samples, n\_outputs),                 default=None{]}
Target values (None for unsupervised transformations).

\item[{{\color{red}\bfseries{}**}fit\_params}] \leavevmode{[}dict{]}
Additional fit parameters.

\end{description}
\begin{description}
\item[{X\_new}] \leavevmode{[}ndarray array of shape (n\_samples, n\_features\_new){]}
Transformed array.

\end{description}

\end{fulllineitems}

\index{transform() (mastml.preprocessing.NoPreprocessor method)@\spxentry{transform()}\spxextra{mastml.preprocessing.NoPreprocessor method}}

\begin{fulllineitems}
\phantomsection\label{\detokenize{api/mastml.preprocessing.NoPreprocessor:mastml.preprocessing.NoPreprocessor.transform}}\pysiglinewithargsret{\sphinxbfcode{\sphinxupquote{transform}}}{\emph{X}}{}
\end{fulllineitems}


\end{fulllineitems}



\subsubsection{SklearnPreprocessor}
\label{\detokenize{api/mastml.preprocessing.SklearnPreprocessor:sklearnpreprocessor}}\label{\detokenize{api/mastml.preprocessing.SklearnPreprocessor::doc}}\index{SklearnPreprocessor (class in mastml.preprocessing)@\spxentry{SklearnPreprocessor}\spxextra{class in mastml.preprocessing}}

\begin{fulllineitems}
\phantomsection\label{\detokenize{api/mastml.preprocessing.SklearnPreprocessor:mastml.preprocessing.SklearnPreprocessor}}\pysiglinewithargsret{\sphinxbfcode{\sphinxupquote{class }}\sphinxcode{\sphinxupquote{mastml.preprocessing.}}\sphinxbfcode{\sphinxupquote{SklearnPreprocessor}}}{\emph{preprocessor}, \emph{as\_frame=False}, \emph{**kwargs}}{}
Bases: {\hyperref[\detokenize{api/mastml.preprocessing.BasePreprocessor:mastml.preprocessing.BasePreprocessor}]{\sphinxcrossref{\sphinxcode{\sphinxupquote{mastml.preprocessing.BasePreprocessor}}}}}

Class to wrap any scikit-learn preprocessor, e.g. StandardScaler
\begin{description}
\item[{Args:}] \leavevmode
preprocessor (str): name of a sklearn.preprocessor object, e.g. StandardScaler

as\_frame (bool): whether to return data as a dataframe

kwargs : key word arguments for the sklearn.preprocessor object

\end{description}

Methods:
\begin{quote}

See documentation of BasePreprocessor
\end{quote}

\end{fulllineitems}



\subsection{Class Inheritance Diagram}
\label{\detokenize{13_preprocessing:class-inheritance-diagram}}
\sphinxincludegraphics[]{None}


\chapter{Indices and tables}
\label{\detokenize{index:indices-and-tables}}\begin{itemize}
\item {} 
\DUrole{xref,std,std-ref}{genindex}

\item {} 
\DUrole{xref,std,std-ref}{modindex}

\item {} 
\DUrole{xref,std,std-ref}{search}

\end{itemize}


\renewcommand{\indexname}{Python Module Index}
\begin{sphinxtheindex}
\let\bigletter\sphinxstyleindexlettergroup
\bigletter{m}
\item\relax\sphinxstyleindexentry{mastml.data\_cleaning}\sphinxstyleindexpageref{1_data_cleaning:\detokenize{module-mastml.data_cleaning}}
\item\relax\sphinxstyleindexentry{mastml.data\_splitters}\sphinxstyleindexpageref{2_data_splitters:\detokenize{module-mastml.data_splitters}}
\item\relax\sphinxstyleindexentry{mastml.datasets}\sphinxstyleindexpageref{3_datasets:\detokenize{module-mastml.datasets}}
\item\relax\sphinxstyleindexentry{mastml.error\_analysis}\sphinxstyleindexpageref{4_error_analysis:\detokenize{module-mastml.error_analysis}}
\item\relax\sphinxstyleindexentry{mastml.feature\_generators}\sphinxstyleindexpageref{5_feature_generators:\detokenize{module-mastml.feature_generators}}
\item\relax\sphinxstyleindexentry{mastml.feature\_selectors}\sphinxstyleindexpageref{6_feature_selectors:\detokenize{module-mastml.feature_selectors}}
\item\relax\sphinxstyleindexentry{mastml.hyper\_opt}\sphinxstyleindexpageref{7_hyper_opt:\detokenize{module-mastml.hyper_opt}}
\item\relax\sphinxstyleindexentry{mastml.learning\_curve}\sphinxstyleindexpageref{8_learning_curve:\detokenize{module-mastml.learning_curve}}
\item\relax\sphinxstyleindexentry{mastml.mastml}\sphinxstyleindexpageref{9_mastml:\detokenize{module-mastml.mastml}}
\item\relax\sphinxstyleindexentry{mastml.metrics}\sphinxstyleindexpageref{10_metrics:\detokenize{module-mastml.metrics}}
\item\relax\sphinxstyleindexentry{mastml.models}\sphinxstyleindexpageref{11_models:\detokenize{module-mastml.models}}
\item\relax\sphinxstyleindexentry{mastml.plots}\sphinxstyleindexpageref{12_plots:\detokenize{module-mastml.plots}}
\item\relax\sphinxstyleindexentry{mastml.preprocessing}\sphinxstyleindexpageref{13_preprocessing:\detokenize{module-mastml.preprocessing}}
\end{sphinxtheindex}

\renewcommand{\indexname}{Index}
\printindex
\end{document}